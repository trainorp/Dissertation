
\section{Dust}
 
This is an example of how to continue a chapter with additional sections by
using addtional files.  

In order to determine the temperature of the dust in these regions, 
the Spectral Energy Distribution (SED)  must be fitted to a thermal 
emission curve. The fitting function we used is based on an optically 
thin modified blackbody. A true black body absorbs all incident 
radiation and emits thermal continuum according to the 
Planck function \cite{irwin2007}.

\begin{equation}
B_{\nu}(T) = \frac{2h\nu^3}{c^2} \frac{1}{e^{\frac{h\nu}{kT}}-1}
\end{equation}

\noindent However, dust does not behave like a true black body, 
as some radiation is reflected. As a result, a modified black body 
curve is used to account for these effects. 

A detailed understanding of dust emission and absorption  requires accounting
for a diversity of factors including composition and grain size. To get a
general idea of dust temperature, however, we can utilize a generic fitting
function used by the Planck consortium.

\begin{equation}
I_{\nu} = \tau_{\nu_0} \cdot (\frac{\nu}{\nu_0})^{\beta} \cdot B_{\nu}(T)
\end{equation}

\noindent Here, I$_{\nu}$, $\tau_{\nu_0}$ is the optical depth at frequency
$\nu_0$, $\beta$ is the spectral emissivity index, and B$_{\nu}$(T) is the
Planck function \cite{planck2011thermal}. In keeping with assumptions from
Planck studies for generic dust clouds, a spectral emissivity index value of
$\beta$ = 1.5 was adopted for all wavelengths greater than 250 $\mu$m and
$\beta$ was taken to be unity below this wavelength. These are typical
assumptions based on extragalactic surveys \cite{planck2015}.

