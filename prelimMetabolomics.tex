\section{Metabolism} [Need citations]

Metabolism is broadly defined as the chemical reactions that enable life [cite]. The metabolic reactions that sustain life can be categorized as catabolic reactions, or the reactions that entail breaking down and oxidizing molecules to provide energy or substrates for other reactions, and anabolic reactions in which complex molecules are synthesized. Series of coupled reactions are referred to as pathways. Anabolic pathways are characterized by endergonic processes as the synthesis of macromolecules from smaller molecules requires energy. Conversely catabolic pathways entail the break down of larger molecules into smaller molecules for use either as inputs in anabolic reactions or for the release of free energy via oxidation and reduction reactions. 

An example catabolic process is the tricarboxylic acid (TCA) cycle in which an input molecule of Acetyl-CoA (from the catabolism of glucose, fats, or proteins) enters into a sequence of reactions which produces three molecules of NADH, one molecule of FADH\textsubscript{2}, and one molecule of GTP. The molecules of NADH and FADH\textsubscript{2} produced during TCA cycle also serve as substrates for the electron transport chain which generates additional molecules of ATP. 

An example anabolic pathway is the cholesterol biosynthesis pathway. In this pathway,  acetyl-CoA molecules as a substrate are chemically transformed through a sequence of reactions with intermediates including HMG-CoA, mevalonate, isopentenyl phosphate, squalene, lanosterol, and finally cholesterol. The cholesterol biosynthesis pathway entails over 20 steps, and requires ATP as an energetic input and the cofactors NADH and NADPH. 


\section{Analytical chemistry}