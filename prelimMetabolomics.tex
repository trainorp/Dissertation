\begin{DoubleSpace*}
\section{Metabolism} [Need citations]
Metabolism is broadly defined as the chemical reactions that enable life [cite]. The metabolic reactions that sustain life can be categorized as catabolic reactions, or the reactions that entail breaking down and oxidizing molecules to provide energy or substrates for other reactions, and anabolic reactions in which complex molecules are synthesized. Series of coupled reactions are referred to as pathways. Anabolic pathways are characterized by endergonic processes as the synthesis of macromolecules from smaller molecules requires energy. Conversely catabolic pathways entail the break down of larger molecules into smaller molecules for use either as inputs in anabolic reactions or for the release of free energy via oxidation and reduction reactions. 
\begin{figure}[ht!]
	\resizebox{\textwidth}{!}{\includegraphics*{./Plots/tca.png}}
	\caption[TCA cycle pathway diagram]{ Diagram of the tricarboxylic acid (TCA) cycle in \emph{Homo sapiens} [add citation].\label{fig:tca} }
\end{figure}

\begin{figure}[ht!]
	\resizebox{\textwidth}{!}{\includegraphics*{./Plots/cholesterol.png}}
	\caption[Cholesterol biosynthesis pathway diagram]{ Diagram of the cholesterol biosynthesis pathway in \emph{Homo sapiens}. [add citation] \label{fig:chol} }
\end{figure}

An example catabolic process is the tricarboxylic acid (TCA) cycle which is illustrated in Fig.~\ref{fig:tca}. In the TCA cycle an input molecule of Acetyl-CoA (from the catabolism of glucose, fats, or proteins) enters into a sequence of reactions which produces three molecules of NADH, one molecule of FADH\textsubscript{2}, and one molecule of GTP. The molecules of NADH and FADH\textsubscript{2} produced during TCA cycle also serve as substrates for the electron transport chain which generates additional molecules of ATP. 

An example anabolic pathway is the cholesterol biosynthesis pathway shown in Fig.~\ref{fig:tca}. In this pathway, acetyl-CoA molecules as a substrate are chemically transformed through a sequence of reactions with intermediates including HMG-CoA, mevalonate, isopentenyl phosphate, squalene, lanosterol, and finally cholesterol. The cholesterol biosynthesis pathway entails over 20 steps, and requires ATP as an energetic input and the cofactors NADH and NADPH. 

\section{Complexity of a single metabolic process}
While many of the biochemical processes of metabolism, including glucose catabolism, glycogen metabolism, gluconeogenesis, lipid metabolism, TCA cycle, electron transport chain, nucleotide metabolism, and protein synthesis  take place in the cell, many of the signaling processes that control metabolism as well as many systems of transport are extracellular \cite{voet2013}. As an example, consider lipid metabolism. While the catabolic process of breaking down fatty acids to yield energetic substrates, $\beta$-oxidation, takes place in the mitochondrial matrix of a cell, many of the processes prior to the point of $\beta$-oxidation are extracellular or take place in distinct cells from the cell in which a specific fatty acid molecule is being broken down. Within adipocytes (specialized cells for the storage of fatty acids), fatty acids are stored as triacylglycerols. When activated by hormone sensitive lipase, triacylglycerols are hydrolyzed yielding free fatty acids which are released into the bloodstream. These fatty acid molecules may be transported in blood in complex with the protein albumin, and may be taken up by other cells (especially by hepatocytes in the liver) for fatty acid oxidation. In addition to the oxidation of fatty acids to yield citrate and acetyl-CoA, cells (especially hepatocytes) also synthesize fatty acids for storage to meet future energetic needs. Newly synthesized fatty acid molecules may be transported in the bloodstream in the form of triacylglycerols within chylomicrons as well as in VLDL. These molecules can be utilized to transport fatty acids for storage in adipocytes. 

Fatty acid metabolism is largely controlled by hormonal signaling \cite{voet2013}. The activity of the enzyme hormone sensitive lipase is modulated by phosphorylation and dephosphorylation by the enzyme PKA. The activation of PKA is in turn regulated by cAMP concentrations which are influenced by multiple hormonal signals including epinephrine, norepinephrine, and glucagon. In addition to modulating the rate of triacylglycerol lipase in liver and muscle cells, PKA also is an important regulator of fatty acid synthesis. In this way the control of fatty acid breakdown and synthesis is controlled in a coupled manner (when the rate of one process increases, the rate of the other process decreases).

In summary then, a single metabolic process may involve biochemical reactions that take place in multiple cells (of different types), across multiple tissues, with metabolic control influenced by paracrine signaling, endocrine signaling, cellular metabolite concentrations, and extracellular metabolite concentrations.

\section{Dysregulation of metabolism and disease}
Dysregulation of metabolic processes is a prominent feature of human diseases. As a disease, cancer provides a salient example of the relationship between metabolism and disease. While cancer represents a broad class of diseases characterized by abnormal cell growth affecting multiple tissue and organ types, the reprogramming of cellular metabolism is a common feature of all cancers \cite{pavlova2016}. For cancer cells to continue to grow and divide, new sources of nutrients are required and the more efficient use of nutrients in the tumor environment is required. These requirements lead to metabolic changes associated with cancer that have been described \cite{pavlova2016} as six features: (1) changes in the glucose and amino acid uptake by cancer cells and tumor tissue, (2) ``opportunistic'' manners of nutrient utilization, (3) increased utilization of glycolysis and the TCA cycle intermediates for synthesizing intermediates as well as for increasing the production of cofactors, (4) increased requirement of nitrogen, (5) changes in gene expression that are modulated by metabolite concentrations, and (6) abnormal metabolic interactions between cells and the surrounding environment. 

[More on Cancer and metabolism] 

In addition to cancer, metabolic syndromes are marked by dysregulation of metabolism. Changes in the metabolism of branch-chain amino acids have emerged as hallmarks of metabolic syndromes, the development of type 2 diabetes, and insulin resistance in general \cite{newgard2017}. Increases in the concentrations of branch chain amino acids (BCAAs) have been observed in human subjects that progress to type 2 diabetes versus those who do not in the Framingham study \cite{wang2011}. Similarly, 2-Aminoadipic acid has been shown to be associated with development of type 2 diabetes, as well as with insulin resistance and beta cell function \cite{wang2013}. The model posited for which these findings are consistent is that increased concentrations of BCAAs interfere with lipid oxidation in skeletal muscle which may increase insulin resistance \cite{newgard2017}. While much of the metabolomics studies using human cohorts to study metabolic syndromes have focused on biomarkers of disease risk and progression, other studies have focused on the mechanistic roles of specific metabolites. For example, while previous studies had shown the association between  3-carboxy-4-methyl-5-propyl-2-furanpropanoic acid (CMPF) in type 2 and gestational diabetes development, CMPF was shown to directly induce $\beta$ cell dysfunction \cite{prentice2014} and to induce metabolic remodeling \cite{liu2016}.

Coronary heart disease, a consequence of atherosclerosis, is the leading cause of death throughout the world \cite{benjamin2017}. Atherosclerosis is a disease process that is defined by the accumulation of lipids and other elements in the sub-endothelial space of arteries \cite{tabas2015,falk2005,lusis2000}. Multiple different cell types play a role in atherosclerosis including leukocytes, endothelial cells, and smooth muscle cells. Atherosclerosis is also characterized by inflammatory activation in both endothelial and smooth muscle cells. The stages of atherosclerotic plaque development have been described previously \cite{stoll2006}, and include endothelial dysfunction, the accumulation of lipoproteins in the vessel intima, recruitment of leukocytes, and the formation of foam cells. In the later stages of plaque progression, an atherosclerotic plaque evolves as fibrosis and calcification take place. While the progression of atherosclerotic plaques can be a complication on their own as the vessel lumen narrows, plaque rupture or erosion can have a severe consequence--acute coronary events \cite{arbab2012}. Plaque rupture occurs when a defect in the fibrous cap of an atherosclerotic cap exposes the core of the cap to blood, which can precipitate the formation of a thrombus (or blood clot) \cite{bentzon2014}. Plaque erosion, in contrast, entails the erosion of the endothelium and subsequent formation of a thrombus overlaying the plaque. 

A thrombus overlaying a disrupted plaque in a coronary artery that disrupts the flow of blood can lead to acute coronary syndromes such as myocardial infarction \cite{arbab2012}. Myocardial infarction (MI) is defined as myocardial cell death that occurs due to ischemia, or inadequate blood supply \cite{thygesen2012}. MI that follows the rupture or erosion of an atherosclerotic plaque is known as Type 1 MI. In addition to this type of myocardial infarction, MI may occur due to other etiological causes such as coronary artery vasospasm or a fixed supply-demand imbalance that follows the narrowing of an artery due to atherosclerosis \cite{thygesen2012}. Other classifications of MI include MI associated with cardiac death and MI associated with revascularization procedures.

Metabolic processes are central to the etiology of both the constitutive disease processes that define heart disease (e.g. atherosclerosis) and acute disease events such as myocardial infarction. The role of lipid metabolism in atherosclerosis has been documented for over sixty years (see, for example, the seminal article ``Lipid metabolism and atherosclerosis'' by Gould, 1951 \cite{gould1951}). For atherosclerotic plaques to form, excess LDL and triglyceride-rich lipoproteins must be present in circulation \cite{nordestgaard2016}. Levels of both remnant cholesterol and triglyceride-rich lipoproteins are both a function of the rate of flux through metabolic processes including cholesterol synthesis, reverse cholesterol transport, and the utilization of cholesterol for anabolic processes \cite{baynes2014}. A ubiquitous therapeutic intervention that has shown benefits in the treatment of coronary heart disease is statins, a class of drugs whose mechanism of action is to inhibit the enzyme 3-hydroxy-3-methylglutaryl-coenzyme A-reductase, an enzyme involved in the synthesis of cholesterol \cite{chait2016}. As endothelial cell dysfunction is a component of the formation of atherosclerotic lesions, metabolic changes in endothelial cells impact the process of lesion formation \cite{gimbrone2016}. While endothelial cell dysfunction and activation leads to a generally maladaptive inflammatory response that contributes to the development and progression of atherosclerotic lesions, the metabolic state of macrophages is also a factor \cite{bories2017}. In addition to total number of macrophages present, the ratio of pro-inflammatory M1 macrophages versus anti-inflammatory M2 macrophages, which have distinct bioenergetic and metabolic phenotypes, may play a role in the progression of atherosclerosis \cite{bories2017}. 

Myocardial infarction (MI) as an acute disease event, may occur as a function of the disposition or fate of a disrupted atherosclerotic plaque in a coronary artery \cite{arbab2015}. First, the formation and progression of atherosclerotic plaques is required, which is largely a product of disordered lipid metabolism and immunometabolic processes, and second a pathological response to plaque disruption is required. A pathological response to plaque rupture may be conceptualized as an imbalance between pro-thrombotic factors and pro-resolving factors \cite{arbab2015}. Thrombosis is the process of blood coagulation in a localized area. Blood coagulation is a metabolic process that involves multiple serine proteases that circulate as proenzymes. A very brief and general description of the process of thrombosis is as follows \cite{baynes2014,gale2010,hoffman2007}. First, activated platelets adhere to and aggregate at the site of vessel injury and form an initial platelet plug \cite{gale2010}. The ``intrinsic pathway'' is then activated by platelets (the ``extrinsic pathway'' may simultaneously be activated due to tissue damage) which consists of a sequence of reactions that culminates in the final ``common pathway'' \cite{hoffman2007}. The common pathway includes a prothrombin activator complex that forms to activate prothrombin to thrombin, and the activation of fibrinogen to fibrin by thrombin. This final step of the common pathway allows for the formation of a fibrin mesh in which platelets and erythrocytes are also trapped. Fibrinolysis, which is also a metabolic process defined by serine proteases that circulate as proenzymes, works to resolve blood clots formed by cross-linked fibrin \cite{gale2010}. Briefly, this process involves the activation of plasminogen into plasmin by either urokinase-type or tissue-type plasminogen activators. Once plasmin is generated from plasminogen, it can cleave fibrin allowing for the breakdown of a clot. While the presentation of thrombosis and fibrinolysis presented herein are fairly simplified, it can be observed that the formation and resolution of blood clots which could occlude a coronary artery and could lead to myocardial infarction are defined by the dysregulation of metabolic processes. 

\section{From metabolism to metabolomics}
While multiple definitions of metabolomics have been proposed, the fundamental conceptual definition is that metabolomics is the study of small molecules in a biological sample \cite{nicholson2008,johnson2016,newgard2017}. The term metabonomics has previously been used almost synonymously with metabolomics, although the focus of this discipline as posited previously study of the changes in metabolic processes following experimental manipulation \cite{nicholson2008}. Given theses definitions, it can be noted that metabonomics requires metabolomics, that is in order to study changes in metabolic processes that follow an experimental manipulation, one must measure small molecules from biological samples. In other words, ``metabonomics'' represents the goal of many ``metabolomics'' studies. For the purposes of the current work we use the term ``metabolomics'' to refer to both the analysis of small molecules from a sample, as well as the study of changes in metabolic processes that follow an experimental manipulation.

5,498 metabolite-disease associations have been noted in the Human Metabolome Database (HMDB) \cite{wishart2018}.

As an -omics discipline, several aspects of metabolomics are unique. The first such aspect is that the background set of which metabolites could be present in humans is unknown and remains elusive. Specifically 18,557 unique metabolites have been detected and quantified in humans (as recorded in the HMDB), while 82,274 ``expected'' metabolites have been determined \cite{wishart2018}. ``Expected'' metabolites are metabolites that have not been detected in human samples, yet the molecular structure is known and it is hypothesized that due to exposure, these metabolites are expected to be present in humans. This stands in sharp contrast to genomics, in which a reference genome was produced following the completion of the human genome project in 2003 \cite{collins2003,oleary2016}. While functional annotation of the genome remains ongoing \cite{oleary2016}, there is substantially less uncertainty as to what the ``reference'' genome should be than what a ``reference'' metabolome should be. Second, the determination of tissue specificity is substantially less advanced in metabolomics than in transcriptomics and proteomics, as well in evaluating the effect of gene variants on tissue specific gene expression. For example, the Tissue-specific Gene Expression and Regulation (TiGER) database began in 2007 with an expressed sequence tag (EST) approach to cataloging tissue-specific gene expression profiles \cite{liu2008}. The study of tissue-specific gene expression has continued with the utilization of RNA-seq data as has been done with the Genotype-Tissue Expression (GTEx) project \cite{lonsdale2013}. Additional efforts have used genomic sequencing in conjunction with gene expression profiling to evaluate whether expression quantitative trait loci (eQTLs) are active in specific tissues \cite{aguet2017,brown2017}. To generate a map of tissue-specific protein expression Uhlen, et al. \cite{uhlen2015} used both RNA-seq data and antibodies corresponding to 16,975 protein-encoding genes to quantify tissue-specificity, which is included in the Human Protein Atlas \cite{uhlen2010}. In contrast to genomics, transcriptomics, and proteomics, a comprehensive mapping of metabolite tissue-specificity does not yet exist. To summarize the first two nuances we have described, it is often not feasible to have \emph{a priori} knowledge of what metabolites could or should be present in a human sample irrespective of the type of sample (tissue, urine, plasma, serum, cell culture). 

The third aspect that distinguishes mass spectrometry-based metabolomics from other -omics disciplines is that often a substantial proportion of the mass features that are detected in a sample (using un-targeted mass spectrometry approaches) are not able to be identified (often greater than 2/3 of the features) \cite{newgard2017}. At this point it is informative to differentiate the analytical strategies employed in mass spectrometry-based metabolomics studies into three different types: targeted, un-targeted / non-targeted, and stable isotope resolved. Stable isotope resolved metabolomics can be conducted using either a targeted or non-targeted approaches. Targeted metabolomics corresponds to the measurement of metabolites specified in advance, for which an analytical techniques such as selected reaction monitoring (SRM) or multiple reaction monitoring (MRM) assays have been developed \cite{roberts2012,zamboni2015}.  Multiple reaction monitoring makes use of mass spectrometers in which parent ions are first selected, fragmented into smaller ions, with these ions being selected and detected \cite{roberts2012}. The use internal standards allows for such methods to be quantitative or semi-quantitative. Targeted metabolomics experiments require substantial upfront methods development in order to determine the specific transitions for each metabolite as well as retentions times and other experimental parameters prior to analysis of samples. In contrast to this approach, un-targeted or non-targeted metabolomics does not stipulate the prior specification of compounds to be quantified in the analysis \cite{zamboni2015,putri2013,roberts2012}. The principal challenge of un-targeted metabolomics is the identification and resolution of the chemical identity of the thousands of features that are detected from any one biological sample \cite{zamboni2015}. To summarize metabolomics is distinct as an -omics discipline as there is a lack of knowledge about what compounds (of endogenous or exogenous origin) should be found in a human, a lack of knowledge about where such compounds should be localized to, and a relatively low identification rate in determining the chemical identity of ions in mass spectrometry-based metabolomics.

Stable Isotope Resolved Metabolomics, while a subset of metabolomics experiments, represents a distinct approach to studying metabolism using metabolomics. SIRM entails the treatment of a system with nutrients that are enriched with isotopes such as \textsuperscript{13}C or \textsuperscript{15}N  \cite{bruntz2017, newgard2017,higashi2014,fan2012}. A specific amount of time is allowed to elapse (specific to the design of experiments) to allow the system to incorporate the stable isotope labeled intermediates and for them to propagate across metabolic pathways. Targeted or un-targeted quantification of metabolites by either mass spectrometry or nuclear magnetic resonance may then be utilized. By examining the fractional enrichment of isotopologues, inference can be made regarding the flux through specific metabolic pathways \cite{higashi2014,fan2012}. Critically, this approach can be utilized to study metabolic reprogramming, as is associated with the initiation and progression of cancer \cite{bruntz2017}.

[Biomarkers versus systems biology inference] \cite{johnson2016}

[Acute vs stable diseases]

\section{Analytical chemistry}
\end{DoubleSpace*}