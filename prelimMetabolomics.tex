\doublespacing
\section{Metabolism} [Need citations]

Metabolism is broadly defined as the chemical reactions that enable life [cite]. The metabolic reactions that sustain life can be categorized as catabolic reactions, or the reactions that entail breaking down and oxidizing molecules to provide energy or substrates for other reactions, and anabolic reactions in which complex molecules are synthesized. Series of coupled reactions are referred to as pathways. Anabolic pathways are characterized by endergonic processes as the synthesis of macromolecules from smaller molecules requires energy. Conversely catabolic pathways entail the break down of larger molecules into smaller molecules for use either as inputs in anabolic reactions or for the release of free energy via oxidation and reduction reactions. 

\begin{figure}[ht!]
	\resizebox{\textwidth}{!}{\includegraphics*{./Plots/tca.png}}
	\caption[TCA cycle pathway diagram]{ Diagram of the tricarboxylic acid (TCA) cycle in \emph{Homo sapiens} [add citation].\label{fig:tca} }
\end{figure}

An example catabolic process is the tricarboxylic acid (TCA) cycle which is illustrated in Fig.~\ref{fig:tca}. In the TCA cycle an input molecule of Acetyl-CoA (from the catabolism of glucose, fats, or proteins) enters into a sequence of reactions which produces three molecules of NADH, one molecule of FADH\textsubscript{2}, and one molecule of GTP. The molecules of NADH and FADH\textsubscript{2} produced during TCA cycle also serve as substrates for the electron transport chain which generates additional molecules of ATP. 

An example anabolic pathway is the cholesterol biosynthesis pathway shown in Fig.~\ref{fig:tca}. In this pathway, acetyl-CoA molecules as a substrate are chemically transformed through a sequence of reactions with intermediates including HMG-CoA, mevalonate, isopentenyl phosphate, squalene, lanosterol, and finally cholesterol. The cholesterol biosynthesis pathway entails over 20 steps, and requires ATP as an energetic input and the cofactors NADH and NADPH. 

\begin{figure}[ht!]
	\resizebox{\textwidth}{!}{\includegraphics*{./Plots/cholesterol.png}}
	\caption[Cholesterol biosynthesis pathway diagram]{ Diagram of the cholesterol biosynthesis pathway in \emph{Homo sapiens}. [add citation] \label{fig:chol} }
\end{figure}

While many of the biochemical processes of metabolism, including glucose catabolism, glycogen metabolism, gluconeogenesis, lipid metabolism, TCA cycle, electron transport chain, nucleotide metabolism, and protein synthesis  take place in the cell, many of the signaling processes that control metabolism as well as many systems of transport are extracellular \cite{voet2013}. As an example, consider lipid metabolism. While the catabolic process of breaking down fatty acids to yield energetic substrates, $\beta$-oxidation, takes place in the mitochondrial matrix of a cell, many of the processes prior to the point of $\beta$-oxidation are extracellular or take place in distinct cells from the cell in which a specific fatty acid molecule is being broken down. Within adipocytes (specialized cells for the storage of fatty acids), fatty acids are stored as triacylglycerols. When activated by hormone sensitive lipase, triacylglycerols are hydrolyzed yielding free fatty acids which are released into the bloodstream. These fatty acid molecules may be transported in blood in complex with the protein albumin, and may be taken up by other cells (especially by hepatocytes in the liver) for fatty acid oxidation. In addition to the oxidation of fatty acids to yield citrate and acetyl-CoA, cells (especially hepatocytes) also synthesize fatty acids for storage to meet future energetic needs. Newly synthesized fatty acid molecules may be transported in the bloodstream in the form of triacylglycerols within chylomicrons as well as in VLDL. These molecules can be utilized to transport fatty acids for storage in adipocytes. 

LOH control of fatty acid metabolism

\section{From metabolism to metabolomics}
While multiple definitions of metabolomics have been proposed, the fundamental conceptual definition is that metabolomics is the study of small molecules in a biological sample \cite{nicholson2008}. The term metabonomics has previously been used almost synonymously with metabolomics, although the focus of this discipline as posited previously study of the changes in metabolic processes following experimental manipulation. Given theses definitions, it can be noted that metabonomics requires metabolomics, that is in order to study changes in metabolic processes that follow an experimental manipulation, one must measure small molecules from biological samples. In other words, ``metabonomics'' represents the goal of many ``metabolomics'' studies. For the purposes of the current work we use the term ``metabolomics'' to refer to both the analysis of small molecules from a sample, as well as the study of changes in metabolic processes that follow an experimental manipulation.

\section{Analytical chemistry}