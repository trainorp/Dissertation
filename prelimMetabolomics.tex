\section{Metabolism} [Need citations]

Metabolism is broadly defined as the chemical reactions that enable life [cite]. The metabolic reactions that sustain life can be categorized as catabolic reactions, or the reactions that entail breaking down and oxidizing molecules to provide energy or substrates for other reactions, and anabolic reactions in which complex molecules are synthesized. Series of coupled reactions are referred to as pathways. Anabolic pathways are characterized by endergonic processes as the synthesis of macromolecules from smaller molecules requires energy. Conversely catabolic pathways entail the break down of larger molecules into smaller molecules for use either as inputs in anabolic reactions or for the release of free energy via oxidation and reduction reactions. 
\begin{figure}[ht!]
	\resizebox{\textwidth}{!}{\includegraphics*{./Plots/tca.png}}
	\caption[TCA cycle pathway diagram]{ Diagram of the tricarboxylic acid (TCA) cycle in \emph{Homo sapiens} [add citation].\label{fig:tca} }
\end{figure}

\begin{figure}[ht!]
	\resizebox{\textwidth}{!}{\includegraphics*{./Plots/cholesterol.png}}
	\caption[Cholesterol biosynthesis pathway diagram]{ Diagram of the cholesterol biosynthesis pathway in \emph{Homo sapiens}. [add citation] \label{fig:chol} }
\end{figure}

An example catabolic process is the tricarboxylic acid (TCA) cycle which is illustrated in Fig.~\ref{fig:tca}. In the TCA cycle an input molecule of Acetyl-CoA (from the catabolism of glucose, fats, or proteins) enters into a sequence of reactions which produces three molecules of NADH, one molecule of FADH\textsubscript{2}, and one molecule of GTP. The molecules of NADH and FADH\textsubscript{2} produced during TCA cycle also serve as substrates for the electron transport chain which generates additional molecules of ATP. 

An example anabolic pathway is the cholesterol biosynthesis pathway shown in Fig.~\ref{fig:tca}. In this pathway, acetyl-CoA molecules as a substrate are chemically transformed through a sequence of reactions with intermediates including HMG-CoA, mevalonate, isopentenyl phosphate, squalene, lanosterol, and finally cholesterol. The cholesterol biosynthesis pathway entails over 20 steps, and requires ATP as an energetic input and the cofactors NADH and NADPH. 

\section{Complexity of a single metabolic process}
While many of the biochemical processes of metabolism, including glucose catabolism, glycogen metabolism, gluconeogenesis, lipid metabolism, TCA cycle, electron transport chain, nucleotide metabolism, and protein synthesis  take place in the cell, many of the signaling processes that control metabolism as well as many systems of transport are extracellular \cite{voet2013}. As an example, consider lipid metabolism. While the catabolic process of breaking down fatty acids to yield energetic substrates, $\beta$-oxidation, takes place in the mitochondrial matrix of a cell, many of the processes prior to the point of $\beta$-oxidation are extracellular or take place in distinct cells from the cell in which a specific fatty acid molecule is being broken down. Within adipocytes (specialized cells for the storage of fatty acids), fatty acids are stored as triacylglycerols. When activated by hormone sensitive lipase, triacylglycerols are hydrolyzed yielding free fatty acids which are released into the bloodstream. These fatty acid molecules may be transported in blood in complex with the protein albumin, and may be taken up by other cells (especially by hepatocytes in the liver) for fatty acid oxidation. In addition to the oxidation of fatty acids to yield citrate and acetyl-CoA, cells (especially hepatocytes) also synthesize fatty acids for storage to meet future energetic needs. Newly synthesized fatty acid molecules may be transported in the bloodstream in the form of triacylglycerols within chylomicrons as well as in VLDL. These molecules can be utilized to transport fatty acids for storage in adipocytes. 

Fatty acid metabolism is largely controlled by hormonal signaling \cite{voet2013}. The activity of the enzyme hormone sensitive lipase is modulated by phosphorylation and dephosphorylation by the enzyme PKA. The activation of PKA is in turn regulated by cAMP concentrations which are influenced by multiple hormonal signals including epinephrine, norepinephrine, and glucagon. In addition to modulating the rate of triacylglycerol lipase in liver and muscle cells, PKA also is an important regulator of fatty acid synthesis. In this way the control of fatty acid breakdown and synthesis is controlled in a coupled manner (when the rate of one process increases, the rate of the other process decreases).

In summary then, a single metabolic process may involve biochemical reactions that take place in multiple cells (of different types), across multiple tissues, with metabolic control influenced by paracrine signaling, endocrine signaling, cellular metabolite concentrations, and extracellular metabolite concentrations.

\section{Dysregulation of metabolism and disease}
Dysregulation of metabolic processes is a prominent feature of human diseases. As a disease, cancer provides a salient example of the relationship between metabolism and disease. While cancer represents a broad class of diseases characterized by abnormal cell growth affecting multiple tissue and organ types, the reprogramming of cellular metabolism is a common feature of all cancers \cite{pavlova2016}. For cancer cells to continue to grow and divide, new sources of nutrients are required and the more efficient use of nutrients in the tumor environment is required. These requirements lead to metabolic changes associated with cancer that have been described \cite{pavlova2016} as six features: (1) changes in the glucose and amino acid uptake by cancer cells and tumor tissue, (2) ``opportunistic'' manners of nutrient utilization, (3) increased utilization of glycolysis and the TCA cycle intermediates for synthesizing intermediates as well as for increasing the production of cofactors, (4) increased requirement of nitrogen, (5) changes in gene expression that are modulated by metabolite concentrations, and (6) abnormal metabolic interactions between cells and the surrounding environment. 

[More on Cancer and metabolism] 

In addition to cancer, metabolic syndromes are marked by dysregulation of metabolism. Changes in the metabolism of branch-chain amino acids have emerged as hallmarks of metabolic syndromes, the development of type 2 diabetes, and insulin resistance in general \cite{newgard2017}. Increases in the concentrations of branch chain amino acids (BCAAs) have been observed in human subjects that progress to type 2 diabetes versus those who do not in the Framingham study \cite{wang2011}. Similarly, 2-Aminoadipic acid has been shown to be associated with development of type 2 diabetes, as well as with insulin resistance and beta cell function \cite{wang2013}. The model posited for which these findings are consistent is that increased concentrations of BCAAs interfere with lipid oxidation in skeletal muscle which may increase insulin resistance \cite{newgard2017}. While much of the metabolomics studies using human cohorts to study metabolic syndromes have focused on biomarkers of disease risk and progression, other studies have focused on the mechanistic roles of specific metabolites. For example, while previous studies had shown the association between  3-carboxy-4-methyl-5-propyl-2-furanpropanoic acid (CMPF) in type 2 and gestational diabetes development, CMPF was shown to directly induce $\beta$ cell dysfunction \cite{prentice2014} and to induce metabolic remodeling \cite{liu2016}.

\section{From metabolism to metabolomics}
While multiple definitions of metabolomics have been proposed, the fundamental conceptual definition is that metabolomics is the study of small molecules in a biological sample \cite{nicholson2008,johnson2016,newgard2017}. The term metabonomics has previously been used almost synonymously with metabolomics, although the focus of this discipline as posited previously study of the changes in metabolic processes following experimental manipulation \cite{nicholson2008}. Given theses definitions, it can be noted that metabonomics requires metabolomics, that is in order to study changes in metabolic processes that follow an experimental manipulation, one must measure small molecules from biological samples. In other words, ``metabonomics'' represents the goal of many ``metabolomics'' studies. For the purposes of the current work we use the term ``metabolomics'' to refer to both the analysis of small molecules from a sample, as well as the study of changes in metabolic processes that follow an experimental manipulation.

5,498 metabolite-disease associations have been noted in the Human Metabolome Database (HMDB) \cite{wishart2018}.

As an -omics discipline, several aspects of metabolomics are unique. The first such aspect is that the background set of which metabolites could be present in humans is unknown and remains elusive. Specifically 18,557 unique metabolites have been detected and quantified in humans (as recorded in the HMDB), while 82,274 ``expected'' metabolites have been determined \cite{wishart2018}. ``Expected'' metabolites are metabolites that have not been detected in human samples, yet the molecular structure is known and it is hypothesized that due to exposure, these metabolites are expected to be present in humans. This stands in sharp contrast to genomics, in which a reference genome was produced following the completion of the human genome project in 2003 \cite{collins2003,oleary2016}. 

[Distinction between targeted and un-targeted, SIRM] \cite{newgard2017}

[Biomarkers versus systems biology inference] \cite{johnson2016}

[Acute vs stable diseases]

\section{Analytical chemistry}