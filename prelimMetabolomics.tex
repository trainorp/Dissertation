\section{Our Data}

The narrow band, wide field data collected  are inventoried in Table~\ref{tab:datacube}, which
may be used as an example of how to include a table in your dissertation.


\begin{table}
\caption[Data Cube Inventory]{A complete list of all 
  bands included in the data cube, along with identifying index numbers and 
  a summary of physical processes responsible for this type of emission. \label{tab:datacube} }
\begin{center}
\resizebox{\textwidth}{!}{
\begin{tabular}{ l c c c c }
\hline
 Instrument & Band Index & Wavelengths  & Frequencies & Nebula Processes\\ 
  & & (m) & (Hz)&  \\ \hline \hline
Effelsberg  & 1 & 2.127 $\times 10^{-1}$ & 1.410 $\times 10^{9}$ & HI 21 cm line and  \\
100 m& 2 & 1.111 $\times 10^{-1}$ & 2.700 $\times 10^{9}$ &	free-free emission. \\
& 3 & 6.316 $\times 10^{-2}$ & 4.750 $\times 10^{9}$ &	\\ \hline
 WMAP & 4 & 1.364 $\times 10^{-2}$ & 2.280 $\times 10^{10}$ & Free-free, thermal dust, synchrotron,\\
& 6 & 1.000 $\times 10^{-2}$ & 3.300 $\times 10^{10}$ & thermal dust, CO line, \\
& 7 & 7.500 $\times 10^{-3}$ & 4.070 $\times 10^{10}$ & \& anomalous microwave emission\\
& 9 & 5.000 $\times 10^{-3}$ & 6.080 $\times 10^{10}$	& (likely due to spinning dust).\\
& 11 & 3.333 $\times 10^{-3}$ & 9.350 $\times 10^{10}$ &   \\ \hline
Planck & 5 & 1.000 $\times 10^{-2}$ & 3.000 $\times 10^{10}$ & Free-free, thermal dust, synchrotron,\\
& 8 & 6.818 $\times 10^{-3}$ & 4.400 $\times 10^{10}$ & thermal dust, CO line,\\
& 10 & 4.286 $\times 10^{-3}$ & 7.000 $\times 10^{10}$ & \& anomalous microwave emission\\
& 12 & 3.000 $\times 10^{-3}$ & 1.000 $\times 10^{11}$ & (likely due to spinning dust).\\
& 13 & 2.098 $\times 10^{-3}$ & 1.430 $\times 10^{11}$ & \\
& 14 & 1.383 $\times 10^{-3}$ & 2.170 $\times 10^{11}$ & \\
& 15 & 8.499 $\times 10^{-4}$ & 3.530 $\times 10^{11}$ & \\
& 16 & 5.505 $\times 10^{-4}$ & 5.450 $\times 10^{11}$ & \\
& 17 & 3.501 $\times 10^{-4}$ & 8.570 $\times 10^{11}$ & \\ \hline
Akari  & 18 & 1.600 $\times 10^{-4}$ & 1.875 $\times 10^{12}$ & Thermal continuum from dust,\\ 
& 19 & 1.400 $\times 10^{-4}$ & 2.143 $\times 10^{12}$ & line emission from [OI], [OII], \\
& 21 & 9.000 $\times 10^{-5}$ & 3.333 $\times 10^{12}$ & [CII] and others. \\
& 22 & 6.500 $\times 10^{-5}$ & 4.615 $\times 10^{12}$ & \\ \hline
IRAS & 20 & 1.000 $\times 10^{-4}$ &  3.000 $\times 10^{12}$ &  Thermal dust emission, \\
& 23 & 6.000 $\times 10^{-5}$ & 5.000 $\times 10^{12}$ &  PAH band emission. \\ 
& 24 & 2.400 $\times 10^{-5}$ & 1.2500 $\times 10^{13}$ &  \\
& 29 & 1.200 $\times 10^{-5}$ & 2.500 $\times 10^{13}$ &  \\ \hline
MSX & 26 & 2.134 $\times 10^{-5}$ & 1.410 $\times 10^{13}$ & Thermal dust emission,  \\
& 27 & 1.465 $\times 10^{-5}$ & 2.040 $\times 10^{13}$ &  PAH band emission \\
& 28 & 1.213 $\times 10^{-5}$ & 2.479 $\times 10^{13}$ &   \\ 
& 31 & 8.280 $\times 10^{-6}$ & 3.623 $\times 10^{13}$ &   \\ \hline
WISE & 25 & 2.200 $\times 10^{-5}$ & 1.364 $\times 10^{13}$ & Thermal dust emission,      \\ 
& 30 & 1.200 $\times 10^{-5}$ & 2.500 $\times 10^{13}$ & PAH band emission, \\
& 32 & 4.600 $\times 10^{-6}$ & 6.522 $\times 10^{13}$ &  and CO line emission\\
& 33 & 3.400 $\times 10^{-6}$ &  8.824 $\times 10^{13}$ &  \\ \hline
FSQ & 34 & 6.723 $\times 10^{-7}$ & 4.462 $\times 10^{14}$ & hydrogen, oxygen, and sulfur\\
& 35 & 6.565 $\times 10^{-7}$ & 4.570 $\times 10^{14}$ & atomic line emission \\
& 36 & 5.008 $\times 10^{-7}$ & 5.990 $\times 10^{14}$ &  \\
& 37 & 4.863 $\times 10^{-7}$ & 6.169 $\times 10^{14}$ &  \\ \hline
GALEX & 38 & 2.271 $\times 10^{-7}$ & 1.321 $\times 10^{15}$ & helium line emission \\
 & & & & UV dust scatter           \\ 
\hline
\end{tabular}
}
\end{center}
\end{table}

\begin{figure}
\begin{center}
\resizebox{6in}{!}{\includegraphics*{rosette_data_bands.png}}
\caption[Rosette Band Coverage]{A visualization of frequency bandwidth coverage for each Rosette data cube slice. Index values on the vertical axis correspond to those in Table \ref{tab:datacube}. \label{fig:bands}}
\end{center}
\end{figure}


\section{Archival Data}

Add sections to your chapters as needed.  This example also shows how to incorporate references to the literature by use of a bibliography file and a \verb|\cite| instance.

Because the Rosette has such a large angular size on the sky, wide field single dish radio data covering
the entire region were limited. Among the most recent of the available data were 1410 and 4750~MHz
continuum maps by Celnik \cite{celnik1985}, and a 2700~MHz continuum map  
by Graham \cite{graham1982}. Due to their age, these data were presented in the form of printed contour maps. Fortunately, calibration information
is included in tabular form, allowing a flux calibrated map to be recreated.  The original digital data are not available.


