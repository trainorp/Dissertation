In scientific writing the use of acronyms (sometimes senseless letters,
sometimes cute names) is common.  It can be \textit{very}\/ annoying to a reader
who is not a specialist in your discipline.  Be cautious about how you use
acronymns and always define them on first use.  It is helpful to the reader to
keep a list as you write, and to have an appendix such as this one in which you
explain them.  This is a simple listing, but a \verb|description| environment
would be nicer. 

\bigskip

\noindent
AAT - Anglo-Australian Telescope \\
GALEX - Galaxy Evolution Explorer \\
HFI - Planck's High Frequency Instrument \\
HPBW - Half Power Beam Width \\
IRAS - Infrared Astronomical Satellite \\
IRSA - Infrared Science Archive \\
LFI - Planck's Low Frequency Instrument \\
LTE - local thermodynamic equilibrium \\
JAXA - Japanese Aerospace Exploration Agency \\
JPL - Jet Propulsion Laboratory \\
LAMBDA - Legacy Archive for Microwave Background Data Analysis \\
MAST - Mikulski Archive for Space Telescopes \\
MSX - Midcourse Space Experiment \\
NASA - National Aeronautics and Space Administration \\
PAH - polycyclic aromatic hydrocarbons \\
PDR - photodissociation region \\
PSF - point spread function \\
SED - spectral energy distribution \\
SNR - supernova remnant \\
WCS - world coordinate system \\
WISE - Widefield Infrared Survey Explorer \\
WISPI - Widefield Spectral Imager \\
WMAP - Wilkinson Microwave Anisotropy Probe \\
YSO - young stellar object \\
ZAMS - zero age main sequence \\










