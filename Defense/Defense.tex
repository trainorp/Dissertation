\documentclass[xcolor=dvipsnames]{beamer} 
\usetheme{AnnArbor}
\usecolortheme{beaver}

\usepackage{amsmath,graphicx,booktabs,tikz,subfig,color}
\definecolor{mycol}{rgb}{.4,.85,1}
\setbeamercolor{title}{bg=mycol,fg=black} 
\setbeamercolor{palette primary}{use=structure,fg=white,bg=red}
\setbeamercolor{block title}{fg=white,bg=red!50!black}
%\setbeamercolor{block title}{fg=white,bg=blue!75!black}

\DeclareMathOperator{\PP}{P}
\DeclareMathOperator{\EE}{E}
\DeclareMathOperator{\argmin}{argmin}
\DeclareMathOperator{\argmax}{argmax}
\DeclareMathOperator{\ent}{H}
\DeclareMathOperator{\tr}{tr}
\DeclareMathOperator{\sign}{sign}
\DeclareMathOperator{\DE}{DE}
\DeclareMathOperator{\EXP}{EXP}
\DeclareMathOperator{\GA}{GA}
\DeclareMathAlphabet\mathbfcal{OMS}{cmsy}{b}{n}

\begin{document}
	
\title[Bayesian Methods for Metabolomics]{{\bf Bayesian analytical approaches for metabolomics:}\\ {\small A novel method for molecular structure-informed probabilistic metabolite interaction modeling, a novel diagnostic model for differentiating Myocardial Infarction type, and approaches for compound identification}}
\author[P.J. Trainor]{Patrick J. Trainor}
\institute[U of L]
{
	% %\includegraphics[scale=.4]{/home/patrick/gdrive//MetabClass/Old/RL2015/IMClogo}\\
	Interdisciplinary Ph.D. Program in Bioinformatics \& \\
	Division of Cardiovascular Medicine  \\
	University of Louisville %\\[2ex]
}
\date[July 2018]{June 19, 2018}

\begin{frame}
	\titlepage
\end{frame}

\begin{frame}{Outline}
	\vspace{-10.5pt}
	\tableofcontents[hideallsubsections]
\end{frame}

\section{Probabilistic interaction modeling (Interactome)}
\subsection{Motivation: Higher order inference in metabolomics}
\begin{frame}{Outline}
	\vspace{-10.5pt}
	\tableofcontents[currentsection,subsectionstyle=hide]
\end{frame}

\begin{frame}{Outline}
	\vspace{-10.5pt}
	\tableofcontents[currentsection,subsectionstyle=show/shaded/hide]
\end{frame}

\begin{frame}{Motivation}
	\vspace{-15.5pt}
	{\LARGE In someone else's words...}
	\begin{center}
		%\includegraphics[scale=.5]{../Metabolomics2018/JohnsonNatureReview}
	\end{center}
	(2016). \emph{Nature Rev Mol Cell Bio, 17}. doi: 10.1038/nrm.2016.25
\end{frame}

\begin{frame}{Motivation}
	\vspace{-10pt}
		{\LARGE A tale of two aims:}
		
		\begin{enumerate}[{1)}]
			\item ``...Aim 2 determines which TAM metabolic pathway(s) is regulated by c-Maf using systems metabolomics approach. We will first determine TAM metabolic pathways using Stable Isotope Resolved Metabolomics (SIRM) approach...'' [1R01CA213990-01]
			\item[]
			\item ``...The objective of the proposed project is to perform untargeted metabolite profiling to reveal metabolite abnormalities that are common to sALS patients and to determine metabolic pathways whose dysregulation contributes to the disease...'' [5F31NS095698-02]
		\end{enumerate}
\end{frame}

\begin{frame}{Motivation}{Biomarker hypotheses}
	\vspace{-10pt}
	You are interested in whether a therapeutic ``changes'' the abundance of a metabolite, $X$ in plasma (e.g. a specific lipid)
	\begin{itemize}
		\item $H_0$: The abundance of $X$ does not differ between two (or more) phenotypes 
		\item $H_A$: The abundance of $X$ does differ between two (or more) phenotypes 
		\item[]
	\end{itemize} 
	
	Or formulated as an equivalence test:
	\begin{itemize}
		\item $H_0$: The abundance of $X$ is not equivalent between two (or more) phenotypes  
		\item $H_A$: The abundance of $X$ is equivalent between two (or more) phenotypes  
	\end{itemize}
\end{frame}

\begin{frame}
	\begin{center}
			%\includegraphics[scale=.4]{../Plots/cholesterol.png}
			
			Lieberman, M., Ahles, P., et al. \emph{Wikipathways}: WP197
	\end{center}
\end{frame}

\begin{frame}{Motivation}{Hypotheses}
	\vspace{-10pt}
	You are interested in whether a therapeutic ``changes'' cholesterol biosynthesis
	\begin{itemize}
		\item $H_0$: cholesterol biosynthesis does not differ between two (or more) phenotypes 
		\item $H_A$: cholesterol biosynthesis does differ between two (or more) phenotypes
		\item[]
	\end{itemize} 
	
	Or formulated as an equivalence test:
	\begin{itemize}
		\item $H_0$: cholesterol biosynthesis is the same between two (or more) phenotypes 
		\item $H_A$: cholesterol biosynthesis is not the same between two (or more) phenotypes
	\end{itemize}
\end{frame}

\begin{frame}
	\begin{center}
		%\includegraphics[scale=.4]{../Plots/cholesterol.png}
		
		Lieberman, M., Ahles, P., et al. \emph{Wikipathways}: WP197
	\end{center}
\end{frame}

\begin{frame}{Motivation}{Higher order inference}
	\vspace{-10pt}
	{\Large Higher-order inference: the testing of hypotheses that are more complex than single metabolite hypothesis tests} \vspace{7pt}
	
	\begin{itemize}
		\item \textbf{``Data dependent'' / ``Unbiased'' Approaches:} 
		\begin{itemize}
		\item Analyses of correlation networks
		\item Module analyses (WGCNA)
		\item Multivariate statistical approaches (Clustering; Latent component modeling such as PCA, PLS-R, PLS-DA)
		\end{itemize}
		\item[]
		\item \textbf{\emph{A priori approaches}--Pathway (or other ontology) enrichment analyses:} 
		\begin{itemize}
			\item Discrete statistical tests (e.g. hypergeometric)
			\item KS statistic-based set enrichments (e.g. GSEA/MSEA)
		\end{itemize}
	\end{itemize}
	
\end{frame}

\begin{frame}{Pathway enrichment analysis}
	\vspace{-10pt}
{\Large Test for pathway enrichment: Does a pathway have more ``significant'' differences in metabolite abundances than expected?} \vspace{10pt}

	\begin{itemize}
		\item pmf for the hypergeometric distribution: 
		\begin{itemize}
			\item $N=$ total metabolites,
			\item $K=$ number of differentially abundant metabolites, 
			\item $n=$ metabolites in the pathway of interest, 
			\item $k=$ number of differentially abundant metabolites in the pathway of interest 
		\end{itemize}
		\begin{align*}
		\PP(X=k) = \frac{  {{K}\choose{k}} {{N-K}\choose{n-k}}  }{ {{N}\choose{n}} } 
		\end{align*}
	\end{itemize}
\end{frame}

\begin{frame}{Bias in pathway enrichment analysis}{The reference set}
	\vspace{-15.5pt}
	\begin{figure}
		%\includegraphics[scale=.32]{/home/patrick/gdrive/Dissertation/Proposal/vennHell2_1}
	\end{figure}
\end{frame}

\begin{frame}{Bias in pathway enrichment analysis}{The reference set}
	\vspace{-15.5pt}
	\begin{figure}
		%\includegraphics[scale=.32]{/home/patrick/gdrive/Dissertation/Proposal/vennHell2_2}
	\end{figure}
\end{frame}

\begin{frame}{Bias in pathway enrichment analysis}{Metabolic footprint vs. cellular metabolism}	\vspace{-10.5pt}
	\begin{center}
		%\includegraphics[scale=.3]{../Metabolomics2018/512px-Circulatory_System_en}
	\end{center}
\end{frame}

\begin{frame}{Correlation networks}{What do they tell you?}
	{I simulated this...}
	\begin{center}
		%\includegraphics[scale=.5]{../../Aim2/Plots/AR1_Om}
	\end{center}
\end{frame}

\begin{frame}{Correlation networks}{What do they tell you?}
	{And fit a correlation network...}
	\begin{center}
		%\includegraphics[scale=.5]{../../Aim2/Plots/AR1_Cor}
	\end{center}
\end{frame}

\begin{frame}{Partial correlation}{A better approach}
	\vspace{-15.5pt}
	{\Large \textbf{Gaussian Graphical Models (GGM)}: from correlation to partial correlation...}
	
	\begin{center}
		%\includegraphics[scale=.75]{../Metabolomics2018/ggmPrev} \\
		Krumsiek et al. (2011). \emph{BMC Syst Bio, 5}(21) doi: 10.1186/1752-0509-5-21
	\end{center}
\end{frame}

\begin{frame}{Partial correlation}{A better approach}
	\vspace{-15.5pt}
	\begin{center}
		%\includegraphics[scale=.75]{../Metabolomics2018/ggmPrev} \\
		Krumsiek et al. (2011). \emph{BMC Syst Bio, 5}(21) doi: 10.1186/1752-0509-5-21
	\end{center}
	\begin{itemize}
		\item Problem: requires estimating $\frac{p*(p-1)}{2}$ coefficients
		\item[]
		\item Not feasible for most metabolomics studies 
	\end{itemize}
\end{frame}

\begin{frame}{Motivation}
	\vspace{-15.5pt}
	\begin{itemize}
		\item Succinct problem statement: 
		\begin{itemize}
			\item To make higher-order inference a model of how metabolites are related is required
			\item[]
			\item Unbiased approach (using a GGM to estimate partial correlation coefficients) is often unfeasible 
			\item[]
			\item Pathway-based approach may be inappropriate (given the experimental system, sample media, etc.)
		\end{itemize}
		\item[]
		\item A Bayesian call to action!
	\end{itemize}
\end{frame}

\begin{frame}{Bayesian call to action}
	
	{\LARGE Frameworks}
	\begin{itemize}
		\item Today:
		\begin{enumerate}[{1)}]
			\item Collect experimental data
			\item Do frequentist analysis
			\item Use your expert knowledge of metabolism to interpret what the results mean ``I knocked out DHCR7 which explains why there are changes in 7-Dehydrocholesterol and Cholesterol''
		\end{enumerate}
		\item[]
		\item Hopefully the future:
		\begin{enumerate}[{1)}]
			\item Use knowledge of metabolism to formulate informative (or weakly informative) priors
			\item Collect experimental data
			\item Evaluate posterior distributions to make inference
		\end{enumerate}
	\end{itemize}
\end{frame}

\subsection{Bayesian reasoning}
\begin{frame}{Outline}
	\vspace{-10.5pt}
	\tableofcontents[currentsection,subsectionstyle=show/shaded/hide]
\end{frame}

\subsection{The current model}
\begin{frame}{Outline}
	\vspace{-10.5pt}
	\tableofcontents[currentsection,subsectionstyle=show/shaded/hide]
\end{frame}

\begin{frame}{Gaussian Graphical Models (GGMs)}
	\vspace{-5.5pt}
	\begin{itemize}
		\item Markov Random Fields: A graph $G=(V,E)$ in which random variables $X_i\in V$, $i\in \{1,2,...p\}$ are represented by vertices and edges in the edge set $E \subseteq V \times V$ represent probabilistic interactions
		\item[]
		\item GGM: $\textbf{X}\sim \mathcal{N}(\boldsymbol{\mu},\boldsymbol{\Omega}^{-1})$ where $\boldsymbol{\Omega}$ is the concentration matrix (inverse of covariance matrix)
		\item[]
		\item Gaussian conditional distributions and marginal distributions
		\item[]
		\item From a GGM you can determine partial correlation coefficients 
	\end{itemize}
\end{frame}

\begin{frame}{Gaussian Graphical Models (GGMs)}
	\vspace{-15.5pt}
	\begin{itemize}
		\item Likelihood:
		\begin{align*}
		L(\boldsymbol{\Omega}|\textbf{X})&=(2 \pi)^{-np/2}|\boldsymbol{\Omega}|^{n/2} \exp \left(-\frac{1}{2}\sum_{i=1}^{n} (\textbf{x}_i-\boldsymbol{\mu})^T \boldsymbol{\Omega} (\textbf{x}_i-\boldsymbol{\mu})\right) 
		\end{align*}
		\item Proportionally:
		\begin{align*} 
		l(\boldsymbol{\Omega}|\textbf{S})\propto\log (\det \boldsymbol{\Omega})-\tr \left( \textbf{S} \boldsymbol{\Omega} \right)
		\end{align*}
		\item[]
		\item Given the experimental design of many/most metabolomics studies the likelihood would be non-convex
	\end{itemize}
\end{frame}

\begin{frame}{GGM estimation}
	\vspace{-15.5pt}
	\begin{itemize}
		\item $L_1$ regularized likelihood:
		\begin{align*}
		l(\boldsymbol{\Omega})\propto\log (\det \boldsymbol{\Omega})-\tr \left( \textbf{S} \boldsymbol{\Omega} \right)-\lambda ||\boldsymbol{\Omega}||_1
		\end{align*}
		\item Method for optimizing: the Graphical Lasso. Friedman, J., et al. (2007). \emph{Biostatistics, 9}. doi: 10.1093/biostatistics/kxm045
		\item[]
		\item Adaptive graphical Lasso: Zou, H. (2006).  \emph{J Amer Stat Assoc, 101}. doi: 10.1198/016214506000000735 
		\begin{align*}
				l(\boldsymbol{\Omega})\propto \log(\det \boldsymbol{\Omega})-\tr \left(\mathbf{S} \boldsymbol{\Omega}\right) - \lambda \sum_{1\leq i \leq p} \sum_{1 \leq j \leq p} w_{ij} |\omega_{ij}|.
		\end{align*}
		with adaptive weights: $w_{ij}=|\hat{\omega}_{ij}|^\alpha$
	\end{itemize}
\end{frame}

\begin{frame}{Bayesian estimation}
	\vspace{-5.5pt}
\begin{itemize}
	\item A hierarchical Bayesian model for the graphical Lasso (Wang, H. [2012]. \emph{Bayesian Analysis, 7}. doi: 10.1214/12-ba729): 
	
		\begin{align*}
		p(\textbf{x}_i|\boldsymbol{\Omega}) =& \mathcal{N}(\textbf{0},\boldsymbol{\Omega}^{-1}) \quad \text{for} \; i=1,2,\hdots,n\\
		p(\boldsymbol{\Omega}|\lambda) =& \frac{1}{C} \prod_{i<j} \DE(\omega_{ij}|\lambda) \prod_{i=1}^{p} \EXP (\omega_{ii} | \lambda / 2) \cdot 1_{\boldsymbol{\Omega}\in M^+},
		\end{align*}
	\item[]
	\item As a scale mixture of Gaussian distributions:
\end{itemize}
		\begin{align*}
		p(\boldsymbol{\omega}| \boldsymbol{\tau},\lambda)=\frac{1}{C_{\boldsymbol{\tau}}} \prod_{i<j} \left[ \frac{1}{\sqrt{2\pi \tau_{ij}}} \exp \left(- \frac{\omega_{ij}^2}{2\tau_{ij}}\right) \right] 
		\prod_{i=1}^{p}  \left[\frac{\lambda}{2} \exp \left(-\frac{\lambda}{2}\omega_{ii} \right)\right] \cdot 1_{\boldsymbol{\Omega}\in M^+}
		\end{align*}
\end{frame}

\begin{frame}{Bayesian estimation}
	\vspace{-15.5pt}
	Also a hierarchical Bayesian model for the adaptive graphical Lasso (Wang, H. [2012]. \emph{Bayesian Analysis, 7}. doi: 10.1214/12-ba729):
	
	\begin{align*}
	p(\mathbf{x}_i|\boldsymbol{\Omega}) = & \mathcal{N}(\mathbf{0,\boldsymbol{\Omega}}^{-1}) \quad \text{for} \; i=1,2,\hdots,n\\
	p(\boldsymbol{\Omega}|\{\lambda_{ij}\}_{i\leq j}) = & C^{-1} \prod_{i<j} \DE(\omega_{ij}|\lambda_{ij}) \prod_{i=1}^{p} \EXP (\omega_{ii} | \lambda_{ii} / 2) \cdot 1_{\boldsymbol{\Omega}\in M^+}\\
	p(\{\lambda_{ij}\}_{i<j}|\{\lambda_{ii}\}_{i=1}^p) &\propto C_{\{\lambda_{ij}\}_{i\leq j}} \prod_{i<j} \GA(r,s)
	\end{align*}
\end{frame}

\subsection{Using molecular similarity for estimation}
\begin{frame}{Outline}
	\vspace{-10.5pt}
	\tableofcontents[currentsection,subsectionstyle=show/shaded/hide]
\end{frame}
\begin{frame}{An informative Bayeisan estimation}{The current work}
	\vspace{-15.5pt}
	\begin{itemize}
		\item \textbf{Idea}:  incorporate prior knowledge regarding the substructures that are shared between compounds (molecular similarity). That is:
		\begin{align*}
			\lambda_{ij}\sim Gamma(r,s_{ij})
		\end{align*}
		\item[]
		\item Conditional expectation: $\EE(\lambda_{ij}|\boldsymbol{\Omega})=(1+r)/(|\omega_{ij} |+s_{ij})$.
	\end{itemize}
\end{frame}

\begin{frame}{Informative Bayeisan estimation}{The current work}
	\begin{center}
			%\includegraphics[scale=.7]{../../Aim2/Plots/lambdasVsSim}
	\end{center}
\end{frame}

\begin{frame}{Informative Bayeisan estimation}{Block Gibbs Sampler}
	\vspace{-10.5pt}
	\begin{itemize}
		\item Need a way to simulate the posterior distribution 
		\item[]
		\item Gibb's sampling is a MCMC technique for simulating a target distribution if conditional distributions are known
		\item[]
		\item Conditional distribution for concentration matrix columns:
	\end{itemize}
	\begin{align*}
	p(\boldsymbol{\omega}_{12}, \omega_{22}|\boldsymbol{\Omega}_{11},\mathbfcal{T},\textbf{X},\lambda) & \propto \left(\omega_{22}-\boldsymbol{\omega}_{12}^T \boldsymbol{\Omega}_{11}^{-1}\boldsymbol{\omega}_{12} \right)^{n/2} \\ &\exp \left( - \frac{1}{2}\left[ \boldsymbol{\omega}_{12}^T \textbf{D}_{\boldsymbol{\tau}} \boldsymbol{\omega}_{12}+ 2 
	\textbf{s}_{12}^T \boldsymbol{\omega}_{12} + (s_{22}+\lambda)\omega_{22}\right] \right)
	\end{align*}
\end{frame}

\subsection{Computational nuances}
\begin{frame}{Outline}
	\vspace{-10.5pt}
	\tableofcontents[currentsection,subsectionstyle=show/shaded/hide]
\end{frame}
\begin{frame}{Informative Bayeisan estimation}{Block Gibbs Sampler}
	\vspace{-15.5pt}
	\begin{itemize}
		\item Developed an R package for implementing this sampler: \emph{BayesianGLasso}
		\item[]
		\item Sampler is written in C++ utilizing the Armadillo linear algebra library
		\item[]
		\item To interface between R and C++, utilized \emph{Rcpp} and \emph{RcppArmadillo}
	\end{itemize}
\end{frame}

\begin{frame}{Simulated experiment}
	\begin{center}
		%\includegraphics[scale=.33]{../../Aim2/Plots/AR1}
	\end{center}
\end{frame}

\subsection{A Heart Disease interactome}
\begin{frame}{Outline}
	\vspace{-10.5pt}
	\tableofcontents[currentsection,subsectionstyle=show/shaded/hide]
\end{frame}
\begin{frame}{Heart Disease Interactome}
	\vspace{-15.5pt}
	\begin{itemize}
		\item Goal: We sought to construct a stable heart disease plasma metabolite interactome
		\item[]
		\item Long term motivation:  to serve as a reference for analyzing metabolic changes to an acute state (MI)
		\item[]
		\item 47 plasma samples from human subjects with heart disease
		\item[]
		\item 522 compounds identified from and quantified by UPLC-MS/MS and GC-MS
	\end{itemize}
\end{frame}

\begin{frame}{Heart Disease Interactome}
		\vspace{-15.5pt}
	\begin{center}
		%\includegraphics[scale=.35]{../../Aim2/Plots/StructHeatmaps}
		
		Local substructure similarity measure adapted from Mitchell, J. (2014). doi: 10.3389/fgene.2014.00237 (collaboration)
	\end{center}
\end{frame}

\begin{frame}{Heart Disease Interactome}{MCMC sampling}
	\vspace{-10.5pt}
	\begin{center}
		%\includegraphics[scale=.52]{../../Aim2/Plots/MCMC}
	\end{center}
\end{frame}

\begin{frame}{Heart Disease Interactome}
	\vspace{-10.5pt}
	\begin{center}
		%\includegraphics[scale=.52]{../../Aim2/Plots/aiBGL1CorBigGraph}
	\end{center}
\end{frame}

\begin{frame}{Heart Disease Interactome}
	\vspace{-10.5pt}
	\begin{center}
		%\includegraphics[scale=.175]{../../Aim2/Plots/aiBGL1Cor(2)}
	\end{center}
\end{frame}

\begin{frame}{The future}
	\vspace{-15pt}
	\begin{center}
		%\includegraphics[scale=.225]{../Metabolomics2018/FlowChart2}
		
		Trainor, P. (2018), \emph{J Biomed Info, 81}. doi: 10.1016/j.jbi.2018.03.007 
	\end{center}
	\begin{itemize}
		\item Use the stable disease interactome to determine systems-levels changes that occur in the relationships between metabolites in plasma following acute disease events (e.g. heart attacks)
		\item[]
		\item Prior distributions 
	\end{itemize}
\end{frame}

\section{Diagnostic model for Acute MI}
\subsection{Acute Myocardial Infarction}
\begin{frame}{Outline}
	\vspace{-10.5pt}
	\tableofcontents[currentsection,subsectionstyle=hide]
\end{frame}

\begin{frame}{Outline}
	\vspace{-10.5pt}
	\tableofcontents[currentsection,subsectionstyle=show/shaded/hide]
\end{frame}

\subsection{Clinical cohort \& samples}
\begin{frame}{Outline}
	\vspace{-10.5pt}
	\tableofcontents[currentsection,subsectionstyle=show/shaded/hide]
\end{frame}

\subsection{Statistical model for discriminating MI type}
\begin{frame}{Outline}
	\vspace{-10.5pt}
	\tableofcontents[currentsection,subsectionstyle=show/shaded/hide]
\end{frame}

\section{Acknowledgments}
\begin{frame}{Outline}
	\vspace{-10.5pt}
	\tableofcontents[currentsection,subsectionstyle=hide]
\end{frame}

\begin{frame}{Acknowledgments}
	\vspace{-5.5pt}
	\begin{center}
		%\includegraphics[scale=.95]{../../../TempPics/Lab/Edits/DSC_0278-2}
	\end{center}
\end{frame}

\begin{frame}{Acknowledgments}
	\vspace{-5.5pt}
	\begin{center}
		%\includegraphics[scale=.25]{fam}
	\end{center}
\end{frame}

\begin{frame}{Acknowledgments}
	\vspace{-5.5pt}
	\begin{center}
		%\includegraphics[scale=.25]{sam}
	\end{center}
\end{frame}

\begin{frame}{Acknowledgments}
	\vspace{-15.5pt}
	\begin{center}
		%\includegraphics[scale=.5]{../Metabolomics2018/SUSignatureHorizontal-2color}
	\end{center}
\end{frame}

\begin{frame}{Acknowledgments}
	\begin{columns}
		\begin{column}{0.5\textwidth}
			\textbf{Doctoral supervision}:
			\begin{itemize}
				\item Andrew DeFilippis, MD, MSc
				\item Shesh Rai, PhD
				\item[]
			\end{itemize}
			\textbf{Funding support}:
			\begin{itemize}
				\item NIH NIGMS 2P20GM103492-06 (Metabolomic Analysis of Atherothrombosis)
				\item Alpha Phi Foundation Heart-to-Heart Award
				\item NIH 1U24DK097154 (PI: Oliver Fiehn) Subproject Award
				\item[]
			\end{itemize}
		\end{column}
		\vspace{-25.5pt}
		\begin{column}{0.5\textwidth}
			\textbf{Doctoral committee }
			\begin{itemize}
				\item Eric Rouchka, DSc
				\item Juw Won Park, PhD
				\item[]
			\end{itemize}
			\textbf{Collaborators}:
			\begin{itemize}
				\item Joshua Mitchell 
				\item Hunter Moseley, PhD
				\item Samantha Carlisle, MS
				\item[]
			\end{itemize}
			\textbf{Lab}: Atherosclerosis/Atherothrombosis Research Laboratory (Twitter: @AtheroLab)
		\end{column}
	\end{columns}
\end{frame}

\end{document}