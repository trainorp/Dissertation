\documentclass[xcolor=dvipsnames]{beamer} 
\usetheme{AnnArbor}
\usecolortheme{beaver}

\usepackage{amsmath,graphicx,booktabs,tikz,subfig,color}
\definecolor{mycol}{rgb}{.4,.85,1}
\setbeamercolor{title}{bg=mycol,fg=black} 
\setbeamercolor{palette primary}{use=structure,fg=white,bg=red}
\setbeamercolor{block title}{fg=white,bg=red!50!black}
%\setbeamercolor{block title}{fg=white,bg=blue!75!black}

\DeclareMathOperator{\PP}{P}
\DeclareMathOperator{\EE}{E}
\DeclareMathOperator{\argmin}{argmin}
\DeclareMathOperator{\argmax}{argmax}
\DeclareMathOperator{\ent}{H}
\DeclareMathOperator{\tr}{tr}
\DeclareMathOperator{\sign}{sign}
\DeclareMathOperator{\DE}{DE}
\DeclareMathOperator{\EXP}{EXP}
\DeclareMathOperator{\GA}{GA}
\DeclareMathAlphabet\mathbfcal{OMS}{cmsy}{b}{n}

\begin{document}
	
\title[Bioinformatics for MS and Systems Inference]{{\bf Bioinformatics for mass spectrometry-based metabolomics/lipidomics \& a novel paradigm for systems-level inference}}
\author[P.J. Trainor]{Patrick J. Trainor, PhD, MS, MA}
{
	% \includegraphics[scale=.4]{/home/patrick/gdrive//MetabClass/Old/RL2015/IMClogo}\\

}
\date[February 2019]{February 14, 2019}

\begin{frame}
	\titlepage
\end{frame}

\section{Overview}

\begin{frame}{Part \#1}{Bioinformatic processing of metabolomics \& lipidomics data generated using mass spectrometry}
\vspace{-15pt}
\begin{center}
	\includegraphics[scale=.3]{SLU1}
\end{center}
\end{frame}

\begin{frame}{Part \#2}{A novel method for making ``systems-level'' inference from metabolomics \& lipidomics data}
\vspace{-15pt}
\begin{center}
\includegraphics[scale=.64]{path2}
\end{center}
\end{frame}

\begin{frame}{Introduction}{Where do metabolites and lipids belong in the central dogma?}
\vspace{-7 pt}
\begin{center}
		\includegraphics[scale=1]{../../centralDogma}
	\end{center}
\end{frame}

\begin{frame}{Introduction}{Where do metabolites and lipids belong in the central dogma?}
\vspace{-7 pt}
\begin{center}
	\includegraphics{../../centralDogmaB.pdf}
\end{center}
\end{frame}

\begin{frame}
	\vspace{-10.5pt}
	\begin{center}
		\includegraphics[scale=.5]{../../tryptophan}
		
		\text{\tiny{(commons.wikimedia.org/wiki/File:Tryptophan\_metabolism.svg)}}
	\end{center}
\end{frame}

\begin{frame}
\begin{center}
	\includegraphics[scale=.5]{../../metabolomics1}
\end{center}
\end{frame}

\begin{frame}
\begin{center}
	\includegraphics[scale=.5]{../../metabolomics2}
\end{center}
\end{frame}

\begin{frame}
	\begin{center}
		\includegraphics[scale=.3]{../../Eicosanoid_synthesis}
		
		\text{\tiny{(commons.wikimedia.org/wiki/File:Eicosanoid\_synthesis.svg)}}
	\end{center}
\end{frame}

\begin{frame}
\begin{center}
	\includegraphics[scale=.3]{../../Eicosanoid_synthesis2}
\end{center}
\end{frame}

\begin{frame}{Introduction}{Where do metabolites and lipids belong in the central dogma?}
\vspace{-7 pt}
\begin{center}
	\includegraphics{../../centralDogmaB.pdf}
\end{center}
\end{frame}

\section{Bioinformatic processing of LC-MS data}
\begin{frame}{Mass spectrometry is ubiquitous}
	\vspace{-15pt}
\begin{center}
	\includegraphics[scale=.4]{35168-8377116}
\end{center}
\end{frame}

\begin{frame}{Liquid Chromatography - Mass Spectrometry}{Chromatography}
	content...
\end{frame}

\begin{frame}{Liquid Chromatography - Mass Spectrometry}{Ionization (Electrospray Ionization)}
\vspace{-7pt}
\begin{center}
	\includegraphics[scale=.45]{../../ESI}
	
	\tiny{(Cartoon from Agilent Q-ToF MS manual)}
\end{center}
\end{frame}

\begin{frame}{Liquid Chromatography - Mass Spectrometry}{Quadrupole Time-of-Flight Instruments}
\vspace{-7pt}
\begin{center}
	\includegraphics[scale=.6]{../../qtof}
\end{center}
\end{frame}

\begin{frame}{Liquid Chromatography - Mass Spectrometry}{Form of the data}
\vspace{-7pt}
\includegraphics{johanes}

\tiny{Sketch courtesy of Dr. Johannes Rainer (Institute for Biomedicine, Eurac Research, Bolzano, Italy)}
\end{frame}

\begin{frame}{Bioinformatic processing}{Overview of steps}
\vspace{-10pt}
\begin{enumerate}
	\setbeamertemplate{enumerate items}[default]
	\item File conversion
	\item[]
	\item Peak detection
	\item[]
	\item Retention time alignment
	\item[]
	\item Peak grouping
	\item[]
	\item Peak filling
	\item[]
	\item Compound identification
\end{enumerate}
\end{frame}

\begin{frame}{Bioinformatic processing}{1. File conversion}
\vspace{-10pt}
	\begin{enumerate}
 \setbeamertemplate{enumerate items}[default]
		\item File conversion
		\begin{itemize}
			\item Vendor data files (or directories) containing scan-level data are non-standard
			\item Need to convert to mzML or mzXML
			\item msconvert from proteowizard
		\end{itemize}
	\end{enumerate}
\end{frame}

\begin{frame}{Bioinformatic processing}{2. Peak detection}
\vspace{-7pt}
\begin{center}
	\includegraphics[scale=.7]{../../xcmsPlots/TIC}
\end{center}
\end{frame}

\begin{frame}{Bioinformatic processing}{2. Peak detection}
\vspace{-7pt}
\begin{center}
	\includegraphics[scale=.7]{../../xcmsPlots/TICZoom}
\end{center}
\end{frame}

\begin{frame}{Bioinformatic processing}{2. Peak detection}
\vspace{-10pt}
\begin{center}
	\includegraphics[scale=.625]{../../xcmsPlots/Spec440}
\end{center}
\end{frame}

\begin{frame}{Bioinformatic processing}{2. Peak detection}
\vspace{-10pt}
\begin{center}
	\includegraphics[scale=.625]{../../xcmsPlots/Spec440Log}
\end{center}
\end{frame}

\begin{frame}{Bioinformatic processing}{2. Peak detection}
\vspace{-10pt}
\begin{center}
	\includegraphics[scale=.625]{../../xcmsPlots/nicePeak}
\end{center}
\end{frame}

\begin{frame}{Bioinformatic processing}{2. Peak detection}
\vspace{-10pt}
\begin{center}
	\includegraphics[scale=.625]{../../xcmsPlots/notNicePeak}
\end{center}
\end{frame}

\begin{frame}{Bioinformatic processing}{3. Retention time alignment}
\vspace{-10pt}
\begin{center}
	\includegraphics[scale=.7]{../../xcmsPlots/NotAligned}
\end{center}
\end{frame}

\begin{frame}{Bioinformatic processing}{3. Retention time alignment}
\vspace{-10pt}
\begin{center}
\includegraphics[scale=.7]{../../xcmsPlots/rtAdjust}
\end{center}
\end{frame}

\begin{frame}{Bioinformatic processing}{3. Retention time alignment}
\vspace{-10pt}
\begin{center}
	\includegraphics[scale=.7]{../../xcmsPlots/Aligned}
\end{center}
\end{frame}

\begin{frame}{Bioinformatic processing}{4. Peak grouping}
\vspace{-7pt}
\textbf{Peak grouping} can refer to: \pause
		\begin{itemize}
			\item Grouping the same peak across multiple samples \pause
			\item Grouping isotopic peaks 
		\end{itemize}
\end{frame}

\begin{frame}{Bioinformatic processing}{4. Peak grouping}
\vspace{-10pt}
\begin{center}
	\includegraphics[scale=.625]{../../xcmsPlots/nicePeak}
\end{center}
\end{frame}

\begin{frame}{Bioinformatic processing}{4. Peak grouping}
\vspace{-7pt}
\textbf{Peak grouping} can refer to:
\begin{itemize}
	\item Grouping ``split peaks''
	\item Grouping the same peak across multiple samples
	\item Grouping isotopic peaks 
	\item Grouping other molecular features (adducts, in-source fragments) together
\end{itemize}
\end{frame}

\begin{frame}{Bioinformatic processing}{4. Peak grouping}
These should not be grouped!
	\begin{center}
		\includegraphics[scale=.6]{../../xcmsPlots/chromPeakDensity}
	\end{center}
\end{frame}

\begin{frame}{Bioinformatic processing}{6. Compound identification}
\vspace{-10pt}
\begin{columns}
	\column{0.5\textwidth}
	Parameters we use:
		\begin{itemize}
			\item Retention time \pause
			\item m/z from MS1 scans \pause
			\item m/z spectra from MS\textsuperscript{2} / MS\textsuperscript{(n)} scans \pause
			\begin{itemize}
				\item Spectral matching using DBs: Metlin, NIST, HMDB, Lipid Maps
			\end{itemize}
		\end{itemize}
	\column{0.5\textwidth}
	\begin{center}
			\includegraphics[scale=.3]{../../xcmsPlots/SpecEx}
	\end{center}
\end{columns}
\end{frame}

\begin{frame}{Bioinformatic processing}{5 + 6. Peak grouping + Compound identification}
\vspace{-10pt}
\begin{center}
	\includegraphics[scale=.45]{../../xcmsPlots/metAssign}
	\\
	\tiny{Ronan Daly, et al. (2014). MetAssign: probabilistic annotation of metabolites from LC–MS data using a Bayesian clustering approach. \emph{Bioinformatics, 19}.}
\end{center}
\end{frame}

\section{Systems-level inference given metabolomics \& lipidomics data}
\begin{frame}{Part \#2}{A novel method for making ``systems-level'' inference from metabolomics \& lipidomics data}
\vspace{-15pt}
\begin{center}
	\includegraphics[scale=.64]{path2}
\end{center}
\end{frame}

\begin{frame}{Motivation}{Biomarker hypotheses}
	\vspace{-10pt}
	You are interested in whether a therapeutic ``changes'' the abundance of a metabolite, $X$ in plasma (e.g. a specific lipid) \pause
	\begin{itemize}
		\item $H_0$: The abundance of $X$ does not differ between two (or more) phenotypes \pause
		\item $H_A$: The abundance of $X$ does differ between two (or more) phenotypes 
		\item[]
	\end{itemize} \pause
	
	Or formulated as an equivalence test:
	\begin{itemize}
		\item $H_0$: The abundance of $X$ is not equivalent between two (or more) phenotypes  
		\item $H_A$: The abundance of $X$ is equivalent between two (or more) phenotypes  
	\end{itemize}
\end{frame}

\begin{frame}
	\begin{center}
			\includegraphics[scale=.4]{../Plots/cholesterol.png}
			
			Lieberman, M., Ahles, P., et al. \emph{Wikipathways}: WP197
	\end{center}
\end{frame}

\begin{frame}{Motivation}{Hypotheses}
	\vspace{-10pt}
	You are interested in whether a therapeutic ``changes'' cholesterol biosynthesis \pause
	\begin{itemize}
		\item $H_0$: cholesterol biosynthesis does not differ between two (or more) phenotypes 
		\item $H_A$: cholesterol biosynthesis does differ between two (or more) phenotypes \pause
		\item[]
	\end{itemize} 
	
	Or formulated as an equivalence test:
	\begin{itemize}
		\item $H_0$: cholesterol biosynthesis is the same between two (or more) phenotypes 
		\item $H_A$: cholesterol biosynthesis is not the same between two (or more) phenotypes
	\end{itemize}
\end{frame}

\begin{frame}
	\begin{center}
		\includegraphics[scale=.4]{../Plots/cholesterol.png}
		
		Lieberman, M., Ahles, P., et al. \emph{Wikipathways}: WP197
	\end{center}
\end{frame}

\begin{frame}{Motivation}{Higher order inference}
	\vspace{-10pt}
	{\Large Higher-order inference: the testing of hypotheses that are more complex than single metabolite hypothesis tests} \vspace{7pt} \pause
	
	\begin{itemize}
		\item \textbf{``Data dependent'' / ``Unbiased'' Approaches:} 
		\begin{itemize}
		\item Analyses of correlation networks
		\item Module analyses (WGCNA)
		\item Multivariate statistical approaches (Clustering; Latent component modeling such as PCA, PLS-R, PLS-DA) \pause
		\end{itemize}
		\item[]
		\item \textbf{\emph{A priori approaches}--Pathway (or other ontology) enrichment analyses:} 
		\begin{itemize}
			\item Discrete statistical tests (e.g. hypergeometric)
			\item KS statistic-based set enrichments (e.g. GSEA/MSEA)
		\end{itemize}
	\end{itemize}
	
\end{frame}

\begin{frame}{Pathway enrichment analysis}
	\vspace{-10pt}
{\Large Test for pathway enrichment: Does a pathway have more ``significant'' differences in metabolite abundances than expected?} \vspace{10pt} \pause

	\begin{itemize}
		\item pmf for the hypergeometric distribution: 
		\begin{itemize}
			\item $N=$ total metabolites,
			\item $K=$ number of differentially abundant metabolites, 
			\item $n=$ metabolites in the pathway of interest, 
			\item $k=$ number of differentially abundant metabolites in the pathway of interest 
		\end{itemize}
		\begin{align*}
		\PP(X=k) = \frac{  {{K}\choose{k}} {{N-K}\choose{n-k}}  }{ {{N}\choose{n}} } 
		\end{align*}
	\end{itemize}
\end{frame}

\begin{frame}{Bias in pathway enrichment analysis}{The reference set}
	\vspace{-15.5pt}
	\begin{figure}
		\includegraphics[scale=.32]{/home/patrick/gdrive/Dissertation/Proposal/vennHell2_1}
	\end{figure}
\end{frame}

\begin{frame}{Bias in pathway enrichment analysis}{The reference set}
	\vspace{-15.5pt}
	\begin{figure}
		\includegraphics[scale=.32]{/home/patrick/gdrive/Dissertation/Proposal/vennHell2_2}
	\end{figure}
\end{frame}

\begin{frame}{Bias in pathway enrichment analysis}{Metabolic footprint vs. cellular metabolism}	\vspace{-10.5pt}
	\begin{center}
		\includegraphics[scale=.3]{../Metabolomics2018/512px-Circulatory_System_en}
	\end{center}
\end{frame}

\begin{frame}{Correlation networks}{What do they tell you?}
	{I simulated this...}
	\begin{center}
		\includegraphics[scale=.5]{../../Aim2/Plots/AR1_Om}
	\end{center}
\end{frame}

\begin{frame}{Correlation networks}{What do they tell you?}
	{And fit a correlation network...}
	\begin{center}
		\includegraphics[scale=.5]{../../Aim2/Plots/AR1_Cor}
	\end{center}
\end{frame}

\begin{frame}{Partial correlation}{A better approach}
	\vspace{-15.5pt}
	{\Large \textbf{Gaussian Graphical Models (GGM)}: from correlation to partial correlation...}
	
	\begin{center}
		\includegraphics[scale=.75]{../Metabolomics2018/ggmPrev} \\
		Krumsiek et al. (2011). \emph{BMC Syst Bio, 5}(21) doi: 10.1186/1752-0509-5-21
	\end{center}
\end{frame}

\begin{frame}{Partial correlation}{A better approach}
	\vspace{-15.5pt}
	\begin{center}
		\includegraphics[scale=.75]{../Metabolomics2018/ggmPrev} \\
		Krumsiek et al. (2011). \emph{BMC Syst Bio, 5}(21) doi: 10.1186/1752-0509-5-21
	\end{center} \pause
	\begin{itemize}
		\item Problem: requires estimating $\frac{p*(p-1)}{2}$ coefficients
		\item[]
		\item Not feasible for most metabolomics studies 
	\end{itemize}
\end{frame}

\begin{frame}{Motivation}
	\vspace{-15.5pt}
	\begin{itemize}
		\item Problem statement: \pause
		\begin{itemize}
			\item To make higher-order inference a model of how metabolites are related is required \pause
			\item[]
			\item Unbiased approach (using a GGM to estimate partial correlation coefficients) is often unfeasible \pause
			\item[]
			\item Pathway-based approach may be inappropriate (given the experimental system, sample media, etc.) \pause
		\end{itemize}
		\item[]
		\item Solution: Utilize prior scientific knowledge (biochemistry) to inform the search for graphical models that can then be used to estimate partial correlation coefficients
	\end{itemize}
\end{frame}

\begin{frame}{Gaussian Graphical Models (GGMs)}
	\vspace{-5.5pt}
	\begin{itemize}
		\item Markov Random Fields: A graph $G=(V,E)$ in which random variables $X_i\in V$, $i\in \{1,2,...p\}$ are represented by vertices and edges in the edge set $E \subseteq V \times V$ represent probabilistic interactions \pause
		\item[]
		\item GGM: $\textbf{X}\sim \mathcal{N}(\boldsymbol{\mu},\boldsymbol{\Omega}^{-1})$ where $\boldsymbol{\Omega}$ is the concentration matrix (inverse of covariance matrix) \pause
		\item[]
		\item Gaussian conditional distributions and marginal distributions \pause
		\item[]
		\item From a GGM you can determine partial correlation coefficients 
	\end{itemize}
\end{frame}

\begin{frame}{Gaussian Graphical Models (GGMs)}
	\vspace{-15.5pt}
	\begin{itemize}
		\item Likelihood:
		\begin{align*}
		L(\boldsymbol{\Omega}|\textbf{X})&=(2 \pi)^{-np/2}|\boldsymbol{\Omega}|^{n/2} \exp \left(-\frac{1}{2}\sum_{i=1}^{n} (\textbf{x}_i-\boldsymbol{\mu})^T \boldsymbol{\Omega} (\textbf{x}_i-\boldsymbol{\mu})\right) 
		\end{align*}\pause
		\item Proportionally:
		\begin{align*} 
		l(\boldsymbol{\Omega}|\textbf{S})\propto\log (\det \boldsymbol{\Omega})-\tr \left( \textbf{S} \boldsymbol{\Omega} \right)
		\end{align*}\pause
		\item[]
		\item Given the experimental design of many/most metabolomics studies the likelihood would be non-convex
	\end{itemize}
\end{frame}

\begin{frame}{GGM estimation}
	\vspace{-15.5pt}
	\begin{itemize}
		\item $L_1$ regularized likelihood:
		\begin{align*}
		l(\boldsymbol{\Omega})\propto\log (\det \boldsymbol{\Omega})-\tr \left( \textbf{S} \boldsymbol{\Omega} \right)-\lambda ||\boldsymbol{\Omega}||_1
		\end{align*}\pause
		\item Method for optimizing: the Graphical Lasso. Friedman, J., et al. (2007). \emph{Biostatistics, 9}. doi: 10.1093/biostatistics/kxm045 \pause
		\item[]
		\item Adaptive graphical Lasso: Zou, H. (2006).  \emph{J Amer Stat Assoc, 101}. doi: 10.1198/016214506000000735 
		\begin{align*}
				l(\boldsymbol{\Omega})\propto \log(\det \boldsymbol{\Omega})-\tr \left(\mathbf{S} \boldsymbol{\Omega}\right) - \lambda \sum_{1\leq i \leq p} \sum_{1 \leq j \leq p} w_{ij} |\omega_{ij}|.
		\end{align*}
		with adaptive weights: $w_{ij}=|\hat{\omega}_{ij}|^\alpha$
	\end{itemize}
\end{frame}

\begin{frame}{Bayesian estimation}
	\vspace{-5.5pt}
\begin{itemize}
	\item A hierarchical Bayesian model for the graphical Lasso (Wang, H. [2012]. \emph{Bayesian Analysis, 7}. doi: 10.1214/12-ba729): 
	
		\begin{align*}
		p(\textbf{x}_i|\boldsymbol{\Omega}) =& \mathcal{N}(\textbf{0},\boldsymbol{\Omega}^{-1}) \quad \text{for} \; i=1,2,\hdots,n\\
		p(\boldsymbol{\Omega}|\lambda) =& \frac{1}{C} \prod_{i<j} \DE(\omega_{ij}|\lambda) \prod_{i=1}^{p} \EXP (\omega_{ii} | \lambda / 2) \cdot 1_{\boldsymbol{\Omega}\in M^+},
		\end{align*} \pause
	\item[]
	\item As a scale mixture of Gaussian distributions:
\end{itemize}
		\begin{align*}
		p(\boldsymbol{\omega}| \boldsymbol{\tau},\lambda)=\frac{1}{C_{\boldsymbol{\tau}}} \prod_{i<j} \left[ \frac{1}{\sqrt{2\pi \tau_{ij}}} \exp \left(- \frac{\omega_{ij}^2}{2\tau_{ij}}\right) \right] 
		\prod_{i=1}^{p}  \left[\frac{\lambda}{2} \exp \left(-\frac{\lambda}{2}\omega_{ii} \right)\right] \cdot 1_{\boldsymbol{\Omega}\in M^+}
		\end{align*}
\end{frame}

\begin{frame}{Bayesian estimation}
	\vspace{-15.5pt}
	Also a hierarchical Bayesian model for the adaptive graphical Lasso (Wang, H. [2012]. \emph{Bayesian Analysis, 7}. doi: 10.1214/12-ba729):
	
	\begin{align*}
	p(\mathbf{x}_i|\boldsymbol{\Omega}) = & \mathcal{N}(\mathbf{0,\boldsymbol{\Omega}}^{-1}) \quad \text{for} \; i=1,2,\hdots,n\\
	p(\boldsymbol{\Omega}|\{\lambda_{ij}\}_{i\leq j}) = & C^{-1} \prod_{i<j} \DE(\omega_{ij}|\lambda_{ij}) \prod_{i=1}^{p} \EXP (\omega_{ii} | \lambda_{ii} / 2) \cdot 1_{\boldsymbol{\Omega}\in M^+}\\
	p(\{\lambda_{ij}\}_{i<j}|\{\lambda_{ii}\}_{i=1}^p) &\propto C_{\{\lambda_{ij}\}_{i\leq j}} \prod_{i<j} \GA(r,s)
	\end{align*}
\end{frame}

\subsection{Using molecular similarity for estimation}

\begin{frame}{An informative Bayesian estimation}{The current work}
	\vspace{-15.5pt}
	\begin{itemize}
		\item \textbf{Idea}:  incorporate prior knowledge regarding the substructures that are shared between compounds (molecular similarity). That is:
		\begin{align*}
			\lambda_{ij}\sim Gamma(r,s_{ij})
		\end{align*}\pause
		\item[]
		\item[]
		\item Conditional expectation: $\EE(\lambda_{ij}|\boldsymbol{\Omega})=(1+r)/(|\omega_{ij} |+s_{ij})$.
	\end{itemize}
\end{frame}

\begin{frame}{Informative Bayesian estimation}{The current work}
	\begin{center}
			\includegraphics[scale=.7]{../../Aim2/Plots/lambdasVsSim}
	\end{center}
\end{frame}

\begin{frame}{Informative Bayesian estimation}{Block Gibbs Sampler}
	\vspace{-10.5pt}
	\begin{itemize}
		\item Need a way to simulate the posterior distribution \pause
		\item[]
		\item Gibb's sampling is a MCMC technique for simulating a target distribution if conditional distributions are known \pause
		\item[]
		\item Conditional distribution for concentration matrix columns:
	\end{itemize}
	\begin{align*}
	p(\boldsymbol{\omega}_{12}, \omega_{22}|\boldsymbol{\Omega}_{11},\mathbfcal{T},\textbf{X},\lambda) & \propto \left(\omega_{22}-\boldsymbol{\omega}_{12}^T \boldsymbol{\Omega}_{11}^{-1}\boldsymbol{\omega}_{12} \right)^{n/2} \\ &\exp \left( - \frac{1}{2}\left[ \boldsymbol{\omega}_{12}^T \textbf{D}_{\boldsymbol{\tau}} \boldsymbol{\omega}_{12}+ 2 
	\textbf{s}_{12}^T \boldsymbol{\omega}_{12} + (s_{22}+\lambda)\omega_{22}\right] \right)
	\end{align*}
\end{frame}

\subsection{Computational nuances}

\begin{frame}{Informative Bayesian estimation}{Block Gibbs Sampler}
	\vspace{-15.5pt}
	\begin{itemize}
		\item Developed an R package for implementing this sampler: \emph{BayesianGLasso} \pause
		\item[]
		\item Sampler is written in C++ utilizing the Armadillo linear algebra library \pause
		\item[]
		\item To interface between R and C++, utilized \emph{Rcpp} and \emph{RcppArmadillo} \pause
		\item[]
		\item To utilize with $\approx 500$ metabolites significant optimization was required (openBLAS \& LAPACK routines)
	\end{itemize}
\end{frame}

\begin{frame}{Simulated experiment}
	\begin{center}
		\includegraphics[scale=.33]{../../Aim2/Plots/AR1}
	\end{center}
\end{frame}

\subsection{A Heart Disease interactome}

\begin{frame}{Heart Disease Interactome}
	\vspace{-15.5pt}
	\begin{itemize}
		\item Goal: We sought to construct a stable heart disease plasma metabolite interactome \pause
		\item[]
		\item Long term motivation:  to serve as a reference for analyzing metabolic changes to an acute state (MI) \pause
		\item[]
		\item 47 plasma samples from human subjects with heart disease \pause
		\item[]
		\item 522 compounds identified from and quantified by UPLC-MS/MS and GC-MS
	\end{itemize}
\end{frame}

\begin{frame}{Heart Disease Interactome}
		\vspace{-15.5pt}
	\begin{center}
		\includegraphics[scale=.35]{../../Aim2/Plots/StructHeatmaps}
		
		Local substructure similarity measure adapted from Mitchell, J. (2014). doi: 10.3389/fgene.2014.00237 (collaboration)
	\end{center}
\end{frame}

\begin{frame}{Heart Disease Interactome}{MCMC sampling}
	\vspace{-10.5pt}
	\begin{center}
		\includegraphics[scale=.52]{../../Aim2/Plots/MCMC}
	\end{center}
\end{frame}

\begin{frame}{Heart Disease Interactome}
	\vspace{-10.5pt}
	\begin{center}
		\includegraphics[scale=.52]{../../Aim2/Plots/aiBGL1CorBigGraph}
	\end{center}
\end{frame}

\begin{frame}{Heart Disease Interactome}
	\vspace{-10.5pt}
	\begin{center}
		\includegraphics[scale=.175]{../../Aim2/Plots/aiBGL1Cor(2)}
	\end{center}
\end{frame}

\begin{frame}{The future}
	\vspace{-15pt}
	\begin{center}
		\includegraphics[scale=.225]{../Metabolomics2018/FlowChart2}
		
		Trainor, P. (2018), \emph{J Biomed Info, 81}. doi: 10.1016/j.jbi.2018.03.007 \pause
	\end{center}
	\begin{itemize}
		\item Use the stable disease interactome to determine systems-levels changes that occur in the relationships between metabolites in plasma following acute disease events (e.g. heart attacks) \pause
		\item[]
		\item Prior distributions 
	\end{itemize}
\end{frame}

\section{Acknowledgments}

\begin{frame}{Acknowledgments}
	\vspace{-15.5pt}
	\begin{center}
		\includegraphics[scale=.5]{../Metabolomics2018/SUSignatureHorizontal-2color}
	\end{center}
\end{frame}

\begin{frame}{Acknowledgments}
	\begin{columns}
		\begin{column}{0.5\textwidth}
			\textbf{Doctoral supervision}:
			\begin{itemize}
				\item Andrew DeFilippis, MD, MSc
				\item Shesh Rai, PhD
				\item[]
			\end{itemize}
			\textbf{Funding support}:
			\begin{itemize}
				\item NIH NIGMS 2P20GM103492-06 (Metabolomic Analysis of Atherothrombosis)
				\item Alpha Phi Foundation Heart-to-Heart Award
				\item NIH 1U24DK097154 (PI: Oliver Fiehn) Subproject Award
				\item[]
			\end{itemize}
		\end{column}
		\vspace{-25.5pt}
		\begin{column}{0.5\textwidth}
			\textbf{Doctoral committee }
			\begin{itemize}
				\item Eric Rouchka, DSc
				\item Juw Won Park, PhD
				\item[]
			\end{itemize}
			\textbf{Collaborators}:
			\begin{itemize}
				\item Joshua Mitchell 
				\item Hunter Moseley, PhD
				\item Samantha Carlisle, MS
				\item[]
			\end{itemize}
			\textbf{Lab}: Atherosclerosis/Atherothrombosis Research Laboratory (Twitter: @AtheroLab)
		\end{column}
	\end{columns}
\end{frame}

\end{document}