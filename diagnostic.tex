\section{Acute Myocardial Infarction}

Acute Myocardial Infarction (AMI), which is an acute manifestation of coronary heart disease, is defined by
myocardial ischemia (exposure of cardiac myocytes to oxygen deprivation) and necrosis (a form of cell death) \cite{trainor2018}. AMI may occur following atherosclerotic plaque disruption or other conditions which cause demand ischemia \cite{thygesen2012,arbab2012}. Irrespective of underlying cause, ischemia and necrosis are the common pathological characteristics of all AMI. Thrombotic MI (MI results from spontaneous atherosclerotic plaque disruption with the formation of an occluding coronary thrombus) versus non-thrombotic MI represents an important etiological distinction \cite{defilippis2017}, as both types necessitate different treatment approaches \cite{thygesen2012}. A diagnosis of AMI can be substantiated by blood-based diagnostic tests that measure isoforms of the protein troponin which is released into the circulation following necrosis \cite{newby2012}. To date a blood-based diagnostic test capable of discriminating between thrombotic and non-thrombotic MI has not been developed, although it has been previously shown that a metabolic signature may differentiate between the types \cite{defilippis2017,trainor2018}. In the present work, we set out to develop a Bayesian model for differentiating thrombotic MI from both non-thrombotic MI and stable coronary artery disease (CAD) using metabolites detected in blood plasma. We regard plasma as a promising media for developing a non-invasive test as plasma contains hormones, enzymes, lipoproteins, and other metabolic intermediates found in circulation. As metabolite concentrations are a product of genetic factors, environmental exposures, and the interaction between the two, sampling metabolites may provide a more robust characterization of the state of an organism than other approaches such as genomics. 

\section{Clinical cohort and samples}
Towards the effort of developing a Bayesian diagnostic model, we utilized previously collected samples from a patient cohort that was recruited specifically for contrasting thrombotic MI from multiple control phenotypes \cite{defilippis2015,defilippis2017}. This cohort was comprised of three phenotypic groups of human subjects: thrombotic MI, non-thrombotic MI, and stable CAD. In reference to the  thrombotic MI group, both control groups (non-thrombotic MI and stable CAD) served as procedural controls as all groups underwent a cardiac catheterization procedure. The stable CAD group provides a stable disease control as both thrombotic MI subjects and stable CAD subjects have underlying coronary artery disease. Non-thrombotic MI subjects presented with myocardial necrosis and thus the non-thrombotic MI group serves as an acute disease event control. 

Whole blood was collected shortly before cardiac catheterization from all study subjects. Details of the sample handling, sample processing, separation by liquid or gas chromatography, and mass spectrometry analysis for quantifying metabolites is provided in Chapter. 