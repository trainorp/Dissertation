\begin{DoubleSpace*}
\section{Acute Myocardial Infarction}
Acute Myocardial Infarction (AMI), which is an acute manifestation of coronary heart disease, is defined by
myocardial ischemia (exposure of cardiac myocytes to oxygen deprivation) and necrosis (a form of cell death) \cite{trainor2018}. AMI may occur following atherosclerotic plaque disruption or other conditions which cause demand ischemia \cite{thygesen2012,arbab2012}. Irrespective of underlying cause, ischemia and necrosis are the common pathological characteristics of all AMI. Thrombotic MI (MI results from spontaneous atherosclerotic plaque disruption with the formation of an occluding coronary thrombus) versus non-thrombotic MI represents an important etiological distinction \cite{defilippis2017}, as both types necessitate different treatment approaches \cite{thygesen2012}. A diagnosis of AMI can be substantiated by blood-based diagnostic tests that measure isoforms of the protein troponin which is released into the circulation following necrosis \cite{newby2012}. To date a blood-based diagnostic test capable of discriminating between thrombotic and non-thrombotic MI has not been developed, although it has been previously shown that a metabolic signature may differentiate between the types \cite{defilippis2017,trainor2018}. In the present work, we set out to develop a Bayesian model for differentiating thrombotic MI from both non-thrombotic MI and stable coronary artery disease (CAD) using metabolites detected in blood plasma. We regard plasma as a promising media for developing a non-invasive test as plasma contains hormones, enzymes, lipoproteins, and other metabolic intermediates found in circulation. As metabolite concentrations are a product of genetic factors, environmental exposures, and the interaction between the two, sampling metabolites may provide a more robust characterization of the state of an organism than other approaches such as genomics. 

\section{Clinical cohort and samples}
Towards the effort of developing a Bayesian diagnostic model, we utilized previously collected samples from a patient cohort that was recruited specifically for contrasting thrombotic MI from multiple control phenotypes \cite{defilippis2015,defilippis2017}. This cohort was comprised of three phenotypic groups of human subjects: thrombotic MI, non-thrombotic MI, and stable CAD. In reference to the  thrombotic MI group, both control groups (non-thrombotic MI and stable CAD) served as procedural controls as all groups underwent a cardiac catheterization procedure. The stable CAD group provides a stable disease control as both thrombotic MI subjects and stable CAD subjects have underlying coronary artery disease. Non-thrombotic MI subjects presented with myocardial necrosis and thus the non-thrombotic MI group serves as an acute disease event control. 

Whole blood was collected shortly before cardiac catheterization from all study subjects. Details of the sample handling, sample processing, separation by liquid or gas chromatography, and mass spectrometry analysis for quantifying metabolites are provided in Chapter~\ref{hdInteractome} Section~\ref{plasma}. 

\section{Feature selection}
A significant analytical challenge in developing a blood-based diagnostic test for differentiating MI types is to determine a small set of metabolites that should be included in the statistical model from a limited number of training samples. Specifically, 1,032 chemical features (identified compounds or unknown compounds) were detected from 11 thrombotic MI, 12 non-thrombotic MI, and 15 stable CAD subjects. To ensure that the statistical model is estimable from a small sample size, and as building a targeted MRM assay or multiplexed ELISA assay can only accommodate a small number of compounds, feature selection is a critical task. While other dimension reduction techniques such as latent variable approaches (e.g. Partial Least Squares models) that create new variables which are linear combinations of metabolites would be amenable for reducing the number of coefficients to be estimated in a classification model, these approaches would not reduce the number of metabolites needing to be quantified by future targeted assays. Consequently, we prioritized reducing the number of metabolites considered. To determine a statistical classifier with five metabolites, $9.7 \times 10^{12}$ combinations are possible. In order to search the space of possible models we employed a feature selection technique that utilizes an evolutionary algorithm and seeks a consensus solution over bootstrapped datasets as described in Trainor et al. \cite{trainor2018}. In this work, small sets of metabolites were included in a multinomial logit classifier. Each model represented an individual in a population. Genetic fitness was determined as the likelihood of each individual model. These populations of models were allowed to evolve given evolutionarily inspired processes such as birth, recombination, and death. Populations were grown over bootstrapped datasets to increase diversity and reduce the correlation between models in the populations. Finally, the frequency that an individual metabolite was included in models within the final population after epochs of evolution was determined yielding a variable importance score for each metabolite. A correlation plot illustrating the metabolites with greatest variable importance score using the described technique is shown in Figure~\ref{fig:include}.

\end{DoubleSpace*}

\begin{figure}[H]
	\resizebox{1.1\textwidth}{!}{\includegraphics*{../../AthroMetab/WoAC/corrplot333}}
	\caption[Feature selection]{Metabolites with the highest variable importance score given the feature selection method we employed in a previous work \cite{trainor2018}. The Pearson correlation coefficients between metabolite transformed and scaled abundances are shown. \label{fig:include}}
\end{figure}

\begin{DoubleSpace*}
\section{Model}
A multinomial logit model was assumed for determining the  probability a sample from a clinical subject was member of the thrombotic MI, non-thrombotic MI, or a stable CAD study group. The multinomial logit model is a generalized linear model and has the following link function and model form: 
\begin{align}
\eta_{ij} = \log \frac{\pi_{ij}}{\pi_{iJ}} = \alpha_j + \textbf{x}_i^T \boldsymbol{\beta}_j,
\end{align} 
where $i$ indexes individual samples, $j$ is the index of study groups, $\boldsymbol{\beta}_j$ is a vector of regression coefficients (with separate coefficients for each group), $\alpha_j$ is a group specific intercept term, and $J$ represents the reference group. Given a specific value for the link function, the probability a sample belongs to specific study group can be computed as:
\begin{align}
\hat{\pi}_{ij} = \frac{\exp \hat{\eta}_{ij}}{\sum_{k=1}^{J}\exp \hat{\eta}_{ik}}.
\end{align}

A MCMC sampler known as the ``No-U-Turn'' sampler was utilized to simulate the posterior distribution of model parameters \cite{hoffman2014}. This algorithm is regarded as an extension to the Hamiltonian Monte Carlo algorithm. The Hamiltonian Monte Carlo algorithm is designed to mimic a Hamiltonian system in which a parameter $\theta$ is viewed as a particle in a $D$-dimensional space \cite{neal2011}. The logarithm of the joint density of parameters, $\mathcal{L}$ is then conceptualized as the negative potential energy function. Finally $r_d$ is the momentum of the particle, with $d$ representing the specific dimension. 

\section{Results}
Blah blah blah MCMC.

 \begin{figure}[H]
 	\resizebox{1.15\textwidth}{!}{\includegraphics*{../Aim3/ptid2010MCMC}}
 	\caption[Add caption]{Add caption \label{fig:ptid2010MCMC} }
 \end{figure}
 
  \begin{figure}[H]
  	\resizebox{1.15\textwidth}{!}{\includegraphics*{../Aim3/ptid2010Hist}}
  	\caption[Add caption]{Add caption \label{fig:ptid2010Hist} }
  \end{figure}
 
  Leave-one-out cross validation estimates of study group membership for each clinical sample. Group membership was determined by maximum a posteriori estimates, and the class with maximum estimated probability of membership is reported.
 
\begin{table}[H]
	\caption{text}
	\label{tab:modelRes}
	\centering
	\begin{tabular}{l|ccc}
		           & Predicted & & \\
		           group  &     sCAD & Thrombotic MI & Non-Thrombotic MI \\
		           \hline
		           sCAD   &     13    &     0    &     2\\
		           Thrombotic MI  &  0    &    10    &     1\\
		           Non-Thrombotic MI &   1    &     1   &     10
	\end{tabular}
\end{table}

Stan \cite{carpenter2017}
\end{DoubleSpace*}

\newpage
\KOMAoptions{paper=landscape}
\recalctypearea
\begin{figure}[H]
	\includegraphics[scale=1.15]{../Aim3/brm1Coef}
	\caption[Add caption]{Add caption \label{fig:brm1Coef} }
\end{figure}
\newpage
\begin{figure}[H]
	\includegraphics[scale=1.1]{../Aim3/MCMCEx.png}
	\caption[Add caption]{Add caption \label{fig:brm1Coef} }
\end{figure}
\newpage
\KOMAoptions{paper=portrait,pagesize}
\recalctypearea

Blah blah blah blah blah 