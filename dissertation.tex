\RequirePackage{snapshot}
\documentclass[final]{ulthesis}

\def \apjl{Astrophysical Journal Letters}
\def \apjs{Astrophysical Journal Supplement Series}
\def \apj{Astrophysical Journal}
\def \aaps{Astronomy and Astrophysics Supplement}
\def \apss{Astrophysics and Space Science}
\def \mnras{Monthly Notices of the Royal Astronomical Society}
\def \aap{Astronomy and Astrophysics}
\def \aj{Astronomical Journal}
\def \pasp{Publications of the Astronomical Society of the Pacific}
\def \pasj{Publications of the Astronomical Society of Japan}
\def \ao{Applied Optics}


\usepackage[pdfauthor={Patrick Trainor},
            pdftitle={Bayesian analytical approaches for metabolomics},
            pdfsubject={Bayesian Metabolomics},
            pdfkeywords={Bayesian Statistics, Metabolomics, Mass Spectrometry, Interactome},
            pdfproducer={Latex with hyperref},
            pdfcreator={latex->dvips->ps2pdf},
            pdfpagemode=UseOutlines,
            bookmarksopen=true,
            bookmarksnumbered=true]{hyperref}
\usepackage{memhfixc}

%Additional packages that are useful
\usepackage[pdftex]{graphicx}
\usepackage{verbatim,color,amsmath,amssymb,amsthm,amsfonts,listings,subfig} 
\usepackage{hyperref}  % use without arguments for no metatdata
\usepackage[labelsep=period, labelfont=bf]{caption} % boldface with period
\usepackage{chngcntr}   % use to serially number figures and tables 
\usepackage{algorithm,algorithmic}
\usepackage[paper=portrait,pagesize]{typearea}
\usepackage{setspace}
\usepackage{breakcites}

\DeclareMathOperator{\PP}{P}
\DeclareMathOperator{\EE}{E}
\DeclareMathOperator{\argmin}{argmin}
\DeclareMathOperator{\argmax}{argmax}
\DeclareMathOperator{\ent}{H}
\DeclareMathOperator{\tr}{tr}
\DeclareMathOperator{\sign}{sign}
\DeclareMathOperator{\DE}{DE}
\DeclareMathOperator{\EXP}{EXP}
\DeclareMathOperator{\GA}{GA}
\DeclareMathAlphabet\mathbfcal{OMS}{cmsy}{b}{n}

\newtheorem{theorem}{Theorem}[section]
\newtheorem{definition}{Definition}[section]

\makeatletter
\renewcommand\@memmain@floats{%

% Enable serially numbered figures, tables and equations
\counterwithout{figure}{chapter}
\counterwithout{table}{chapter}
\counterwithout{equation}{chapter}}

% Suppress section numbering in TOC
\renewcommand{\cftsectionpresnum}{\begin{lrbox}{\@tempboxa}}
\renewcommand{\cftsectionaftersnum}{\end{lrbox}}
\makeatletter
\setlength{\cftsectionnumwidth}{0pt}

% Restarts section numbered from 1 on each chapter
\renewcommand{\thesection}{\arabic{section}}
\renewcommand\bibname{REFERENCES}
%%%%%%%%%%%%%%%%%%%%%%%%%%%%%%%%%%%%%%%%%%%%%%%%
\RequirePackage[left=1.5in,right=1in,top=1in,bottom=1in,footskip=.5in]{geometry} 
\begin{document}
% ---------------------------------------
% Front matter
% ---------------------------------------
%Dissertation author 
\author{Patrick J. Trainor}

% Degree
\authordegree{M.S. in Biostatistics, 2014\\M.A. in Mathematics, 2014\\B.S. in Mathematics, 2012}

% Dissertation title
\title{BAYESIAN ANALYTICAL APPROACHES FOR METABOLOMICS: A NOVEL METHOD FOR MOLECULAR STRUCTURE-INFORMED METABOLITE INTERACTION MODELING, A NOVEL DIAGNOSTIC MODEL FOR DIFFERENTIATING MYOCARDIAL INFARCTION TYPE, AND APPROACHES FOR COMPOUND IDENTIFICATION GIVEN MASS SPECTROMETRY DATA}
%\title{Bayesian analytical approaches for metabolomics: a novel method for molecular structure-informed metabolite interaction modeling, a novel diagnostic model for differentiating Myocardial Infarction type, and  approaches for compound identification given mass spectrometry data}

% Dissertation degree
\dissertationdegree{Doctor of Philosophy}

% Dissertation discipline (may differ from the department name)
\dissertationdiscipline{Interdisciplinary Studies:  Bioinformatics}

% Month and year of degree award
\dissertationdate{August 2018}

% Month day and year of dissertation approval
\approvaldate{July 19, 2018}

% College or school
\school{School of Interdisciplinary and Graduate Studies }

% Department in which the candidate is enrolled
\department{School of Interdisciplinary and Graduate Studies}

% Advisors and committee members
\advisor{Shesh N. Rai, Ph.D.}
\firstmember{Andrew P. DeFilippis, M.D., M.Sc.}
\secondmember{Eric C. Rouchka, D.Sc.}
\thirdmember{Juw W. Park, Ph.D.}
\fourthmember{Fourth Member}

% Use the same keywords for the pdf headers above
\keywords{Bayesian Statistics, Metabolomics, Mass Spectrometry, Interactome}

%This builds the title page and other front content
\frontmatter

\maketitle
% Now we add edited content page by page.
% Retain each section, edit as needed.
% Keep this order of sections and end with the tables of contents.

\begin{dedication}
This dissertation is dedicated to my family: Colleen, Jack, Katie, Winn, Aaron, Matt, Abby, Cassie, Sam, Betty, Jim, Jane, and Edmond; to the ESL students at the BLC; and to the 3\textsuperscript{rd}-5\textsuperscript{th} grade students at the BLC.
\end{dedication}

\begin{acknowledgments}
Acknowledgments are due to the past and present members of the Atherosclerosis / Atherothrombosis Research Laboratory (AARL), especially: Dr. Andrew DeFilippis, Dr. Alok Amraotkar, Dr. Mahrokh Nokhbehzaeim, Amanda Coulter, Allison Smith, Sharon Vincent, Ayesha Singh, Mallory Hatfield, and Dr. Yong Siow. In addition to the laboratory, special thanks are due to the human participants who have graciously agreed to participate in the clinical studies conducted by the AARL to further research into the leading cause of global mortality: heart disease.

Acknowledgments are due to collaborators who have assisted in this work including: Samantha Carlisle, Dr. Hunter Moseley, and Joshua Mitchell.

Acknowledgments are due to Dr. Shesh Rai, Dr. Aruni Bhatnagar, and Dr. Andrew DeFilippis for providing mentorship and support. 

Finally, acknowledgments are due to the following organizations for providing financial support: the American Heart Association (11CRP7300003), the National Institutes of Health
(1P20 GM103492-06 \& Common Fund metabolomics pilot and feasibility award), and the Alpha Phi Foundation (2016 Heart-to-Heart Grant).
\end{acknowledgments}

% Dissertation abstract 
\begin{dissertationabstract}
\begin{center}
	BAYESIAN ANALYTICAL APPROACHES FOR METABOLOMICS: A NOVEL METHOD FOR MOLECULAR STRUCTURE-INFORMED METABOLITE INTERACTION MODELING, A NOVEL DIAGNOSTIC MODEL FOR DIFFERENTIATING MYOCARDIAL INFARCTION TYPE, AND APPROACHES FOR COMPOUND IDENTIFICATION GIVEN MASS SPECTROMETRY DATA \\
	Patrick J. Trainor \\
	July 19, 2018
\end{center}
Metabolomics, the study of small molecules in biological systems, has enjoyed great success in enabling researchers to examine disease-associated metabolic dysregulation and has been utilized for the discovery biomarkers of disease and phenotypic states. In spite of recent technological advances in the analytical platforms utilized in metabolomics and the proliferation of tools for the analysis of metabolomics data, significant challenges in metabolomics data analyses remain. In this dissertation, we present three of these challenges and Bayesian methodological solutions for each. In the first part we develop a new methodology to serve a basis for making higher order inferences in metabolomics, which we define as the testing of hypotheses that are more complex than single metabolite hypothesis tests. This methodology utilizes informative priors that are generated via the analysis of molecular structure similarity to enable the estimation of metabolite ``interactomes'' (or probabilistic models) which are organism-, sample media-, and condition-specific as well as comprehensive; and that can serve as reference models for studying perturbations in metabolic systems. After discussing the development of our methodology, we present an evaluation of its performance conducted using simulation studies, and we use the methodology for estimating a plasma metabolite interactome for stable heart disease. This interactome may serve as a reference model for evaluating systems-level changes that occur with acute disease events such as myocardial infarction (MI) or unstable angina. In the second part of this work, we present the challenge of developing diagnostic classification models which utilize metabolite abundances and that do not ``overfit'' relatively small sample sizes, especially given the high dimensionality of metabolite data acquired using platforms such as liquid chromatography-mass spectrometry. We use a Bayesian methodology for estimating a multinomial logistic regression classifier for the detection and discrimination of the subtype of acute myocardial infarction utilizing metabolite abundance data quantified from blood plasma. As heart disease is the leading cause of global mortality, a blood-based and non-invasive diagnostic test that could differentiate between MI types at the time of the event would have great utility. In the final part of this dissertation we review Bayesian approaches for compound identification in metabolomics experiments that utilize liquid chromatography-mass spectrometry which remains a challenging problem.
\end{dissertationabstract}

% These commands will generate the required tables of contents. 
% The \clearpage commands insure that the next one begins on a new page.
{\newgeometry{top=1in,left=1.5in,bottom=1in,footskip=.5in} \setlength{\beforechapskip}{.75in}  \tableofcontents* \clearpage}
{\newgeometry{top=1in,left=1.5in,bottom=1in,footskip=.5in} \setlength{\beforechapskip}{.75in} \listoftables \clearpage}
{\newgeometry{top=1in,left=1.5in,bottom=1in,footskip=.5in} \setlength{\beforechapskip}{.75in} \listoffigures \clearpage}

% ---------------------------------------------
% Main matter
% ---------------------------------------------
% Note we need this to insure proper formatting for the body of the dissertation.
\mainmatter

% Use separate files for chapters and include the chapter titles here.
% Having the titles here is necessary to get them properly indexed and 
% formatted.

\chapter{Introduction}
\begin{DoubleSpace*}
\section{Challenges in metabolomic data analyses}
Metabolomics, or the study of small molecules in biological systems, has enjoyed great success in enabling researchers to make inferences regarding disease associated metabolic dysregulation as well as in furnishing biomarkers of disease and phenotypic states \cite{dunn2013,carlisle2016,johnson2016,newgard2017}. In spite of recent technological advances in the analytical platforms utilized in metabolomics \cite{gowda2014,gowda2016}, advances in strategies for studying disease associated changes in metabolism \cite{bruntz2017,krycer2017}, and advances in the democratization (such as by web-based tools) of metabolomics data analyses \cite{warth2017,guijas2018,chong2018}, significant challenges in metabolomics data analyses remain. The first challenge that we confront in the current work is making ``higher order inference'' in metabolomics (Chapters~\ref{higher},\ref{structureADBGL}, and \ref{hdInteractome}). Prior to arriving at our discussion of higher order inference in metabolomics, we introduce metabolism and metabolomics more generally in Chapter~\ref{prelimMetab}. We define higher order inference as the testing of hypotheses that are more complex than single metabolite hypothesis tests. In studying metabolism, research questions are often over systems of metabolites or metabolic processes. For example, a biomedical scientist might wish to know how a statin impacts cholesterol metabolism. Likewise a cancer researcher may wish to know how the knockdown or knockout of a specific gene alters cellular metabolism. Formulating such research questions as hypothesis tests is not straightforward. The first step in developing such statistical tests is developing a reference model for the system of interest. In the example of a biomedical scientist studying a statin, a reference model of how the metabolites of cholesterol metabolism interact (probabilistically) is required in order to determine how the system has changed following the treatment of a human (or model organism) with a statin. As we argue in Chapter~\ref{higher}, the current paradigm of pathway-based or \emph{a priori} knowledge-based enrichment analyses may suffer from extreme bias, especially in the case of analyzing biofluids such as blood plasma, serum, or urine. Unbiased, or data-dependent approaches such as the construction of correlation networks also suffer from significant methodological issues, as we discuss (Chapter~\ref{higher}). Consequently, we propose a new approach that allows for the construction of metabolite ``interactomes'' or probabilistic models which are organism and sample media specific that can be used as reference models for studying perturbations in metabolic systems. Given the high-dimensionality of metabolite abundance data from untargeted metabolomics experiments, we propose utilizing informative priors that are generated via the analysis of molecular structure similarity in the estimation of such models. 

In the second part of this work, we confront the challenge of developing diagnostic classification models utilizing metabolite abundances that do not ``overfit'' relatively small sample sizes. Often poor generalization error results from unrestricted optimization of likelihoods in models with a high ratio of model parameters to sample size of a training dataset \cite{hastie2009}. This is often referred to as the $p>>n$ problem (where, in this context, $p$ is the number of metabolites and $n$ is the number of samples). Many metabolomics datasets are characterized by a large $p$ (e.g. $>30,000$ mass features can be observed in high resolution mass spectrometry experiments), while the substantial costs associated with the acquisition of metabolomics data implies that small datasets are a persistent feature of the field. As more resources are devoted to personalized medicine \cite{hamburg2010,wishart2016} and deep phenotyping \cite{delude2015}, diagnostic classifiers that are robust given $p>>n$ and minimize the likelihood of overfitting are essential. 

In Chapter~\ref{diagnostic} we discuss such a classification methodology, applied to a critical clinical problem: the detection and discrimination of the subtype of acute myocardial infarction (colloquially: heart attack) utilizing metabolite abundance data quantified from blood plasma. Heart disease is the leading cause of global mortality \cite{benjamin2017}. Myocardial Infarction (MI), an acute manifestation of heart disease, is characterized by heterogeneous etiology \cite{thygesen2012}. Consequently, a blood-based and non-invasive diagnostic test that could be administered upon presentation to an emergency department and could differentiate between thrombotic MI and non-thrombotic MI would be of great utility. Our laboratory (the Atherosclerosis / Atherothrombosis Research Laboratory [AARL] at the University of Louisville) has recruited two clinical cohorts that are designed for the development and validation of a diagnostic test capable of differentiating thrombotic MI, non-thrombotic MI, and stable coronary artery disease; the first cohort has been described previously \cite{defilippis2015,defilippis2017,trainor2017}. As a critical step towards the development of such a diagnostic, in Chapter~\ref{diagnostic} we present multinomial logistic regression models utilizing the metabolite abundances for the metabolites that were selected as part of a separate feature selection work \cite{trainor2018}. We compare the performance of the Bayesian approach we have employed for estimating model coefficients with maximum likelihood estimation and discuss the merits of the Bayesian approach. 

\section{Bayesian approaches}
Blah

\section{Why a bayesian approach to metabolomics}

\section{Aim 1 Introduction}

\section{Aim 2 Introduction}
\end{DoubleSpace*}

\chapter{Preliminaries: Metabolomics}
\doublespacing
\section{Metabolism} [Need citations]

Metabolism is broadly defined as the chemical reactions that enable life [cite]. The metabolic reactions that sustain life can be categorized as catabolic reactions, or the reactions that entail breaking down and oxidizing molecules to provide energy or substrates for other reactions, and anabolic reactions in which complex molecules are synthesized. Series of coupled reactions are referred to as pathways. Anabolic pathways are characterized by endergonic processes as the synthesis of macromolecules from smaller molecules requires energy. Conversely catabolic pathways entail the break down of larger molecules into smaller molecules for use either as inputs in anabolic reactions or for the release of free energy via oxidation and reduction reactions. 

\begin{figure}[ht!]
	\resizebox{\textwidth}{!}{\includegraphics*{./Plots/tca.png}}
	\caption[TCA cycle pathway diagram]{ Diagram of the tricarboxylic acid (TCA) cycle in \emph{Homo sapiens} [add citation].\label{fig:tca} }
\end{figure}

An example catabolic process is the tricarboxylic acid (TCA) cycle which is illustrated in Fig.~\ref{fig:tca}. In the TCA cycle an input molecule of Acetyl-CoA (from the catabolism of glucose, fats, or proteins) enters into a sequence of reactions which produces three molecules of NADH, one molecule of FADH\textsubscript{2}, and one molecule of GTP. The molecules of NADH and FADH\textsubscript{2} produced during TCA cycle also serve as substrates for the electron transport chain which generates additional molecules of ATP. 

An example anabolic pathway is the cholesterol biosynthesis pathway shown in Fig.~\ref{fig:tca}. In this pathway, acetyl-CoA molecules as a substrate are chemically transformed through a sequence of reactions with intermediates including HMG-CoA, mevalonate, isopentenyl phosphate, squalene, lanosterol, and finally cholesterol. The cholesterol biosynthesis pathway entails over 20 steps, and requires ATP as an energetic input and the cofactors NADH and NADPH. 

\begin{figure}[ht!]
	\resizebox{\textwidth}{!}{\includegraphics*{./Plots/cholesterol.png}}
	\caption[Cholesterol biosynthesis pathway diagram]{ Diagram of the cholesterol biosynthesis pathway in \emph{Homo sapiens}. [add citation] \label{fig:chol} }
\end{figure}

While many of the biochemical processes of metabolism, including glucose catabolism, glycogen metabolism, gluconeogenesis, lipid metabolism, TCA cycle, electron transport chain, nucleotide metabolism, and protein synthesis  take place in the cell, many of the signaling processes that control metabolism as well as many systems of transport are extracellular \cite{voet2013}. As an example, consider lipid metabolism. While the catabolic process of breaking down fatty acids to yield energetic substrates, $\beta$-oxidation, takes place in the mitochondrial matrix of a cell, many of the processes prior to the point of $\beta$-oxidation are extracellular or take place in distinct cells from the cell in which a specific fatty acid molecule is being broken down. Within adipocytes (specialized cells for the storage of fatty acids), fatty acids are stored as triacylglycerols. When activated by hormone sensitive lipase, triacylglycerols are hydrolyzed yielding free fatty acids which are released into the bloodstream. These fatty acid molecules may be transported in blood in complex with the protein albumin, and may be taken up by other cells (especially by hepatocytes in the liver) for fatty acid oxidation. In addition to the oxidation of fatty acids to yield citrate and acetyl-CoA, cells (especially hepatocytes) also synthesize fatty acids for storage to meet future energetic needs. Newly synthesized fatty acid molecules may be transported in the bloodstream in the form of triacylglycerols within chylomicrons as well as in VLDL. These molecules can be utilized to transport fatty acids for storage in adipocytes. 

LOH control of fatty acid metabolism

\section{From metabolism to metabolomics}
While multiple definitions of metabolomics have been proposed, the fundamental conceptual definition is that metabolomics is the study of small molecules in a biological sample \cite{nicholson2008}. The term metabonomics has previously been used almost synonymously with metabolomics, although the focus of this discipline as posited previously study of the changes in metabolic processes following experimental manipulation. Given theses definitions, it can be noted that metabonomics requires metabolomics, that is in order to study changes in metabolic processes that follow an experimental manipulation, one must measure small molecules from biological samples. In other words, ``metabonomics'' represents the goal of many ``metabolomics'' studies. For the purposes of the current work we use the term ``metabolomics'' to refer to both the analysis of small molecules from a sample, as well as the study of changes in metabolic processes that follow an experimental manipulation.

\section{Analytical chemistry}

\chapter{Preliminaries: Bayesian statistics}
\label{prelimBayesian}
\begin{DoubleSpace*}
\section{Bayes rule}

The central foundation of Bayesian statistics is Bayes rule \cite{gelman2004}. 
\begin{theorem} If $A$ and $B$ are events with $\PP(B)>0$, then:
	\begin{gather}
		\PP(A|B) = \frac{\PP(B|A)\PP(A)}{\PP(B)}
	\end{gather}.	
\end{theorem}

Further, Bayes rule can be extended to any arbitrary partitioning of a sample space \cite{casella2002}.
\begin{theorem}
	Let $\mathcal{P}=\{A_i: i=1,2, \hdots, N\}$ represents an arbitrary partitioning of a sample space $S$, then for a specific $A_i$:
	\begin{gather}
			 \PP(A_i|B)=\frac{\PP(B|A_i)P(A_i)}{\sum_{A_j\in \mathcal{P}}\PP(B|A_j)P(A_j)}
	\end{gather}.
\end{theorem}

\section{Bayesian inference regarding a parameter}
Let $\theta$ (or $\boldsymbol{\theta}$ in the multivariate case) represent a parameter of a probability mass function (pmf) or probability distribution function (pdf)  $f(x|\theta)$ of a random variable $X$. Without loss of generality, we discuss the univariate case of one random variable $X$ and one population parameter $\theta$, although multivariate generalization is straightforward. The objective of statistical inference regarding $\theta$ is to utilize a random sample $X_1, X_2, \hdots X_n$ from a population with pmf/pdf $f(x|\theta)$ to determine likely values of $\theta$ \cite{casella2002}. Denoting a fixed sample as $\textbf{x}=\{x_1, x_2,\hdots, x_n \}$, we note that $\textbf{x}$ is a realization from the sample space $\mathcal{X}$ of all possible samples from the population. In Bayesian inference, the population parameter $\theta$ is regarded as unknown \cite{hoff2009}. It is assumed that the uncertainty regarding the true population value of $\theta$, can be represented by the prior probability distribution $p(\theta)$. Since the true population value is unknown, a set of possible values for $\theta$, that is the parameter space $\Theta$ comprises the support of $p(\theta)$. In Bayesian analysis, a sampling model describing the probability of observing realization $x$ of the random variable $X$ conditioned on a fixed value of the parameter $\theta$, that is $p(x|\theta)$ must also be specified. When considering a fixed sample, this term is also referred to as the  likelihood \cite{casella2002}. 
\begin{theorem}
	 If $X_1, X_2, \hdots, X_n$ are independent and identically distributed (iid) random variables, then $f(x_1,x_2,\hdots, x_n|\theta)=\prod_{i=1}^n f(x_i|\theta)$.
\end{theorem}
More specifically, we refer to the joint distribution function of a sample $x$, conditional on the value of the parameter $\theta$ as the likelihood of theta given the sample, or $\mathcal{L}(\theta|\textbf{x})$. Given a prior distribution for $\theta\in \Theta$, a random sample from the population $\textbf{x}$,and a sampling model or likelihood, the posterior distribution for $\theta$ can then be determined utilizing Bayes rule, as below\cite{hoff2009}:
\begin{gather}
	p(\theta|\textbf{x})=\frac{p(\textbf{x}|\theta)p(\theta)}{\int_{\tilde{\theta} \in \Theta} p(\textbf{x}|\tilde{\theta})p(\tilde{\theta})d\tilde{\theta}}=\frac{p(\textbf{x}|\theta)p(\theta)}{p(\textbf{x})}.
\end{gather}
The denominator term is referred to as the marginal likelihood. As the marginal likelihood involves only a fixed sample, the denominator is a constant, which justifies the following relation involving the posterior distribution of $\theta$:
\begin{gather}
	p(\theta|\textbf{x}) \propto p(\textbf{x}|\theta)p(\theta). 
\end{gather}
From this relation, it can be noted that the posterior distribution for $\theta$, after observing a random sample $\textbf{x}$ is proportional to the likelihood times the prior.

\section{Example Bayesian inference regarding a parameter}
Consider a mass spectrometer with a counting detector that records the counts per second over a fixed $m/z$ window. The number of counts recorded by the detector over a fixed time interval may be a Poisson process. Let $X$ be the random variable representing the number of counts recorded by the detector over a one second time interval, with an average number of counts per second (population parameter) of $\lambda$. We then say that $X\sim Pois(\lambda)$, and note that the probability mass function for $X$ is:
\begin{gather}
	f(x|\lambda)=\frac{\lambda^x e^{-\lambda}}{x!}, \quad x \in \mathbb{N}, \quad \lambda \in \left[0, \infty \right)
\end{gather}
 Assume a sample has been observed with the following counts: $\{107, 103, 110, 99, 108, 108, 105\}$ and that we wish to determine plausible values of the rate parameter $\lambda$. In order to determine plausible values, we will conduct Bayesian inference over $\lambda$. Assume that we have collected ten prior count observations and that the sum of counts from these observations was 1,000. We may then choose a prior distribution for $\lambda$ of $Gamma(\alpha, \beta)$ with parameters $\alpha=1000$ and $\beta=10$. We note the following form of the Gamma distribution pdf: 
\begin{gather}
	f(\lambda|\alpha,\beta) =\frac{\beta^{\alpha}}{\Gamma(\alpha)}\lambda^{\alpha-1}e^{-\beta \lambda}, \quad \lambda \in \left[0, \infty \right), \quad \alpha,\beta >0.
\end{gather}
We note that the likelihood for the Poisson distribution factorizes as in the following:
\begin{align}
	\mathcal{L}(\lambda|\textbf{x})=p(\textbf{x}|\lambda)=\prod_{i=1}^n \frac{\lambda^{x_i}e^{-\lambda}}{x_i!}= \left( \prod_{i=1}^n x_i! \right)^{-1} \times \lambda^{\sum_{i=1}^n x_i} e^{-n\lambda}. 
	\label{eq:poissonLik}
\end{align}
From Eq. \ref{eq:poissonLik} we can note that the likelihood term that involves $\lambda$ and are is not fixed by the sample $\textbf{x}$ is $\lambda^{\sum_{i=1}^n x_i} e^{-n\lambda}$. Thus the posterior distribution for $\lambda$ has the following form:
\begin{align}
	p(\lambda|\textbf{x})\propto p(\lambda) p(\textbf{x}|\lambda) \propto p(\lambda)\lambda^{\sum_{i=1}^n x_i} e^{-n\lambda}
\end{align}
Considering the prior distribution that we have specified:
\begin{align}
	p(\lambda|\textbf{x})&\propto \frac{\beta^{\alpha}}{\Gamma(\alpha)}\lambda^{\alpha-1}e^{-\beta \lambda} \lambda^{\sum_{i=1}^n x_i} e^{-n\lambda} \\
	&\propto \frac{\beta^{\alpha}}{\Gamma(\alpha)} \lambda^{\alpha + \sum_{i=1}^n x_i -1} e^{-(\beta+n)\lambda},
\end{align}
we can note that:
\begin{align}
p(\lambda|\textbf{x}) =c(\textbf{x},\alpha,\beta)^{-1} \lambda^{\alpha + \sum_{i=1}^n x_i -1} e^{-(\beta+n)\lambda},
\end{align}
where $c(\textbf{x},\alpha,\beta)$ is a normalizing constant that depends only on the prior distribution parameters and the observed sample. Thus it is evident that the posterior distribution for $\lambda$ is also a Gamma distribution, specifically $p(\lambda|\textbf{x})\sim Gamma(\alpha+ \sum_{i=1}^n x_i, \beta+n)$. Fig.~\ref{fig:pois} shows the pdf of both the prior and posterior probability distributions for the Poisson rate parameter $\lambda$.

\begin{figure}[ht]
	\resizebox{\textwidth}{!}{\includegraphics*{./Plots/Poisson.png}}
	\caption[Example Bayesian estimation of a Poisson rate parameter]{\DoubleSpacing Example Bayesian estimation of a Poisson rate parameter. The plot shows the probability density function of the prior on posterior distributions for the Poisson rate parameter $\lambda$. Dashed blue vertical line shows the mode of the posterior distribution. Also shown as a dashed black vertical line is the maximum likelihood estimator determined from the sample $\textbf{x}$. \label{fig:pois} }
\end{figure}

As can be observed, the center (mode or mean) of the posterior probability distribution for $\lambda$ lies between the maximum likelihood estimator (MLE) for $\lambda$ and the prior distribution. In fact, while the MLE for $\lambda$ is $\hat{\lambda}_{MLE}=105.7143$, the mode of the posterior distribution of $\lambda$ is:
\begin{align*}
	\tilde{\lambda} = \frac{\alpha+ \sum_{i=1}^n x_i -1}{\beta+n} =\frac{1000+105.7143}{10+n} = 102.2941.
\end{align*}

In addition to point-wise summaries (such as the mode of the posterior distribution), interval summaries can be provided utilizing the posterior distribution of a parameter. In this case, a Bayesian credible interval \cite{gelman2004} can be analytically computed using a quantile function as the analytical form of the posterior distribution is known. Specifically, a two-sided 95\% Bayesian credible interval (CI) can be determined by finding the unique solution to $(l,u)$, such that $F(l|\alpha+ \sum_{i=1}^n x_i -1,\beta+n)=0.025$ and $F(u|\alpha+ \sum_{i=1}^n x_i -1,\beta+n)=.975$ where $F(x)$ is the cumulative distribution function for the Gamma distribution. In the current example, this yields a 95\% CI of $(97.600, 107.218)$.
 
\section{Specification of prior distribution}
Conjugate distributions \\

Given the form of the data-augmented posterior distribution, $p(\lambda|\textbf{x})\sim Gamma(\alpha+ \sum_{i=1}^n x_i, \beta+n)$,  the posterior Gamma distribution parameters can be noted as $\alpha'=\alpha+ \sum_{i=1}^n x_i$ and $\beta'=\beta+n$, where $\alpha$ and $\beta$ are the prior parameters. This provides a natural way to encapsulate prior empirical evidence. Noting that prior expectation for $\lambda$ is $\EE(\lambda)=\frac{\alpha}{\beta}$, while the posterior expectation conditional expectation is:
\begin{align}
	\EE(\lambda|\textbf{x})=\frac{\alpha+ \sum_{i=1}^n x_i}{\beta+n} =\frac{\alpha}{\beta+n}+\frac{\sum_{i=1}^n x_i}{\beta+n},
\end{align} it can be seen that if the sum of counts observed in $\beta$ previous experiments (where $\beta$ is a natural number) was $\alpha$, then the conditional expectation of $\lambda$ is a weighted average of the prior observed counts and the current observed counts (weighted by the respective number of observations). 

\section{MCMC methods for inference}

\subsection{Gibbs Sampling}
In addition to not having an analytical form of the joint posterior distribution of model parameters (which justifies the use of posterior simulation) it may be the case that it is straightforward to sample from conditional distributions of parameters but not for sampling from the joint distribution. In such cases, Gibbs sampling may be employed \cite{gelman2004,hoff2009}. The idea of Gibbs sampling is straightforward. First, parameters estimates are initialized to some initial values. Next, one parameter is fixed as the target parameter. A sample is then drawn from the conditional distribution of this target parameter (conditioned on both the non-target parameter estimates and the observed sample). Gibbs sampling can be extended to using target sets of parameters without loss of generality. A more precise definition of a Gibbs sampler is as follows and pseudocode is presented as Alg.~\ref{alg:gibbs}. Let $\boldsymbol{\theta}=\{\theta_1, \theta_2, \hdots,  \theta_p \}$ be a vector of parameters and let $\boldsymbol{\theta}_j$ be a vector defined by a subset of the parameters with the subsets indexed by $j=1, 2, \hdots, J$. Let $\tilde{\boldsymbol{\theta}}_j^{(t)}$ denote the estimate of the subset of parameters at time $t$. 

\begin{algorithm}
	\caption{Gibbs sampler
		\label{alg:gibbs}}
\begin{algorithmic}[1]
	\STATE Set $\boldsymbol{\theta}^{(0)}$ to some initial values
	\WHILE{$t\leq t_{max}$ or convergence criteria not met}
		\STATE Set $\tilde{\boldsymbol{\theta}} \leftarrow \tilde{\boldsymbol{\theta}}^{(t-1)}$
		\FOR{$j=1$ to $J$}
			\STATE Sample $\tilde{\boldsymbol{\theta}}_j^{*}\sim p \left(\boldsymbol{\theta}_j|\tilde{\boldsymbol{\theta}}_{-j}, \textbf{x} \right)$
			\STATE Update $\tilde{\boldsymbol{\theta}}_j \leftarrow \tilde{\boldsymbol{\theta}}_j^{*}$
		\ENDFOR
		\STATE Set $\tilde{\boldsymbol{\theta}}^{(t)} \leftarrow \tilde{\boldsymbol{\theta}}$ and return $\tilde{\boldsymbol{\theta}}^{(t)} $
		\STATE $t \leftarrow t+1$
	\ENDWHILE
\end{algorithmic}
\end{algorithm} 

\subsection{Example application of Gibb's sampling}
Continuing with the example of a mass spectrometer with a counting detector, we illustrate the use of Gibb's sampling for simulating the posterior distribution of a set of model parameters. In the previous example, we assumed that there was only one population rate parameter. In the current example, we consider the fact that over a fixed m/z window, counts may be observed or not observed based on whether an ion is present. If the ion is not present, we expect to observe a count of zero; if the ion is present we expect the counts to follow a Poisson distribution. This can be formulated as the following hierarchical model:
\begin{align}
	Z|p,\lambda \sim Bernoulli(p) \\
	X|p,\lambda,z \sim F(x|z,\lambda)
\end{align}
where
\begin{align}
f(x; z, \lambda) =\frac{\lambda^x e^{-\lambda}}{\lambda!} I_{\{z=1\}}(z)+ I_{\{z=0\}}
\end{align}
\end{DoubleSpace*}

\chapter{Preliminaries: Gaussian graphical modeling}

\section{Introduction}

Across multiple domains including finance, economics, molecular biology, and machine vision, stochastic systems are observed via the realization of multiple random variables and the determination of the relationship between the random variables is an essential inferential task. For example, researchers in the field of metabolomics often sample from the repertoire of small molecules contained in a cell or biofluid to make inference regarding metabolic responses to the environment or differences across phenotypes \cite{johnson2016}. Central to making inferences regarding metabolic processes is determining the structure of probabilistic interactions between sets metabolites. In macroeconomics, to quantify financial risk that could propagate across  borders, international bank settlements may be sampled and interrogation of the dependence structure of these random variables reveals cross border flows that may result in cross-border contagion during a financial crisis \cite{giudici2016}.

\section{Gaussian graphical models}
An undirected probabilistic graphical model also known as a Markov Random Fields (MRFs) is a graph $G=(V,E)$ in which random variables $X_i\in V$, $i\in \{1,2,...p\}$ are represented by vertices and edges in the edge set $E \subseteq V \times V$ represent probabilistic interactions  \cite{koller2009}. If the joint distribution of the random variables  is assumed to be a multivariate normal distribution, that is $\textbf{X}\sim \mathcal{N}(\boldsymbol{\mu},\boldsymbol{\Omega}^{-1})$ where $\boldsymbol{\Omega}$ is the concentration matrix and the inverse of the covariance matrix, then the MRF is a Gaussian Graphical Model (GGM), in which each vertex $X_i$ has a marginal normal distribution, and normal conditional distributions $X_i|X_j$. We immediately note the likelihood given a sample $\textbf{X}=(\textbf{x}_1,\textbf{x}_1,\hdots,\textbf{x}_n)^T$:
\begin{align}
L(\boldsymbol{\Omega}|\textbf{X})&=(2 \pi)^{-np/2}|\boldsymbol{\Omega}|^{n/2} \exp \left(-\frac{1}{2}\sum_{i=1}^{n} (\textbf{x}_i-\boldsymbol{\mu})^T \boldsymbol{\Omega} (\textbf{x}_i-\boldsymbol{\mu})\right) \\
&=(2 \pi)^{-np/2}|\boldsymbol{\Omega}|^{n/2} \exp \left(-\frac{1}{2} \langle \textbf{K}, \boldsymbol{\Omega}\rangle\right)
\end{align}
where $\sum_{i=1}^{n}$

In order to determine a Gaussian Graphical model, the graph topology and parameters must be estimated separately \cite{meinshausen2006} or jointly \cite{friedman2007,yuan2007,banerjee2008}. Given a mean centered data matrix $\textbf{X}$ with $\dim(\textbf{X})=n\times p$, estimation of the concentration matrix (inverse of the joint distribution covariance) $\boldsymbol{\Omega}=\boldsymbol{\Sigma}^{-1}$ fully determines the graph topology as well as the multivariate Gaussian distribution parameters. The entries of $\boldsymbol{\Omega}$ are of particular importance; $\omega_{ij}$ is the partial correlation between $X_i$ and $X_j$. Consequently, $\omega_{ij}=0$ implies $X_i$ and $X_j$ are conditionally independent.

\subsection{The Graphical Lasso}
To find the maximum likelihood estimator of $\boldsymbol{\Omega}$ the log likelihood of the concentration matrix is noted:
\begin{align} 
l(\boldsymbol{\Omega})\propto\log (\det \boldsymbol{\Omega})-\tr \left( \textbf{S} \boldsymbol{\Omega} \right),
\end{align}
where $\textbf{S}=\frac{1}{n}\textbf{X}^T \textbf{X}$ is the empirical covariance matrix. In the case that $p>n$ maximization of the log likelihood function is not guaranteed to be convex [check]. To overcome this problem, several approaches have been proposed for maximizing the $L_1$ norm penalized log-likelihood \cite{friedman2007,yuan2007,banerjee2008}:
\begin{align}
l(\boldsymbol{\Omega})\propto\log (\det \boldsymbol{\Omega})-\tr \left( \textbf{S} \boldsymbol{\Omega} \right)-\rho ||\boldsymbol{\Omega}||_1
\label{eq:likelihood}
\end{align}
where $\rho$ is the penalty parameter and the optimization is over the space of positive definite matrices of the same dimension as $\boldsymbol{\Omega}$.
The solution proposed by \cite{friedman2007}, known as the graphical Lasso employs a block coordinate descent for maximizing the penalized likelihood. To formulate a block-wise procedure the \cite{friedman2007} first note the following partitioning along the last row and last column of $\textbf{S}$ and $\textbf{W}=\hat{\boldsymbol{\Sigma}}$:
\begin{align}
\textbf{W}=
\begin{bmatrix}
\textbf{W}_{11} & \textbf{w}_{12} \\
\textbf{w}_{12}^T & w_{22}.
\end{bmatrix}, \quad
\begin{bmatrix}
\textbf{S}_{11} & \textbf{s}_{12} \\
\textbf{s}_{12}^T & s_{22}.
\end{bmatrix}
\end{align}  
Given this partition, the maximization of the log-likelihood (Eq. \ref{eq:likelihood}) is equivalent to the following constraint problem \cite{banerjee2008}:
\begin{align}
\textbf{w}_{12}=\argmin_{\textbf{y}} \{ \textbf{y}^T \textbf{W}_{11} ^{-1} \textbf{y} : ||\textbf{y}-\textbf{s}_{12}||_{\infty} \leq \rho \}.
\end{align}
\cite{banerjee2008} show that by convex duality, if $\hat{\beta}$ minimizes:
\begin{align}
\hat{\beta}=\argmin_{\beta} \{\frac{1}{2} ||\textbf{W}_{11}^{1/2}\beta-b||^2 +\rho ||\beta||_1 \}
\label{eq:dual}
\end{align}
where $b=\textbf{W}_{11}^{-1/2} \textbf{s}_{12}$, then $\textbf{w}_{12}=\textbf{W}_{11} \hat{\beta}$.

Critically, \cite{friedman2007} [LOH]

[Add lasso problem and lambda optimization]

\subsection{The SCAD penalty and the Adaptive Graphical Lasso}
It has been shown that the linear increase penalization relative to the norm incurred with $L_1$ regularization of parameters introduces bias, especially in the case of parameters that have large magnitude \cite{fan2009,lam2009}. Two alternative penalized likelihoods have been proposed as a solution to this known problem and have been shown to satisfy the oracle property: the smoothly clipped absolute deviation (SCAD) penalization and the adaptive lasso. The oracle property states that an optimal estimator of [LOH] and it has been shown that the oracle property is not universally satisfied by the lasso \cite{zou2006}. The lasso, SCAD penalized estimation, and adaptive lasso share similar developmental histories, being developed first for variable selection and linear/generalized linear model parameter estimation with later extension to GGM parameter estimation. The smoothly clipped absolute deviation is a non-concave penalty function defined as \cite{fan2001}:
\begin{align}
f_{\lambda,a}(\theta)=
\begin{cases} 
\lambda |\theta| & \text{if} \; |\theta|\leq \lambda \\
-\left(\frac{|\theta|^2 -2a \lambda |\theta|+\lambda^2}{2(a-1)}\right)& \text{if}\; |\theta| \in (\lambda,a \lambda] \\
\frac{(a+1)\lambda^2}{2} & \text{if} \; |\theta|>a \lambda 
\end{cases}
\end{align}
The behavior of this penalty with respect to coefficient magnitude can be examined given the continuous derivative derivative of the SCAD penalty function:
\begin{align}
f'_{\lambda,a}(\theta)=\lambda \left[I(\theta \leq \lambda) + \frac{(a \lambda-\theta)_+}{(a-1)\lambda}I(\theta>\lambda)\right]
\end{align}
for $\theta > 0$ and $\alpha>2$. In general, the advantage of this penalization over $L_1$ norm penalization is that large values of $\theta$ are not excessively penalized. [LOH: Oracle property and behavior of the derivative] The two parameters of the penalty function can be optimized by cross-validation or via minimization of Bayes risk as was done in \cite{fan2001}. To generalize the SCAD penalty for regularized estimation of GGMs, the SCAD penalized log-likelihood is noted [LOH]:
\begin{align*}
l(\boldsymbol{\Omega}) \propto \log(\det \boldsymbol{\Omega})-\tr \left(\frac{\textbf{S}}{n} \boldsymbol{\Omega} \right) - \sum_{i=1}^{p} \sum_{j=1}^{p} f_{\lambda,a} (|\omega_{ij}|)
\end{align*}

After demonstrating the conditions in which lasso variable selection is not guaranteed to be consistent and hence not satisfying the oracle property, \cite{zou2006} modified the $L_1$ lasso penalty $\lambda ||\boldsymbol{\theta}||_1$ to incorporate parameter specific weights $\textbf{w}$ [LOH] yielding penalty $\lambda ||\textbf{w} \boldsymbol{\theta}||_1$. The penalized likelihood for the adaptive graphical lasso is then:
\begin{align}
\log(\det \boldsymbol{\Omega})-\tr \left(\frac{\mathbf{S}}{n} \boldsymbol{\Omega}\right) - \lambda \sum_{1\leq i \leq p} \sum_{1 \leq j \leq p} w_{ij} |\omega_{ij}|.
\end{align}
In this likelihood, the weights that contribute to the penalization are $w_{ij}=|\hat{\omega}_{ij}|^\alpha$ for a fixed $\alpha >0$ and a consistent estimate of the concentration matrix with entries $\hat{\omega}_{ij}$.

\subsection{The Bayesian Graphical Lasso}
A Bayesian interpretation of the regular graphical lasso \cite{friedman2007} has been shown previously \cite{wang2012}. Given the following hierarchical model:
\begin{align}
p(\textbf{x}_i|\boldsymbol{\Omega}) =& \mathcal{N}(\textbf{0},\boldsymbol{\Omega}^{-1}) \quad \text{for} \; i=1,2,\hdots,n\\
p(\boldsymbol{\Omega}|\lambda) =& \frac{1}{C} \prod_{i<j} \DE(\omega_{ij}|\lambda) \prod_{i=1}^{p} \EXP (\omega_{ii} | \lambda / 2) \cdot 1_{\boldsymbol{\Omega}\in M^+},
\label{eq:bGLassoModel}
\end{align}
it has been demonstrated the mode of the posterior distribution of $\boldsymbol{\Omega}$ is the graphical lasso estimate given penalty parameter $\rho=\lambda/n$. In this model, the prior distribution of the off-diagonal entries of the concentration matrix follow a double exponential distribution centered at zero with scale parameter $\lambda$, while the prior distribution of the diagonal entries of the concentration matrix is exponential with scale parameter $\lambda/2$. \cite{wang2012}  noted that the hierarchical model in (\ref{eq:bGLassoModel}) can be represented as a scale mixture of normal distributions \cite{andrews1974,west1987} leading to the following prior distribution:
\begin{align}
p(\boldsymbol{\omega}| \boldsymbol{\tau},\lambda)=\frac{1}{C_{\boldsymbol{\tau}}} \prod_{i<j} \left[ \frac{1}{\sqrt{2\pi \tau_{ij}}} \exp \left(- \frac{\omega_{ij}^2}{2\tau_{ij}}\right) \right] \prod_{i=1}^{p} \left[\frac{\lambda}{2} \exp \left(-\frac{\lambda}{2}\omega_{ii} \right)\right] \cdot 1_{\boldsymbol{\Omega}\in M^+}
\end{align}

\cite{wang2012} exploited this representation to develop a block Gibbs sampler for simulating the posterior distribution. This sampler is predicated on identical partitioning of the concentration matrix $\boldsymbol{\Omega}$, products matrix $\boldsymbol{S}$, and latent scale parameters matrix $\mathbfcal{T}$:
\begin{align}
\begin{bmatrix}
\boldsymbol{\Omega}_{11} & \boldsymbol{\omega}_{12} \\
\boldsymbol{\omega}_{12}^T & \omega_{22}
\end{bmatrix},
\begin{bmatrix}
\mathbf{S}_{11} & \mathbf{s}_{12} \\
\mathbf{s}_{12}^T & s_{22}
\end{bmatrix},
\begin{bmatrix}
\mathbfcal{T}_{11} & \boldsymbol{\tau}_{12} \\
\boldsymbol{\tau}_{12}^T & \tau_{22}
\end{bmatrix}.
\end{align}
The Gibbs sampler then samples from the conditional distribution of the last column, $(\boldsymbol{\omega}_{12}, \omega_{22})^T$ of $\Omega$:
\begin{align}
p(\boldsymbol{\omega}_{12}, \omega_{22}|\boldsymbol{\Omega}_{11},\mathbfcal{T},\textbf{X},\lambda) \propto \left(\omega_{22}-\boldsymbol{\omega}_{12}^T \boldsymbol{\Omega}_{11}^{-1}\boldsymbol{\omega}_{12} \right)^{n/2} \exp \{ - \frac{1}{2}\left[ \boldsymbol{\omega}_{12}^T \textbf{D}_{\boldsymbol{\tau}} \boldsymbol{\omega}_{12}+ 2 
\textbf{s}_{12}^T \boldsymbol{\omega}_{12} + (s_{22}+\lambda)\omega_{22}\right] \}
\end{align}

In the same work, \cite{wang2012} proposed a Bayesian approach to the adaptive graphical lasso developed by \cite{fan2009}. The hierarchical model proposed by for the adaptive lasso is:
\begin{align}
p(\mathbf{x}_i|\boldsymbol{\Omega}) = & \mathcal{N}(\mathbf{0,\boldsymbol{\Omega}}^{-1}) \quad \text{for} \; i=1,2,\hdots,n\\
p(\boldsymbol{\Omega}|\{\lambda_{ij}\}_{i\leq j}) = & C^{-1} \prod_{i<j} \DE(\omega_{ij}|\lambda_{ij}) \prod_{i=1}^{p} \EXP (\omega_{ii} | \lambda_{ii} / 2) \cdot 1_{\boldsymbol{\Omega}\in M^+}\\
p(\{\lambda_{ij}\}_{i<j}|\{\lambda_{ii}\}_{i=1}^p) &\propto C_{\{\lambda_{ij}\}_{i\leq j}} \prod_{i<j} \GA(r,s)
\end{align}
As with the frequentist adaptive graphical lasso, the Bayesian adaptive graphical lasso proposed by \cite{wang2012} incorporates differential shrinkage of entries $\omega_{ij}$ according to an estimate $\hat{\omega}_{ij}$. However, as opposed to a fixed relationship between the norm of the current estimate $\hat{\omega}_{ij}$ and the size of the shrinkage parameter $\lambda$ uncertainty about $\lambda$ is incorporated by assuming a gamma distribution for $\lambda_{ij}$. Conditional on the concentration matrix, $\lambda_{ij}$ are distributed as:
\begin{align}
\lambda_{ij}|\boldsymbol{\Omega}\sim \GA(1+r,|\omega_{ij}|+s)
\end{align}

Given that the Bayesian adaptive graphical lasso allows for differential shrinkage for each $\omega_{ij}$ via shrinkage parameter $\lambda_{ij}$ drawn from a non-informative distribution, it is natural to suppose that adaptive penalization would allow for the incorporation of apriori knowledge about the conditional relationship between $X_i$ and $X_j$. \cite{peterson2013} propose that rather than fixing a non-informative prior gamma distribution scale $s$ parameter, this parameter could capture prior belief regarding the conditional relationship between $X_i$ and $X_j$. They assume that if an a priori defined unweighted graph describes the biological relationship between random variables $X_i$, then  the pairwise distance between vertices $X_i$ and $X_j$, that is $d(X_i,X_j)$, could be used as a hyperparameter for $s_{ij}$. In their specific case they set:
\begin{align}
s_{ij}=
\begin{cases} 
d_{ij}^{-1} \cdot  10^{-6+c} & \text{if} \; d_{ij}<\infty \\
10^{-6} & \text{if} \; d_{ij}=\infty
\end{cases},
\end{align}
where $d_{ij}=d(X_i,X_j)$ was defined as the minimum path distance determined via breadth-first search, and $c\in (0,6)$ is a positive constant. This specification of the gamma scale parameter has the effect of shifting the mean of the prior distribution for the penalty parameter $\lambda_{ij}$ towards zero when verteces $X_i$ and $X_j$ are close with respect to the a priori graph.

An important contrast between the frequentist and Bayesian graphical lasso is that given a continuous prior distribution, the Bayesian lasso does not possess and innate edge selection ability. Specifically, as the Bayesian graphical lasso given a continuous prior places positive probability on the posterior entries of the concentration matrix, $\omega_{ij}$, a graph selected from the posterior distribution of $\boldsymbol{\Omega}$ will be fully connected, that is an edge will exist between each pair of vertices of the graph $G$. Heuristics are for conducting edge selection are discussed in both \cite{wang2012}  and \cite{peterson2013}. The selection operator discussed in \cite{wang2012} was based on a heuristic appeal to the argument made in \cite{carvalho2010}.  \cite{carvalho2010} discusses a discrete mixture model:
\begin{align}
\theta_i \sim (1-p)\delta_0 + pg(\theta_i)
\end{align}
where $\delta_0$ is the Dirac distribution and $p$ is the prior mixing probability. Under this model the posterior mean of $\theta_i$ is then:
\begin{align}
\EE(\theta_i|\textbf{X})=\PP(\theta_i\neq 0|\textbf{X})\EE_g(\theta_i|\textbf{X},\theta_i\neq 0).
\end{align}
\cite{wang2012} then claims that using the Wishart conjugate prior for the concentration matrix as the distribution $g$, the same factorization can be used to substantiate the claim that $\omega \neq 0$ if and only if:
\begin{align}
\tilde{\pi}_{ij}=\frac{\tilde{\rho}_{ij}}{\EE_g(\rho_{ij}|\textbf{X})}>0.5
\end{align}
where $\tilde{\rho}_{ij}$ is the posterior mean estimator of $\rho_{ij}$ and $\tilde{\pi}_{ij}$ is the amount of shrinkage enforced by the graphical lasso prior. In contrast, \cite{peterson2013} proposes that a rejection region approach could be employed using posterior credible intervals. Specifically, they suppose edge selection should include an edge between $X_i$ and $X_j$ if and only if the 95\% posterior credible interval for $\omega_{ij}$ does not include 0. The authors then use a thresholding approach in the application discussed, omitting edges between vertices if $|\omega_{ij}|\leq 0.1$.

\section{Bayesian inference given \emph{G}-Wishart priors}
An alternative Bayesian treatment focuses on  \cite{dawid1993,roverato2002,atay2005,dobra2011a,dobra2011b,wang2012,cheng2012} the estimation of GGMs via conjugate inference for $\boldsymbol{\Omega}$ given that $\boldsymbol{\Omega}$ is faithful to a fixed graph $G$, as opposed to estimation with shrinkage.

Early work \cite{dawid1993} focused on the case of decomposable GGMs. A decomposition of a graph $G$ is a pair subsets $(R,S)$, $R\subseteq V$, $S\subseteq V$ such that $V=R \cup S$, the graph defined by $R \cap S$ is complete, and $R \cap S$ is the separator of $R$ and $S$--that is any path between $R$ and $S$ must go through $R \cap S$. A graph $G$ is a decomposable graph if it complete or there exists a proper decomposition $(R,S)$ of $G$. Equivalently, a graph is decomposable if each prime component of the graph is complete \cite{roverato2002}. \cite{dawid1993} first describe the Hyper Inverse Wishart distribution for cliques $C\in \mathcal{C}$ where $\{C_1,C_2, \hdots,C_k \}$ is a perfectly ordered set of cliques in the decomposable graph $G$ and $S$ [LOH]

\begin{align}
f_G(\boldsymbol{\Sigma}^{\mathcal{V}}|\delta,\textbf{D}^{\mathcal{V}})= \frac{\prod_{j=1}^k f_{C_j}(\boldsymbol{\Sigma}_{C_j C_j}|\delta,\textbf{D}_{C_j C_j})}{\prod_{j=2}^k f_{S_j}(\boldsymbol{\Sigma}_{S_j S_j}|\delta,\textbf{D}_{S_j S_j})}
\end{align}

\cite{roverato2002} advanced the theory substantially by proposing methodology for inference given arbitrary graphs (including non-decomposable graphs). \cite{roverato2002} shows that if $G=(V,E)$ is an arbitrary graph then $\boldsymbol{\Sigma} \sim HIW_G$

This $G$-Wishart distribution has the following form: 
\begin{align}
p(\boldsymbol{\Omega}|G)=I_G(b,D)^{-1}|\boldsymbol{\Omega}|^{(b-2)/2} \exp \left[-\frac{1}{2} \tr(D \boldsymbol{\Omega}) \right] 1_{ \{\boldsymbol{\Omega} \in M^+(G)\} }
\end{align}
where $I_G(b,D)$ is a normalization constant that ensures that $p(\boldsymbol{\Omega}|G)$ is a proper distribution:
\begin{align}
I_G(b,D)=\int |\boldsymbol{\Omega}|^{(b-2)/2} \exp \left[-\frac{1}{2} \tr(D \boldsymbol{\Omega}) \right] 1_{ \{\boldsymbol{\Omega} \in M^+(G)\} } d\boldsymbol{\Omega}
\end{align}


The resulting map of V-band extinction is shown in Fig.~\ref{fig:radio_av}. 

The figure is an example of how to use a short caption that will appear in the list of figures in the
front matter, and to include a label that is used in the reference in the text.  Take care not to
confuse \verb|\ref| with \verb|\cite|.  The former is for a label, the latter for a literature
citation.

\begin{figure}
\resizebox{\textwidth}{!}{\includegraphics*{radio_av.png}}
\caption[Rosette A(V) Map from Radio Ratio]{ A plot of 
V-band extinction (A(V)) across the Rosette Nebula, derived from the flux 
calibrated 1410 MHz / H($\alpha$) ratio map and reddening factors determined 
from Cardelli 1989 \cite{cardelli1989}. The map is shown on a square root scale 
from a minimum ratio value of zero to a maximum of 7. Note that the Effelsberg 1410 MHz 
data had a half power beam width of 9.24 arcminutes, so detail below that level is 
oversampling from the H$\alpha$ map. This map was used in the extinction correction 
of the [OIII]/[SII] map of the ionization parameter discussed later.\label{fig:radio_av} }
\end{figure}




\chapter{Higher-order inference in metabolomics}
\label{higher}
\begin{DoubleSpace*}
Higher-order inference in metabolomics corresponds to the testing of hypotheses that are more complex than single metabolite hypothesis tests. For example, an experimenter might want to determine if a specific biological process (e.g. cholesterol biosynthesis) differs between two or more sample phenotypes. The central question of higher order inference is how to formally test a hypothesis such as:
\begin{itemize}
	\item $H_0$: cholesterol biosynthesis does not differ between two (or more) phenotypes 
	\item $H_A$: cholesterol biosynthesis does differ between two (or more) phenotypes
\end{itemize} 
Or formulated as an equivalence test:
\begin{itemize}
	\item $H_0$: cholesterol biosynthesis is the same between two (or more) phenotypes 
	\item $H_A$: cholesterol biosynthesis is not the same between two (or more) phenotypes
\end{itemize}

Other forms of higher-order inference are not explicitly hypothesis driven. Many metabolomics experiments are motivated to address open-ended questions such as ``What are the metabolic consequences of a specific disease or disease state?'' or ``What is the impact on cellular metabolism of the up or down-regulation of a specific gene''. An example of the first type can be found in Trainor et al. \cite{trainor2017}, who sought to determine the metabolic consequence of thrombotic MI using blood plasma across a disease state transition. An example of the second type can be found in Carlisle et al. \cite{carlisle2016}. Previous methods to make such higher order inferences generally fit one of the following categories: (1) statistical tests for pathway enrichment, (2) metabolite set enrichment analyses, or (3) correlation based analyses. 

\section{Statistical tests for pathway enrichment}
Pathway enrichment analyses seek to determine if specific pathways are represented more than expected by chance alone in a list of differentially abundant metabolites (or other biomolecules such as mRNA transcripts). These analyses take the results of univariate statistical tests that first identify sets of metabolite features for which significant evidence of differences between experimental conditions or phenotypes are observed. After identifying interesting metabolite features, these sets can be tested for enrichment of specific metabolic pathways or biological processes greater than that expected by chance \cite{goeman2007,xia2010}. 

\section{Exact tests of pathway enrichment}
\begin{table}[H]
		\caption[Example $2 \times 2$ table for two random variables $X$ and $Y$]{\DoubleSpacing Example $2 \times 2$ table for two random variables $X$ and $Y$. \label{tab:twoByTwo} }
	\centering
	\begin{tabular}{l  | c  c}
		& & \\
		 &	$X$ & \\
		\hline
		$Y$ & $n_{x_1 y_1}$ & $n_{x_2 y_1}$ \\
		& $n_{x_1 y_2}$ & $n_{x_2 y_2}$ \\
	\end{tabular}
\end{table}
To discuss the use of Fisher's exact for determining pathway over-representation we first introduce the computation of the probabilities of $2 \times 2$ contingency tables. The probability of a specific $2\times 2$ contingency table (as shown in Table~\ref{tab:twoByTwo})can be computed using the hypergeometric distribution as follows \cite{agresti2013}:
\begin{align}
	p = \frac{{n_{x_1 y_1}+n_{x_2 y_1}\choose n_{x_1 y_1}} {n_{x_1 y_2}+n_{x_2 y_2}\choose n_{x_1 y_2}} }{{n\choose n_{x_1 y_1}+n_{x_1 y_2}}},
\end{align}
where $n=n_{x_1 y_1}+  n_{x_2 y_1} +n_{x_1 y_2}+n_{x_2 y_2}$. Fisher's exact test is a test of the independence of rows and columns. In other words the test evaluates whether the random variables $X$ and $Y$ are independent. In order to determine the statistical significance of the test, the probability of the observed table can be compared to the probability of tables with the same marginal totals. In the case of evaluating whether a specific metabolic pathway is over-represented within a list of differentially abundant metabolites, we can represent pathway membership as the first random variable and differential abundances indicator as the second random variable. 

\begin{table}[H]
	\caption[Example Fisher's exact test for determining if a pathway is over-represented in a list of significant metabolites]{\DoubleSpacing Example Fisher's exact test for determining if a pathway is over-represented in a list of significant metabolites. \label{tab:pathExact} }
	\begin{center}
		\begin{tabular}{ l c c c }
			\hline
			& Differentially abundant (DA) & Not DA & Total \\
			\hline
			Pathway A & 10 & 20 & 30 \\
			Not in Pathway A & 40 & 350 &  390 \\
			Total & 50 & 370 & 420\\
			\hline
		\end{tabular}
	\end{center}
\end{table}
A hypothetical exact test for determining if a pathway is over-represented in a list of significant metabolites is presented in Table~\ref{tab:pathExact}. In this example, it is assumed that a total of 420 metabolites were observed. Of these metabolites, 30 were in Pathway A, while 390 were not in Pathway A. Of the metabolites in Pathway A, 10 were observed to be differentially abundant, while of the metabolites not in Pathway A, 40 were observed to be differentially abundant. A two-sided Fisher's test given the observed contingency table yields a p-value of 0.001, indicating that there is evidence of an association between the differential abundance indicator and the pathway membership indicator. 

\begin{figure}[H]
	\resizebox{\textwidth}{!}{\includegraphics*{../Proposal/vennHell2_2}}
	\caption[Graphical representation of the source of bias in discrete pathway enrichment analyses]{\DoubleSpacing Graphical representation of the source of bias in discrete pathway enrichment analyses. A single metabolic pathway is shown as a bordered rectangle. The blue circle represents cases in which a metabolite that is in a metabolic pathway is annotated as being a member of that pathway in pathway databases. The red circle represents cases in which a metabolite that is a member of a metabolic pathway is quantifiable given the sample medium utilized in the analysis. For example, a metabolite localized to the mitochondria may not be detectable in plasma samples. The orange circle represents cases in which a metabolite is detectable from a sample given the analytical technology and platform utilized. For example, a metabolite may be not detectable given the sensitivity of NMR. \label{fig:vennHell} }
\end{figure}

\section{Metabolite set enrichment analysis}
[To do: Add description of MSEA Here]

\section{Barupal Chem}
A promising alternative to pathway analyses discussed in \cite{barupal2017} is to use structural similarity and chemical ontology as a priori knowledge to generate study-specific metabolite sets for contextualizing empirical results. 

[To do: add more about Barupal paper]

\section{The need for interactomes in metabolomics}
Untargeted profiling of the metabolome of an organism provides a view into the small molecule determinants of phenotype. While the genome of an organism may be conceptualized as a blueprint for the composition and organization of an organism that is largely immutable (barring epigenetic modifications, DNA damage, and genetic mutations) \cite{gao2015,keating2015,martincorena2015}, the metabolome of an organism is dynamic and variable \cite{dallmann2012,krycer2017}. Sources of variation within the metabolome of a single organism include tissue-, cell-, and organelle-specific localization of metabolic processes \cite{shlomi2008,voet2013}; environmental exposures \cite{southam2014}; and host-microbe interactions and metabolite exchange \cite{moriya2017}. While the generation of reference human genomes has facilitated the interrogation of gene-phenome associations (including human disease associations), the intrinsic variability and dynamic nature of the metabolome of an organism likely precludes the generation of such a reference model. While a single reference model of the metabolome of an organism may not be sensible, significant efforts such as the HUSERMET project \cite{dunn2014} have been undertaken to quantify the repertoire of metabolites in specific biofluids for examining metabolite-metabolite and metabolite-phenotype associations. In order to make systems-level comparisons of the differences in the metabolome across phenotypes, models of the conditional relationships between metabolites are necessary. By the determination of sample media and analytical platform-specific probabilistic interaction models, henceforth called ``interactomes'', systems-level comparisons of phenotypes that can be made. A specific use case for such an interactome model is the generation of a plasma interactome for stable heart disease.

\section{A specific biomedical science use case}
Heart disease is the most prevalent cause of death globally \cite{benjamin2017}. As a disease, heart disease does not represent a uniform condition, but rather a collection of diseases of varying etiologies \cite{kasper2015}. Of particular interest in the study of coronary artery disease (CAD) is the elucidation of the precipitants of acute disease events such as myocardial infarction \cite{arbab2015} or unstable angina, metabolic pathways associated with disease phenotypes \cite{fan2016}, and determining the metabolic consequences of acute events \cite{trainor2017}. To date, an interactome describing the conditional relationships between blood plasma metabolites in humans with heart disease does not exist. If such a reference model were determined, it would facilitate making systems-level inferences regarding metabolic perturbations that accompany acute disease events such as unstable angina or acute myocardial infarction (MI).

\section{The challenge of inferring high-dimensional interactomes}
A significant challenge in evaluating the relationships between metabolites in an untargeted metabolomics experiment is that the dimension of metabolites may be greater than the number of samples. Even given a relatively high ratio of samples to metabolites detected, in the evaluation of pairwise conditional relationships between metabolites, the number of parameters to be estimated can be prohibitive. For example, if $p=500$ metabolites are detected, an evaluation of all pairwise conditional relationships would require the simultaneous estimation of 124,750 parameters. The use of regularization is a well-established approach for guaranteeing the existence of Gaussian Graphical Model parameters, amenable to the case that the sample size $n$ is less than $p$ \cite{banerjee2008,fan2009,friedman2007,meinshausen2006, yuan2007}.

Penalized estimation of GGM parameters provides a natural mechanism for integrating a priori knowledge regarding the molecular structure of metabolites with experimental metabolomics data. The integration of empirical data and scientific knowledge regarding metabolism is common in metabolomics studies. The current work is of a similar paradigm and predicated on the assumption that the individual biochemical reactions that result in statistical dependence between metabolic intermediates also generate statistical dependence in structural similarity between the same intermediates. In our application, structural similarity is determined by an approach that considers overlap in shared local structure between metabolites. Rather than considering fixed sets of metabolites such as pathways, sets, or modules and subsequently quantifying enrichment of these sets in empirical results, we consider a priori knowledge of the relationships between metabolites as probabilistic statements about the relatedness of compounds. Thus, the a priori scientific knowledge is used to generate prior probability distributions that influence GGM model selection, so that posterior inference probabilistically combines empirical data and prior scientific knowledge to yield an updated model of the probabilistic interactions between metabolites. In the present work, we introduce a methodology for using molecular structure similarity to generate prior distributions that control the degree of penalization in parameter estimation for learning a GGM metabolite interactome from metabolomics data. 
\end{DoubleSpace*}

{\newgeometry{top=1in,left=1.5in,bottom=1in,footskip=.5in} 
\chapter{Inferring metabolite interactomes via Bayesian graphical model selection utilizing molecular structure informative priors}
\section{Desired inference}

\section{Intractability}
}

{\newgeometry{top=1in,left=1.5in,bottom=1in,footskip=.5in} 
\chapter{A plasma interactome for stable heart disease}
\section{Cohort}

In order to determine changes in the plasma metabolome associated with myocardial infarction (MI) characterized by thrombotic etiology versus non-thrombotic etiology, DeFilippis and colleagues assembled a human cohort as previously described (DeFilippis et al., 2016; DeFilippis et al., 2017; Trainor et al., 2017). Briefly, 80 human subjects presenting with suspected acute MI or stable coronary artery disease (CAD) were enrolled. Utilizing a stringent criteria based on clinical presentation, angiographic evidence, and histological evidence, MI subjects were adjudicated as thrombotic MI or non-thrombotic MI. Blood samples were collected at the time of acute presentation (presentation to the coronary artery catheterization lab prior to procedures) and at a follow-up evaluation approximately three months later. To estimate the structure of a stable heart disease plasma interactome, we used the follow-up evaluations from all available MI subjects as well as the evaluations from stable CAD subjects. The analytical sample thus consisted of 47 whole blood samples from human subjects with definitive heart disease who were not experiencing an acute event at the time of sampling. 

\section{Plasma metabolomics}
Details of the metabolite quantification have been described previously (Trainor et al., 2017), but a brief overview is provided as follows. Plasma samples were prepared from whole blood and a recovery standard was added. Vigorous shaking was applied utilizing a GenoGrinder 2000 (Glen Mills, Metuchen, NJ) and methanol was added and to precipitate proteins. The extract containing small molecules was divided into five aliquots, four of which were analyzed using different platforms while the remaining aliquot was reserved. Two aliquots were analyzed by ultra-performance liquid chromatography-tandem mass spectrometry (UPLC-MS/MS) with negative and positive ion mode electrospray ionization (ESI). A third aliquot was also analyzed by UPLC-MS/MS with negative ion mode ESI and a method optimized for polar metabolite detection. The fourth aliquot was analyzed by gas chromatography-mass spectrometry (GC-MS). 1,032 chemical features  were detected utilizing the multiple platforms in the analysis of the plasma samples. Of these, 590 compounds were identified by matching to authentic standards based on retention index, mass to charge ratio, and MS2 data; 73 were identified based on experimental data matched to curated databases; and 369 could not be confidently identified. As the original data dependent acquisition was conducted utilizing both acute event samples and stable heart disease samples, metabolites not detected in the stable heart disease samples were removed. Metabolites missing from greater than 70\% of the samples or without compound identification were also removed, resulting in a final dataset with 522 metabolites across 47 samples. Minimum values were then imputed for the remaining metabolite relative abundances with missing data. As many of the metabolites exhibited approximately log-normal relative abundance distributions, metabolite abundances were log-transformed. Finally, the data was mean centered so that each metabolite’s relative abundance distribution was centered about zero. 

\section{Generation of chemical structure informed priors}
A heatmap representation of the structural similarity between metabolites is shown in Figure 4. This heatmap was constructed using agglomerative hierarchical clustering using Ward’s method and squared distances (with distance computed as $d_{ij}=1-s_{ij}$, where $s_{ij}$ is the structural similarity between compounds). For illustrative purposes, the cluster containing cholate was retrieved from the root dendrogram by extracting the branch with height X, as the structural-adaptive BGL subnetwork generated by cholate is considered later. Considering clusters generated by branches with low merge heights (high structural similarity), cholate was a member of a cluster with other closely related compounds such as deoxycholate, 3b-hydroxy-5-cholenoic acid, and glycocholate. Considering the more inclusive cluster generated by the branch at join height x, other members included many intermediates in progestagen, androgen, glucocorticoid, and mineralocorticoid steroid metabolic pathways. These steroid hormone metabolites were all members of a cluster with similar within-cluster distances. Finally, at the same branch height that joined steroid hormone and cholate metabolites, a branch consisting of tocopherols cluster and squalene cluster was also joined.

\begin{figure}[ht]
	\resizebox{\textwidth}{!}{\includegraphics*{../Aim2/Plots/StructHeatmaps}}
	\caption[Add caption]{Add caption \label{fig:heatmap} }
\end{figure}

\section{Gibbs sampling}
To approximate the posterior distribution of $p(\boldsymbol{\Omega}|\textbf{X})$, a Markov Chain was generated of length 1,000 with a 250 iteration burn-in period. From each sample from the posterior distribution $p(\boldsymbol{\Omega}|\textbf{X})$, a partial correlation coefficient matrix was computed, yielding a simulated posterior distribution for the matrix of partial correlation coefficients.
}

\chapter{A Bayesian diagnostic model for differentiating MI type}
\begin{DoubleSpace*}
\section{Acute Myocardial Infarction}
Acute Myocardial Infarction (AMI), which is an acute manifestation of coronary heart disease, is defined by
myocardial ischemia (exposure of cardiac myocytes to oxygen deprivation) and necrosis (a form of cell death) \cite{trainor2018}. AMI may occur following atherosclerotic plaque disruption or other conditions which cause demand ischemia \cite{thygesen2012,arbab2012}. Irrespective of underlying cause, ischemia and necrosis are the common pathological characteristics of all AMI. Thrombotic MI (MI results from spontaneous atherosclerotic plaque disruption with the formation of an occluding coronary thrombus) versus non-thrombotic MI represents an important etiological distinction \cite{defilippis2017}, as both types necessitate different treatment approaches \cite{thygesen2012}. A diagnosis of AMI can be substantiated by blood-based diagnostic tests that measure isoforms of the protein troponin which is released into the circulation following necrosis \cite{newby2012}. To date a blood-based diagnostic test capable of discriminating between thrombotic and non-thrombotic MI has not been developed, although it has been previously shown that a metabolic signature may differentiate between the types \cite{defilippis2017,trainor2018}. In the present work, we set out to develop a Bayesian model for differentiating thrombotic MI from both non-thrombotic MI and stable coronary artery disease (CAD) using metabolites detected in blood plasma. We regard plasma as a promising media for developing a non-invasive test as plasma contains hormones, enzymes, lipoproteins, and other metabolic intermediates found in circulation. As metabolite concentrations are a product of genetic factors, environmental exposures, and the interaction between the two, sampling metabolites may provide a more robust characterization of the state of an organism than other approaches such as genomics. 

\section{Clinical cohort and samples}
Towards the effort of developing a Bayesian diagnostic model, we utilized previously collected samples from a patient cohort that was recruited specifically for contrasting thrombotic MI from multiple control phenotypes \cite{defilippis2015,defilippis2017}. This cohort was comprised of three phenotypic groups of human subjects: thrombotic MI, non-thrombotic MI, and stable CAD. In reference to the  thrombotic MI group, both control groups (non-thrombotic MI and stable CAD) served as procedural controls as all groups underwent a cardiac catheterization procedure. The stable CAD group provides a stable disease control as both thrombotic MI subjects and stable CAD subjects have underlying coronary artery disease. Non-thrombotic MI subjects presented with myocardial necrosis and thus the non-thrombotic MI group serves as an acute disease event control. 

Whole blood was collected shortly before cardiac catheterization from all study subjects. Details of the sample handling, sample processing, separation by liquid or gas chromatography, and mass spectrometry analysis for quantifying metabolites are provided in Chapter~\ref{hdInteractome} Section~\ref{plasma}. 

\section{Feature selection}
A significant analytical challenge in developing a blood-based diagnostic test for differentiating MI types is to determine a small set of metabolites that should be included in the statistical model from a limited number of training samples. Specifically, 1,032 chemical features (identified compounds or unknown compounds) were detected from 11 thrombotic MI, 12 non-thrombotic MI, and 15 stable CAD subjects. To ensure that the statistical model is estimable from a small sample size, and as building a targeted MRM assay or multiplexed ELISA assay can only accommodate a small number of compounds, feature selection is a critical task. While other dimension reduction techniques such as latent variable approaches (e.g. Partial Least Squares models) that create new variables which are linear combinations of metabolites would be amenable for reducing the number of coefficients to be estimated in a classification model, these approaches would not reduce the number of metabolites needing to be quantified by future targeted assays. Consequently, we prioritized reducing the number of metabolites considered. To determine a statistical classifier with five metabolites, $9.7 \times 10^{12}$ combinations are possible. In order to search the space of possible models we employed a feature selection technique that utilizes an evolutionary algorithm and seeks a consensus solution over bootstrapped datasets as described in Trainor et al. \cite{trainor2018}. In this work, small sets of metabolites were included in a multinomial logit classifier. Each model represented an individual in a population. Genetic fitness was determined as the likelihood of each individual model. These populations of models were allowed to evolve given evolutionarily inspired processes such as birth, recombination, and death. Populations were grown over bootstrapped datasets to increase diversity and reduce the correlation between models in the populations. Finally, the frequency that an individual metabolite was included in models within the final population after epochs of evolution was determined yielding a variable importance score for each metabolite. A correlation plot illustrating the metabolites with greatest variable importance score using the described technique is shown in Figure~\ref{fig:include}.

\end{DoubleSpace*}

\begin{figure}[H]
	\resizebox{1.1\textwidth}{!}{\includegraphics*{../../AthroMetab/WoAC/corrplot333}}
	\caption[Feature selection]{Metabolites with the highest variable importance score given the feature selection method we employed in a previous work \cite{trainor2018}. The Pearson correlation coefficients between metabolite transformed and scaled abundances are shown. \label{fig:include}}
\end{figure}

\begin{DoubleSpace*}
\section{Model}
A multinomial logit model was assumed for determining the  probability a sample from a clinical subject was member of the thrombotic MI, non-thrombotic MI, or a stable CAD study group. The multinomial logit model is a generalized linear model and has the following link function and model form: 
\begin{align}
\eta_{ij} = \log \frac{\pi_{ij}}{\pi_{iJ}} = \alpha_j + \textbf{x}_i^T \boldsymbol{\beta}_j,
\end{align} 
where $i$ indexes individual samples, $j$ is the index of study groups, $\boldsymbol{\beta}_j$ is a vector of regression coefficients (with separate coefficients for each group), $\alpha_j$ is a group specific intercept term, and $J$ represents the reference group. Given a specific value for the link function, the probability a sample belongs to specific study group can be computed as:
\begin{align}
\hat{\pi}_{ij} = \frac{\exp \hat{\eta}_{ij}}{\sum_{k=1}^{J}\exp \hat{\eta}_{ik}}.
\end{align}

The model can be stated as a Bayesian model with the following priors and deterministic component:
\begin{align}
\begin{split}
	\alpha_j \sim N(0,4) \\
	\beta_j \sim N(0,1) \\
	\log \frac{\pi_{ij}}{\pi_{iJ}} = \alpha_j + \textbf{x}_i^T \boldsymbol{\beta}_j \\
	Y \sim Multinom(\boldsymbol{\pi})
\end{split}
\end{align}

\section{MCMC sampling of the posterior distribution}
A MCMC sampler known as the ``No-U-Turn'' sampler was utilized to simulate the posterior distribution of model parameters \cite{hoffman2014}. This algorithm is an extension of the Hamiltonian Monte Carlo algorithm. The Hamiltonian Monte Carlo algorithm is designed to model a target probability distribution as a Hamiltonian system \cite{betancourt2017}. In such a system, a parameter vector $\boldsymbol{\theta}$ is viewed as a particle in a $D$-dimensional space \cite{neal2011} by defining a vector field that is aligned with the typical set (the region of a parameter space with both significant volume and desnity) \cite{betancourt2017}. Defining such a vector field is a complex task. While a vector field could be defined using the gradient of the target probability distribution, this vector field would pull a particle towards the mode of the distribution. By adding a momentum term, the vector field can be defined so as to restrict a particle to maintaining the Hamiltonian at a fixed value. The Hamiltonian Monte Carlo algorithm utilizes Metropolis-Hastings proposals for updating both the momentum and position variables \cite{neal2011,betancourt2017}. A critical aspect of Hamiltonian Monte Carlo sampling is determining the optimal integration time for a particle to travel along a Hamiltonian path. One approach to dynamically determining this parameter is the ``No-U-Turn'' termination criteria \cite{hoffman2014,betancourt2017}. Conceptually, this criteria ensures that expansion of a trajectory continues to visit previously unexplored neighborhoods while terminating the trajectory when it returns to previously explored neighborhoods.

The ``No-U-Turn'' Hamiltonian MCMC sampler was implemented by others in \emph{Stan}, a probabilistic programming language \cite{carpenter2017}. Stan is written in C++ and provides the ``No-U-Turn'' sampler, a Hamiltonian Monte Carlo Sampler, a variational inference algorithm, and multiple optimizers. Stan provides grammar and syntax for succinctly specifying Bayesian models; can design a sampler for the posterior distribution of model parameters; and using a C++ compiler, Stan compiles the sampler to byte code. Additionally, Stan provides data structures for storing MCMC chains and model output. An R to Stan interface, \emph{rstan} \cite{stan2018}, has been developed allowing an end user to pass datasets between Stan and to return MCMC chains and model results. 

\section{Results}
In order to simulate the posterior distribution of model parameters, four MCMC chains were generated utilizing the No-U-Turn algorithm. An example MCMC chain is illustrated in Figure~\ref{fig:brm1Coef}. In this figure 2,000 of the 10,000 iterations of the regression coefficient parameter for 3-hydroxypyridine sulfate from one of the MCMC chains (chain \#1) is shown. 

\end{DoubleSpace*}

\newpage
\KOMAoptions{paper=landscape}
\recalctypearea
\begin{figure}[H]
	\includegraphics[scale=1.1]{../Aim3/MCMCEx.png}
	\caption[Add caption]{Add caption \label{fig:brm1Coef} }
\end{figure}
\newpage
\begin{figure}[H]
	\includegraphics[scale=1.15]{../Aim3/brm1Coef}
	\caption[Add caption]{Add caption \label{fig:brm1Coef} }
\end{figure}
\newpage
\begin{figure}[H]
	\includegraphics[scale=1.15]{../Aim3/brm2Coef}
	\caption[Add caption]{Add caption \label{fig:brm2Coef} }
\end{figure}
\newpage
\KOMAoptions{paper=portrait,pagesize}
\recalctypearea

\begin{figure}[H]
	\resizebox{1.1\textwidth}{!}{\includegraphics*{../Aim3/coefPost}}
	\caption[Add caption]{Add caption \label{fig:coefPost} }
\end{figure}

\begin{figure}[H]
	\resizebox{1.1\textwidth}{!}{\includegraphics*{../Aim3/coefPost2}}
	\caption[Add caption]{Add caption \label{fig:coefPost2} }
\end{figure}

Blah blah blah blah blah patients

\begin{figure}[H]
	\resizebox{1.15\textwidth}{!}{\includegraphics*{../Aim3/ptid2010MCMC}}
	\caption[Add caption]{Add caption \label{fig:ptid2010MCMC} }
\end{figure}

\begin{figure}[H]
	\resizebox{1.15\textwidth}{!}{\includegraphics*{../Aim3/ptid2010Hist}}
	\caption[Add caption]{Add caption \label{fig:ptid2010Hist} }
\end{figure}
 
Leave-one-out cross validation estimates of study group membership for each clinical sample. Group membership was determined by maximum a posteriori estimates, and the class with maximum estimated probability of membership is reported.
 
 \begin{table}[H]
 	\caption{text}
 	\label{tab:modelRes}
 	\centering
 	\begin{tabular}{l|ccc}
 		& Predicted & & \\
 		group  &     sCAD & Thrombotic MI & Non-Thrombotic MI \\
 		\hline
 		sCAD   &     13    &     0    &     2\\
 		Thrombotic MI  &  0    &    10    &     1\\
 		Non-Thrombotic MI &   1    &     1   &     10
 	\end{tabular}
 \end{table}

\chapter{Bayesian approaches for compound identification given LC-MS data}
\section{The challenge of identifying compounds from mass spectrometry data}
The identification of compounds from chromatography-coupled mass spectrometry experiments remains a great challenge of untargeted metabolomics \cite{domingo2017,uppal2017,uppal2016,dunn2012}. Many pieces of information may be available to assist in determining the chemical identity of features detected in such experiments including \cite{dunn2012}: the mass-to-charge ratio (m/z) of the ions that have been observed, the fragmentation pattern of either parent or fragment ions, isotopic distribution (e.g. the relative abundance of isotopes can be evaluated by comparing the ratio of 13C to 12C), and the observed elution times from separation such as liquid or gas chromatography. Which of these classes of information should be used to identify compounds is largely analytical platform dependent. A further complication of assigning compound identity to the features observed in an experiment is adduct formation. In GC-MS analyses the derivatization process is not a uniform process \cite{halket2003}. When electrospray ionization (ESI) is utilized to ionize injected molecules, multiple types of adducted ions may be formed (e.g. sodium adducts) in pronated, non-pronated, or de-pronated form dependent on ESI mode \cite{dunn2012}.

A system for representing the level of confidence in a compound annotation has been proposed previously as part of the Metabolomics Standards Initiative (MSI) \cite{sumner2007}. The first level (Level 1) represents the most confident type of identification and requires matching two ``orthogonal'' properties of an observed feature with the properties observed from the analysis of an authentic standard under identical analytical conditions. An example of two orthogonal properties for an LC-MS analysis would be observed m/z and retention time or retention index. In an LC-MS/MS experiment observed m/z and MS/MS fragmentation pattern compared to an authentic standard analyzed under identical analytical condition would constitute a Level 1 identification.  A Level 2 identification also requires matching on ``orthogonal'' properties, but in comparison to previously archived data (e.g. databases such as the NIST libraries) as opposed to authentic standards analyzed under identical conditions). A Level 3 identification is more general and requires association of the properties of an observed feature to those of classes of biochemicals as opposed to specific compounds. This framework asserts in a general way that metabolite annotation is a probabilistic process, in which the likelihood of a match from mass feature to compound label depends on both \emph{a priori} knowledge and empirical data. Consequently, a Bayesian approach to compound identification is sensible. 

\section{Prior approaches for ID}


\chapter{Conclusions, discussion, and future directions} 
\section{Conclusions}

After all that work, what have you found out?


The thesis created by this \LaTeX\/ class  should meet the requirements of the
Graduate School.  However, before you print it,  ask them to review it for style
with enough lead time to make changes as needed. The class file you are using
was written in January 2018 to match a style manual dating from 2015.  The
output will have some characteristics of the \LaTeX\/ system that may seem odd
to someone used to processing with Microsoft Word.  Making global changes in the
class file is not trivial but often a solution can be found by searching
Stackexchange or other web resources. If you encounter problems in producing the
desired format you cannot solve,  please contact Prof. Kielkopf in the
Department of Physics and Astronomy.  If you find something you think should be
shared with other students  writing dissertations, also please let me know so
that it can be added.










\backmatter
{
\newgeometry{top=1in,bottom=1in,footskip=.5in}
\begin{DoubleSpace*}

\bibliographystyle{apalike}
%\bibliographystyle{plain}
%\bibliographystyle{osajnl}
\bibliography{dissertation}

\end{DoubleSpace*}
% Appendices and the Vita are treated as chapters in the back matter

\chapter{Appendix A: Acronyms Utilized}
\noindent
AARL: Atherosclerosis / Atherothrombosis Research Laboratory \\
APCI: Atmospheric Pressure Chemical Ionization \\
ATP: Adenosine triphosphate \\
AUC: Area Under the receiving operating characteristic Curve \\
BCAA: Branch chain amino acid \\
BGL: Bayesian Graphical Lasso \\
BLAS: Basic Linear Algebra Subprograms \\
CAD: Coronary Artery Disease \\
cAMP: Cyclic adenosine monophosphate \\
CASS: Chemically Aware Substructure Search algorithm \\
CE: Capillary Electrophoresis \\
ChemRICH: Chemical Similarity Enrichment Analysis \\
CI: Confidence Interval / Credible Interval  \\
CoA: Coenzyme A \\
DA: Differentially Abundant  \\
DI: Direct Injection \\
EC2: Elastic Compute Cloud \\
EI: Electron Ionization  \\
ELISA: Enzyme-Linked Immunosorbent Assay \\
eQTL: Expression Quantitative Trait Loci \\
ESI: Electrospray Ionization  \\
EST: Expressed Sequence Tag \\
FADH: Flavin adenine dinucleotide \\
FT-ICR: Fourier Transform Ion Cyclon Resonance  \\
GC: Gas Chromatography  \\
GGM: Gaussian Graphical Model \\
GSEA: Gene Set Enrichment Analysis \\
GTP: Guanosine triphosphate \\
HILIC: Hydrophilic Interaction Liquid Chromatography  \\
HMDB: Human Metabolome Database \\
HMG-CoA: 3-hydroxy-3-methylglutaryl-CoA \\
KEGG: Kyoto Encyclopedia of Genes and Genomes \\
LAPACK: Linear Algebra PACKage \\
LASSO: Least Absolute Shrinkage and Selection Operator \\
LC: Liquid Chromatography \\
LOOCV: Leave One Out - Cross Validation \\
MALDI: Matrix Assisted Laser Desorption / Ionization  \\
MCMC: Markov Chain Monte Carlo \\
MeSH: Medical Subject Headings \\
MI: Myocardial Infarction \\
MLE: Maximum Likelihood Estimation
MRM: Multiple Reaction Monitoring \\
mRNA: Messenger ribonucleic acid \\
MS: Mass Spectrometry \\
MSEA: Metabolite Set Enrichment Analysis \\
NAD: Nicotinamide adenine dinucleotide \\
NMR: Nuclear Magnetic Resonance  \\
PKA: Protein kinase A \\
RNA: Ribonucleic acid  \\
RP: Reversed Phase \\
SCAD: Smoothly Clipped Absolute Deviation \\
SMPDB: Small Molecule Pathway Database \\
SRM: Selected Reaction Monitoring  \\
TCA: Tricarboxylic acid \\
TiGER: Tissue-specific Gene Expression and Regulation database \\
TOF: Time of Flight \\
UPLC: Ultra Performance Liquid Chromatography  \\



\chapter{Appendix B: Software developed for the Bayesian Graphical Lasso}
This is an example of how to include code listing within dissertation. 

\lstset{language=HTML, linewidth=6in, breaklines=true, 
  breakatwhitespace=true, breakindent=5pt, postbreak=\space}
\begin{lstlisting}{}

<!DOCTYPE html>
<html lang="en">
  <head>
    <title>Rosette Nebula Geometry</title>
    <meta charset="utf-8">
    <meta name="viewport" content="width=device-width, user-scalable=no, minimum-scale=1.0, maximum-scale=1.0">
    <style>
      body {
        color: #fff;
        font-family: Monospace;
        font-size: 13px;
        text-align: center;
        font-weight: bold;

        background-color: #000;
        margin: 0px;
        overflow: hidden;
      }

      #info {
        position: absolute;
        padding: 10px;
        width: 100%;
        text-align: center;
        color: #fff;
      }

      a { color: blue; }

    </style> 
    
  </head>
  <body>
    <div id="info">
            <a href="http://www.astro.louisville.edu" target="_blank">University of Louisville Physics & Astronomy</a>
            <br/>
            Jeremy Huber - Rosette Nebula
    </div>
    
    <!-- Threedotjs -->
    
    <script src="js/three.min.js"></script>

    <!-- Shaders adapted from http://stemkoski.github.io/Three.js/Shader-Glow.html -->

     <script id="starvertexShader" type="x-shader/x-vertex">
      varying float intensity;
      void main() 
      {
                                                              
        intensity = 1.;
        // Pass the 2D position for this vertex to the shader
        gl_Position = projectionMatrix * modelViewMatrix * vec4( position, 1.0 );
      }
    </script>

    <script id="starfragmentShader" type="x-shader/x-vertex"> 
      uniform vec3 starglowColor;
      varying float intensity;
      void main() 
      {
        vec3 glow = starglowColor * intensity;
        gl_FragColor = vec4( glow, 1.0 );
      }
    </script>
    


     <script id="clustervertexShader" type="x-shader/x-vertex">
      void main() 
      {
        // Pass the 2D position for this vertex to the shader
        gl_Position = projectionMatrix * modelViewMatrix * vec4( position, 1.0 );
      }
    </script>

    <script id="clusterfragmentShader" type="x-shader/x-vertex"> 
      uniform vec3 clusterglowColor;
      void main() 
      {
        vec3 glow = clusterglowColor * 1.0;
        gl_FragColor = vec4( glow, 1.0 );
      }
    </script>

  </body>
</html>

\end{lstlisting}


\chapter{CURRICULUM VITAE}
\begin{DoubleSpace*}
\begin{center}
Patrick Trainor 
\end{center}

\bigskip
{\parindent0pt

\section*{Objective}
My objective is to obtain a research position in computational metabolomics. My primary research interests are: (1) developing Bayesian methodology for integrating extra-experimental knowledge and data sources for integrated -omics analyses, (2) developing Bayesian methodologies for solving problems in analytical chemistry, and (3) developing computational tools for metabolomics data analyses.  

\section*{Education}
 PhD Bioinformatics  \\
University of Louisville, Louisville, KY, 2015-2018 (expected)  \\ \\
 MS Biostatistics  \\
University of Louisville, Louisville, KY, 2012-2014  \\ \\
MA Mathematics  \\
University of Louisville, Louisville, KY, 2012-2014 \\ \\
BS Mathematics  \\
Seattle University, Seattle, WA, 2012

\section*{Research Positions}
\emph{Graduate Research Assistant} \hfill 2012-2014, 2015-Current \\
Institute of Molecular Cardiology \& J.G. Brown Cancer Center (formerly), University of Louisville \\

\emph{Consultant \& Intern} \hfill 2012-2013 \\
Theravance Inc. (Theravance Biopharma) 

\section*{Teaching Positions}
\emph{Graduate Teaching Assistant} \hfill 2012-2014 \\
University of Louisville \\

\emph{Undergraduate Research Assistant} \hfill 2010-2012 \\
Seattle University

\section*{Other Professional Experience}
\emph{Clinical Analytics Consultant} \hfill 2014-2015 \\
Humana, Inc., Louisville, KY \\

\emph{Rifleman \& Infantry Squad Leader} \hfill 2007-2013 \\
4th Anti-Terrorism Battalion \\
United States Marine Corps (R) 

\section*{Publications}
*Denotes equally contributing authorship 

\textbf{Trainor, P.J.}, Yampolskiy, R.V.*, \& DeFilippis, A.P.* (2018). Wisdom of artificial crowds feature selection in untargeted metabolomics: An application to the development of a blood-based diagnostic test for thrombotic myocardial infarction. \emph{Journal of Biomedical Informatics, 81}.  doi: 10.1016/j.jbi.2018.03.007  \\

DeFilippis, A.P., \textbf{Trainor, P.J.} (2018). When given a lemon, make lemonade: Revising cardiovascular risk prediction scores. \emph{Annals of Internal Medicine,} online ahead of print. doi: 10.7326/M18-1175. \\

Carlisle, S.M., \textbf{Trainor, P.J.}, Doll, M.A., Stepp, M.W., Klinge, C.M., \& Hein, D.W. (2018). Knockout of human arylamine N-acetyltransferase 1 (NAT1) in MDA-MB-231 breast
cancer cells leads to increased reserve capacity, maximum mitochondrial capacity, and glycolytic reserve capacity. \emph{Molecular Carcinogenesis,} online ahead of print. doi: 10.1002/mc.22869\\

\textbf{Trainor, P.J.}, DeFilippis, A.P., \& Rai, S.N. (2017). Classifier performance for multiclass phenotype discrimination in metabolomics. \emph{Metabolites, 7}(2). doi: 10.3390/metabo7020030 \\

\textbf{Trainor, P.J.}, Hill, B.G., Carlisle, S.M., Rouchka, E.C., Bhatnagar, A., Rai, S.N., \& DeFilippis, A.P. (2017). Systems characterization of differential plasma metabolome perturbations following thrombotic and non-thrombotic myocardial infarction \emph{Journal of Proteomics, 160}. doi: 10.1016/j.jprot.2017.03.014. \\

DeFilippis, A.P., \textbf{Trainor, P.J.}, Hill, B.G., Amraotkar, A.R., Rai, S.N., Hirsch, G.A., Rouchka, E., \& Bhatnagar, A. (2017). Identification of a plasma metabolomic signature of thrombotic myocardial infarction that is distinct from non-thrombotic myocardial infarction and stable coronary artery disease. \emph{PLoS One, 12}(4). doi: 10.1371/journal.pone.0175591. \\

Gibb, A.A., Epstein, P.N., Uchida, S., McNally, L.A., Obal, D., Katragadda, K., \textbf{Trainor, P.J.},  Conklin, D.J., Brittian, K.R., Tseng, M.T., Wang J., Jones, S.P., Bhatnagar, A., \& Hill, B.G. (2017). Exercise-Induced Changes in Glucose Metabolism Promote Physiologic Cardiac Growth \emph{Circulation, 136}(21). doi: 10.1161/CIRCULATIONAHA.117.028274 \\

Khanal, S. \textbf{Trainor, P.J.}, Zahin, M., Ghim, S.J., Joh, J., Rai, S.N., Jenson, A.B., \& Shumway, B.S. (2017). Histologic variation in high grade oral epithelial dysplasia when associated with high-risk human papillomavirus.  \emph{Oral Surgery, Oral Medicine, Oral Pathology, and Oral Radiology, 123}(5). doi: 10.1016/j.oooo.2017.01.008. \\

Sultan, A., Zheng, Y., \textbf{Trainor, P.J.}, Yong, S., Amraotkar, A.R.  Hill, B.G., \& DeFilippis, A.P. (2017). Circulating prolidase activity in patients with myocardial infarction. \emph{Frontiers in Cardiovascular Medicine}. doi: 10.3389/fcvm.2017.00050.  \\

Amraotkar, A.R., Ghafghazi, S., \textbf{Trainor, P.J.}, Hargis C.W., Irfan A.B., Rai S.N., Bhatnagar A., \& DeFilippis A.P. (2017). Presence of multiple coronary angiographic characteristics for the diagnosis of acute coronary thrombus. \emph{Cardiology Journal, 24}(1). doi: 10.5603/CJ.a2017.0004.  \\

Khosravi, F., \textbf{Trainor, P.J.}, Lambert, C., Kloecker, G., Wickstrom, E. Rai, S.N., \& Panchapakesan, B. (2016). Static micro-array isolation, dynamic time series classification, capture and enumeration of spiked breast cancer cells in blood: the nanotube--CTC chip. \emph{Nanotechnology, 27}(44). doi: 10.1088/0957-4484/27/44/44LT03.  \\

Carlisle, S.M., \textbf{Trainor, P.J.}, Yin, X., Doll, M.A., Stepp, M.W., States, J.C., Zhang, X., \& Hein, D.W. (2016). Untargeted polar metabolomics of transformed MDA-MB-231 breast cancer cells expressing varying levels of human arylamine N-acetyltransferase 1. \emph{Metabolomics, 12}(7). doi: 10.1007/s11306-016-1056-z.  \\

Amraotkar, A.R., Song, D.D., \textbf{Trainor, P.J.}, Ismail, I., Kothari, K., Sing, A., Moore, J.B., Rai, S.N., Bhatnagar, A., \& DeFilippis, A.P. (2016). Platelet count and mean platelet volume at the time of and after acute myocardial infarction. \emph{Clinical and Applied Thrombosis/Haemostasis, 23}(8). doi: 10.1177/1076029616683804.  \\

DeFilippis, A.P., Chernyavskiy, I., Amraotkar, A.R., \textbf{Trainor, P.J.}, 
Kothari, S., Ismail, I., Hargis, C.W., Korley, F.K., Tsimikas, S., Rai, S.N., \& Bhatnagar, A. (2016). Circulating levels of plasminogen and oxidized phospholipids bound to plasminogen distinguish between atherothrombotic and non-atherothrombotic myocardial infarction. \emph{Journal of Thrombosis and Thrombolysis, 42}(1). doi: 10.1007/s11239-015-1292-5.  \\

Egan, J.M., Sloughter, J.M., Cardoso, T., \textbf{Trainor, P.J.}, Wu, K., Safford, H., \& Fournier, D. (2016). Multi-temporal ecological analysis of Jeffrey pine beetle outbreak dynamics within the Lake Tahoe Basin. \emph{Population Ecology, 58}(3). doi: 10.1007/s10144-016-0545-2.  \\

Rai, S.N.*, \textbf{Trainor, P.J.}*, Khosravi, F., \& Panchapakesan, B.  (2016). Classification of biosensor time series using dynamic time warping: applications in screening cancer cells with characteristic biomarkers. \emph{Open Access Medical Statistics, 6}. doi: 10.2147/OAMS.S104731.  \\

Khosravi, F., \textbf{Trainor, P.J.}, Rai, S.N., Kloecker, G., Wickstrom, E. \& Panchapakesan, B. (2016). Label-free capture of breast cancer cells spiked in buffy coats using carbon nanotube antibody micro-arrays. \emph{Nanotechnology, 27}(13). doi: 10.1088/0957-4484/27/13/13LT02.  \\

Khanal, S., Cole, ET., Joh, J., Ghim, S.J., Jenson, A.B., Rai, S.N., \textbf{Trainor, P.J.}, \& Shumway, B.S. (2015). Human papillomavirus detection in histologic samples of multifocal epithelial hyperplasia: a novel demographic presentation. \emph{Oral Surgery, Oral Medicine, Oral Pathology, and Oral Radiology, 120}(6). doi: 10.1016/j.oooo.2015.07.035. 

\section*{Published Software}
\textbf{Trainor, P.J.} \& Wang H. (2017). \emph{BayesianGLasso: Bayesian Graphical Lasso}. Available from https://CRAN.R-project.org/package=BayesianGLasso

\section*{Accepted Publications}
Khanal S., Shumway B.S., Zahin M., Redman R., \textbf{Trainor P.J.}, Rai S.N., Ghim S.J., Jenson A.B., \& Joh J. (2018). Human Papillomavirus DNA integration, but not viral methylation, is associated with malignant progression of head and neck cancers. \emph{Oncotarget}, in production.

\section*{Invited Talks}
\textbf{Trainor, P.J.} (2017). Metabolomics of thrombotic myocardial infarction: systems characterization of plasma metabolome perturbations and the development of a diagnostic classifier. \emph{Systems Biology Seminar Series}, University of Kentucky, Lexington, Kentucky.

\section*{Oral / Poster Presentations}
\textbf{Trainor, P.J.} (2018). Inferring metabolite interactomes via Bayesian graphical model selection utilizing molecular structure informative priors. \emph{14\textsuperscript{th} Annual Conference of the Metabolomics Society}, Seattle, Washington.  \\ 

\textbf{Trainor, P.J.} (2018). High Performance R: Parallel computing, seamless C++ integration, optimizing linear algebra routines, and deep learning in the R language. \emph{Kentucky Biomedical Research Infrastructure Network Seminar Series}, University of Louisville, Louisville, Kentucky.  \\ 

\textbf{Trainor, P.J.}, Carlisle, S.M., Hill, B.G., Rouchka, E.C., Bhatnagar, A., Rai, S.N., \& DeFilippis, A.P. (2017). Metabolomic analysis of thrombotic and non-thrombotic Myocardial Infarction: From systems biology to diagnostic classification. \emph{Symposium in Molecular Biology ``Metabolism: Disease Models and Model Organisms''}. Pennsylvania State University, State College, Pennsylvania.  \\ 

\textbf{Trainor, P.J.}, Carlisle, S.M., DeFilippis, A.P.*, \& Rai, S.N.* (2017). Molecular fingerprinting for inferring a Gaussian Graphical Model representation of a stable coronary artery disease plasma interactome using adaptive graphical Lasso penalization. \emph{UT-KBRIN Annual Bioinformatics Summit}, Burns, Tennessee.  \\ 

\textbf{Trainor, P.J.}, Bhatnagar, A., Rai, S.N., \& DeFilippis, A.P. (2016). Dynamic changes in topologically related plasma metabolites following atherothrombotic and non-atherothrombotic myocardial infarction: A weighted network analysis. \emph{Kentucky Academy of Science Annual Meeting}, University of Louisville, Louisville, Kentucky.  \\ 

\textbf{Trainor, P.J.}, \& Rai, S.N. (2015). Comparison of partial least squares-discriminant analysis, random forests, and rotation forests for phenotype discrimination given metabolomic data. \emph{Research!Louisville}, University of Louisville, Louisville, Kentucky.  \\ 

\textbf{Trainor, P.J.} \& Rai, S.N. (2013). Patient rule induction method for identifying subgroups in clinical studies. \emph{Kentucky Academy of Science Annual Meeting}, Morehead State University, Morehead, Kentucky. \\ 

\textbf{Trainor, P.J.}, \& Barnes, C. (2013). Statistical considerations for dedicated cardiovascular safety studies. \emph{Theravance, Inc. (now Theravance Biopharma) company seminar}, San Francisco, California. \\ 

\textbf{Trainor, P.J.}, \& Sloughter, J.M. (2012). Generalized Linear Modeling in Population Ecology. \emph{Northwest Undergraduate Mathematics Symposium}, Lewis \& Clark College, Portland, Oregon. 

\section*{Grants}
 Co-investigator. NIH West Coast Metabolomics Center Pilot and Feasibility Award: ``Targeted plasma metabolomic profiling of thrombotic myocardial infarction''. PI: DeFilippis, A.P. (2017-2018). \$50,000 direct cost. \\

 Co-Investigator. Alpha Phi Foundation Heart-to-Heart Grant (2016 award): ``Circulating Levels of Oxidized Phospholipids, Sub-Clinical Atherosclerosis and Atherosclerotic Cardiovascular Disease''. PI: DeFilippis, A.P. (2016-2017). \$100,000 direct costs. 

\section*{Computing Experience} 
\emph{Programming Languages:} R, Python, C++, MATLAB/Octave \\
\emph{Machine Learning \& Deep Learning:} TensorFlow, Keras, WEKA \\
\emph{Statistical Software:} R, SAS \\
\emph{Bioinformatics Software:} Many packages distributed via Bioconductor \\
\emph{Computer Algebra Systems:} SAGE, Mathematica \\ 
\emph{Operating Systems:} UNIX, Linux (openSUSE is my favorite flavor), and Windows

\section*{AWARDS}
Metabolomics Society Student Travel Award, 2018, Metabolomics Society\\

2\textsuperscript{nd} Place Graduate Research Competition in Health Sciences, 2016, Kentucky Academy of Science\\

1\textsuperscript{st} Place Graduate Research Competition in Mathematics, 2013, Kentucky Academy of Science\\

Janet E. Mills Award for Outstanding Undergraduate Research in Mathematics, Seattle University

}
\end{DoubleSpace*}
}

\end{document}
