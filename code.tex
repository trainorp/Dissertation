This is an example of how to include code listing within dissertation. 

\lstset{language=HTML, linewidth=6in, breaklines=true, 
  breakatwhitespace=true, breakindent=5pt, postbreak=\space}
\begin{lstlisting}{}

<!DOCTYPE html>
<html lang="en">
  <head>
    <title>Rosette Nebula Geometry</title>
    <meta charset="utf-8">
    <meta name="viewport" content="width=device-width, user-scalable=no, minimum-scale=1.0, maximum-scale=1.0">
    <style>
      body {
        color: #fff;
        font-family: Monospace;
        font-size: 13px;
        text-align: center;
        font-weight: bold;

        background-color: #000;
        margin: 0px;
        overflow: hidden;
      }

      #info {
        position: absolute;
        padding: 10px;
        width: 100%;
        text-align: center;
        color: #fff;
      }

      a { color: blue; }

    </style> 
    
  </head>
  <body>
    <div id="info">
            <a href="http://www.astro.louisville.edu" target="_blank">University of Louisville Physics & Astronomy</a>
            <br/>
            Jeremy Huber - Rosette Nebula
    </div>
    
    <!-- Threedotjs -->
    
    <script src="js/three.min.js"></script>

    <!-- Shaders adapted from http://stemkoski.github.io/Three.js/Shader-Glow.html -->

     <script id="starvertexShader" type="x-shader/x-vertex">
      varying float intensity;
      void main() 
      {
                                                              
        intensity = 1.;
        // Pass the 2D position for this vertex to the shader
        gl_Position = projectionMatrix * modelViewMatrix * vec4( position, 1.0 );
      }
    </script>

    <script id="starfragmentShader" type="x-shader/x-vertex"> 
      uniform vec3 starglowColor;
      varying float intensity;
      void main() 
      {
        vec3 glow = starglowColor * intensity;
        gl_FragColor = vec4( glow, 1.0 );
      }
    </script>
    


     <script id="clustervertexShader" type="x-shader/x-vertex">
      void main() 
      {
        // Pass the 2D position for this vertex to the shader
        gl_Position = projectionMatrix * modelViewMatrix * vec4( position, 1.0 );
      }
    </script>

    <script id="clusterfragmentShader" type="x-shader/x-vertex"> 
      uniform vec3 clusterglowColor;
      void main() 
      {
        vec3 glow = clusterglowColor * 1.0;
        gl_FragColor = vec4( glow, 1.0 );
      }
    </script>

  </body>
</html>

\end{lstlisting}
