\section{High-level R code}

\lstset{language=R, linewidth=6in, breaklines=true, 
	breakatwhitespace=true, breakindent=5pt, postbreak=\space}
\subsection{Main high-level Function for generating a BGL object}

\begin{lstlisting}
#' Block Gibbs sampler for Bayesian Graphical Lasso
#'
#' Blockwise sampling from the conditional distribution of a permuted column/row
#' for simulating the posterior distribution for the concentration matrix specifying
#' a Gaussian Graphical Model
#' @param X Data matrix
#' @param iterations Length of Markov chain after burn-in
#' @param burnIn Number of burn-in iterations
#' @param adaptive Logical; Adaptive graphical lasso (TRUE) or regular (FALSE). Default is FALSE.
#' @param lambdaPriora Shrinkage parameter (lambda) gamma distribution shape hyperparameter
#' (Ignored if adaptive=TRUE)
#' @param lambdaPriorb Shrinkage parameter (lambda) gamma distribution scale hyperparameter
#' (Ignored if adaptive=TRUE)
#' @param adaptiveType Choose of adaptive type. Options are "norm" for norm of concentration
#' matrix based adaptivity and "priorHyper" for informative adaptivity
#' @param priorHyper Matrix of gamma scale hyper parameters (Ignord if adaptiveType="norm")
#' @param gammaPriors labmda_ij gamma distribution shape prior (Ignored if adaptive=FALSE)
#' @param gammaPriort lambda_ij gamma distribution rate prior (Ignored if adaptive=FALSE)
#' @param lambdaii lambda_ii hyperparameter (Ignored if adaptive=FALSE)
#' @param keepLambdas Logical: Should lambda MCMC chain (vector / matrix) be kept?
#' @param illStart Method for generating a positive definite estimate of the sample covariance matrix if sample covariance matrix is not semi-positive definite
#' @param rho Regularization parameter for the graphical lasso estimate of the sample covariance matrix (if illStart="glasso")
#' @param verbose logical; if TRUE return MCMC progress
#' @details Implements the block Gibbs sampler for the posterior distribution of
#' a GGM concentration matrix estimated via the Bayesian Graphical Lasso
#' introduced by Wang (2012) or the Bayesian Adaptive Graphical Lasso. For the adaptive case,
#' the element-wise shrinkage parameter is drawn from a gamma distribution where the scale
#' parameter is adaptively modulated directly from the estimated concentration matrix
#' or a user can supply their own informative prior.
#' 
#' @return
#' \item{Omega}{List of concentration matrices from the Markov chains}
#' \item{Lambda}{Vector of simulated lambda parameters}
#' @author Patrick Trainor (University of Louisville)
#' @author Hao Wang
#' @references Wang, H. (2012). Bayesian graphical lasso models and efficient
#' posterior computation. \emph{Bayesian Analysis, 7}(4). <doi:10.1214/12-BA729> .
#' @examples
#' \donttest{
	#' # Generate true covariance matrix:
	#' s<-.9**toeplitz(0:9)
	#' # Generate multivariate normal distribution:
	#' set.seed(5)
	#' x<-MASS::mvrnorm(n=100,mu=rep(0,10),Sigma=s)
	#' blockGLasso(X=x)
	#' }
#' # Same example with short MCMC chain:
#' s<-.9**toeplitz(0:9)
#' set.seed(6)
#' x<-MASS::mvrnorm(n=100,mu=rep(0,10),Sigma=s)
#' blockGLasso(X=x,iterations=100,burnIn=100)
#' @export
blockGLasso<-function(X,iterations=2000,burnIn=1000,adaptive=FALSE,
lambdaPriora=1,lambdaPriorb=1/10,
adaptiveType=c("norm","priorHyper"), priorHyper=NULL,
gammaPriors=1,gammaPriort=1,
lambdaii=1,keepLambdas=TRUE,illStart=c("identity","glasso"),rho=.1,
verbose=TRUE,...)
{
	if(adaptive)
	{
		UseMethod("blockAdGLasso")
	}else{
	UseMethod("blockGLasso")
}
}

\end{lstlisting}

\subsection{Function for calling the non-adaptive block Gibbs sampler}
\begin{lstlisting}
#' @export
blockGLasso.default<-function(X,iterations=2000,burnIn=1000, lambdaPriora=1,lambdaPriorb=1/10,
illStart=c("identity","glasso"), keepLambdas=TRUE,rho=.1,
verbose=TRUE,...){
	# Total iterations:
	totIter<-iterations+burnIn
	
	# Ill conditioned start:
	illStart<-match.arg(illStart)
	
	# Sum of product matrix, covariance matrix, n
	S<-t(X)%*%X
	n=nrow(X)
	Sigma=S/n
	p<-dim(Sigma)[1]
	
	# Concentration matrix and it's dimension:
	if(rcond(Sigma)<.Machine$double.eps){
		if(illStart=="identity"){
			Omega<-diag(nrow(Sigma))+1/(p**2)
		}
		else{
			Omega<-glasso::glasso(cov(X),rho=rho)$wi + 1/(p**2)
		}
	}
	else{
		Omega<-MASS::ginv(Sigma)
	}
	rownames(Omega)<-rownames(Sigma)
	colnames(Omega)<-colnames(Sigma)
	
	# Gamma distirbution posterior parameter a:
	lambdaPosta<-(lambdaPriora+(p*(p+1)/2))
	
	# Keep lambdas?
	keepLambdas<-ifelse(keepLambdas,1L,0L)
	
	# Call sampler:
	bglObj<-bgl(n=n,iters=totIter,lambdaPriorb=lambdaPriorb, lambdaPosta=lambdaPosta,
	S=S,Sigma=Sigma,Omega=Omega,keepLambdas=keepLambdas)
	
	bglObj<-c(bglObj,burnIn=burnIn)
	class(bglObj)<-"BayesianGLasso"
	return(bglObj)
}

\end{lstlisting}

\subsection{Function for calling the non-adaptive block Gibbs sampler}
\begin{lstlisting}
#' @export
blockAdGLasso.default<-function(X,iterations=2000,burnIn=1000, adaptiveType=c("norm","priorHyper"),
priorHyper=NULL,gammaPriors=1,gammaPriort=1,
lambdaii=1,keepLambdas=TRUE,illStart=c("identity","glasso"),
rho=.1,verbose=TRUE,...){
	# Total iterations:
	totIter<-iterations+burnIn
	
	# Ill conditioned start:
	illStart<-match.arg(illStart)
	
	# Sum of product matrix, covariance matrix, n
	S<-t(X)%*%X
	n=nrow(X)
	Sigma=S/n
	p<-dim(Sigma)[1]
	
	# Adaptive type:
	adaptiveType<-match.arg(adaptiveType)
	if(adaptiveType=="priorHyper"){
		if(is.null(priorHyper)) stop("Must specify matrix of prior hyperparameters")
		if(class(priorHyper)!="matrix") stop("Prior hyperparameters must be provided as a matrix")
		if(!all(dim(Sigma)==dim(priorHyper))) stop("Dimesion of hyperparameters does not equal dimension
		of the concentration matrix")
		priorLogical<-TRUE
	}else{
	priorLogical<-FALSE
	priorHyper<-matrix(1,nrow=p,ncol=p)
}

# Concentration matrix and it's dimension:
if(rcond(Sigma)<.Machine$double.eps){
	if(illStart=="identity"){
		Omega<-diag(nrow(Sigma))+1/(p**2)
	}
	else{
		Omega<-glasso::glasso(cov(X),rho=rho)$wi+1/(p**2)
	}
}else{
Omega<-MASS::ginv(Sigma)
}
rownames(Omega)<-rownames(Sigma)
colnames(Omega)<-colnames(Sigma)

# Keep lambdas?
keepLambdas<-ifelse(keepLambdas,1L,0L)

# Call sampler:
bglObj<-bAdgl(n=n, iters=totIter, gammaPriors=gammaPriors,
gammaPriort=gammaPriort, lambdaii=lambdaii, S=S, Sigma=Sigma,
Omega=Omega, priorLogical=priorLogical, priorHyper=priorHyper,
keepLambdas=keepLambdas)

bglObj<-c(bglObj,burnIn=burnIn)
class(bglObj)<-"BayesianGLasso"
return(bglObj)
}

\end{lstlisting}

\subsection{Posterior inference method for BGL object}

\begin{lstlisting}
#' Posterior inference 
#'
#' Inferential statistics for the simulated posterior distributions of the covariance 
#' and concentration parameter matrices for Gaussian Graphical Models estimated
#' via the adaptive or non-adaptive Bayesian Graphical Lasso
#' 
#' @param x Bayesian Graphical Lasso object as returned by the function blockGLasso
#' @param parameter Options are "Omega" for the posterior distribution of the 
#' concentration matrix or "Sigma" for the covariance matrix
#' @return
#' \item{posteriorMean}{Matrix of mean values of the simulated posterior distribution}
#' \item{posteriorSd}{Matrix of standard deviation values of the simulated posterior distribution}
#' \item{posteriorMedian}{Matrix of median values of the simulated posterior distribution}
#' \item{lowerCI}{Lower limit of the Bayesian credible interval for each matrix parameter}
#' \item{upperCI}{Upper limit of the Bayesian credible interval for each matrix parameter}
#' @export
posteriorInference<-function(bglObj,parameter="Omega",
alpha=.05,...){
	UseMethod("posteriorInference")
}

#' @export
posteriorInference.BayesianGLasso<-function(bglObj,parameter="Omega",
alpha=.05){
	# Check parameters
	if(! parameter %in% c("Omega","Sigma")){
	stop("Parameter argument must be Omega (concentration matrix) or 
	Sigma (covariance matrix)")
}

stats<-c("mean","sd","length","median")

# Discard burn in:
x2<-bglObj
x2$Sigmas<-x2$Sigmas[(x2$burnIn+1):length(x2$Sigmas)]
x2$Omegas<-x2$Omegas[(x2$burnIn+1):length(x2$Omegas)]
if(!is.null(x2$lambdas)) x2$lambdas<-x2$lambdas[(x2$burnIn+1):length(x2$lambdas)]

for(stat in stats){
	statMatrix<-matrix(NA,nrow=nrow(x2$Omegas[[1]]), ncol=ncol(x2$Omegas[[1]]))
	for(i in 1:nrow(statMatrix)){
		for(j in 1:ncol(statMatrix)){
			statMatrix[i,j]<-eval(call(stat, sapply(x2$Omegas,FUN=function(y) y[i,j])))
		}
	}
	assign(paste0("stat",stat),statMatrix)
}

lq<-function(z) quantile(z,probs=alpha/2)
uq<-function(z) quantile(z,probs=1-(alpha/2))
lqMatrix<-uqMatrix<-matrix(NA,nrow=nrow(x2$Omegas[[1]]), ncol=ncol(x2$Omegas[[1]]))
for(i in 1:nrow(lqMatrix)){
	for(j in 1:ncol(lqMatrix)){
		lqMatrix[i,j]<-eval(call("lq",sapply(x2$Omegas, FUN=function(y) y[i,j])))
		uqMatrix[i,j]<-eval(call("uq",sapply(x2$Omegas, FUN=function(y) y[i,j])))
	}
}

list(parameter=parameter,alpha=alpha,posteriorMean=statmean,posteriorSd=statsd,
posteriorMedian=statmedian,lowerCI=lqMatrix,upperCI=uqMatrix)
}

\end{lstlisting}

\lstset{language=C++, linewidth=6in, breaklines=true, 
  breakatwhitespace=true, breakindent=5pt, postbreak=\space}
\section{Preamble and internal C++ functions}
\begin{lstlisting}{}
#include <RcppArmadillo.h>
// #include <gperftools/profiler.h>

// [[Rcpp::depends(RcppArmadillo)]]

using namespace Rcpp;
// using namespace arma; 

double rInvG(double mu, double lambda){
	double val = 0;
	double z,y,x,u;
	z = rnorm(1)[0];
	y = z*z;
	x = mu + 0.5*mu*mu*y/lambda - 0.5*(mu/lambda) * sqrt(4*mu*lambda*y + mu*mu*y*y);
	u = runif(1)[0];
	if(u <= mu/(mu+x)){
		val = x;
	}else{
	val = mu*mu/x;
}
return(val);
}

arma::vec upperTri(arma::mat a){
	arma::mat b = trimatu(a);
	arma::mat b2 = trimatl(a);
	arma::uvec c = find(b);
	arma::uvec c2 = find(b2);
	NumericVector d = wrap(c);
	NumericVector d2 = wrap(c2);
	NumericVector e = setdiff(d,d2);
	e=e.sort();
	arma::uvec f = as<arma::uvec>(e);
	arma::vec g = a.elem(f);
	return g; 
}

IntegerMatrix permFun(int p){
	IntegerVector permInt = seq_len(p) - 1;
	IntegerMatrix perms(p-1,p);
	IntegerVector permInt2(p);
	for(int i = 0; i < p; i++)
	{
		permInt2 = permInt[permInt != i];
		perms(_,i) = permInt2;
	}
	return perms;
}
\end{lstlisting}

\section{Regular Bayesian Graphical Lasso C++ code}
\begin{lstlisting}{}
// [[Rcpp::export]]
List bgl(int n, int iters, double lambdaPriorb, double lambdaPosta,
arma::mat S, arma::mat Sigma, arma::mat Omega, int keepLambdas){
	// ProfilerStart("myprof.log");
	
	int p = Omega.n_cols;
	IntegerMatrix idk = permFun(p);
	List OmegaList(iters);
	
	NumericVector lambdaList;
	if(keepLambdas==1){
		lambdaList = NumericVector(iters);
	}
	else{
		lambdaList = NumericVector(1);
	}
	
	for(int iter=0; iter < iters; ++iter){
		
		// Sample lambda:
		arma::vec Omega2 = vectorise(Omega);
		double lambdaPostb = lambdaPriorb + sum(abs(Omega2)) / 2.0;
		double lambda = R::rgamma(lambdaPosta,1.0 / lambdaPostb);
		
		if(keepLambdas==1){
			lambdaList(iter) = lambda;
		}
		
		// Mu prime:
		arma::vec OmegaTemp = abs(upperTri(Omega));
		// arma::vec OmegaTemp = abs(Omega.elem(find(trimatu(Omega,1))));
		arma::vec mup = lambda/OmegaTemp;
		/* double mupThresh = pow(10.0,12.0);
		arma::uvec mupReplace = find(mup > mupThresh);
		mup.elem(mupReplace).fill(mupThresh); */
		
		// Sample tau:
		arma::vec rIG(p*(p-1)/2);
		for(unsigned int i = 0; i < mup.n_elem; i++){
			rIG(i) = rInvG(mup[i],lambda*lambda);
		}
		arma::vec tauVec = 1.0 / rIG;
		arma::mat tau(Omega.n_rows,Omega.n_cols,arma::fill::zeros);
		int k = 0;
		for(unsigned int j = 0; j < tau.n_cols; j++){
			for(unsigned int i = 0; i < j; i++){
				tau(i,j) = tauVec(k);
				tau(j,i) = tauVec(k);
				k++;
			}
		}
		
		IntegerVector tauIPerms(p);
		arma::uvec tauIPerms2(p), tauII(1);
		arma::vec Sigma12(p-1), rnorm1(p-1), beta(p-1);
		arma::mat tauI(p-1,1), Sigma11(p-1,p-1), Omega11inv(p-1,p-1), Ci(p-1,p-1), CiChol(p-1,p-1), mui(p-1,1);
		arma::mat OmegaInvTemp(p-1,1);
		double gamm;
		arma::mat gamm2;
		for(int i = 0; i < p; i++){
			tauIPerms = idk(_,i);
			tauIPerms2 = as<arma::uvec>(tauIPerms);
			tauII(0) = i;
			tauI = tau.submat(tauIPerms2, tauII);
			Sigma11 = Sigma.submat(tauIPerms2, tauIPerms2);
			Sigma12 = Sigma.submat(tauIPerms2, tauII);
			Omega11inv = Sigma11 - ((Sigma12 * Sigma12.t()) / Sigma(i,i));
			Ci = (S(i,i) + lambda) * Omega11inv + diagmat(1 / tauI);
			CiChol = arma::chol(Ci);
			mui = arma::solve(-Ci, S.submat(tauIPerms2, tauII));
			rnorm1 = rnorm(p-1);
			beta = mui + arma::solve(CiChol, rnorm1);
			Omega.submat(tauIPerms2, tauII) = beta;
			Omega.submat(tauII, tauIPerms2) = beta.t();
			gamm = rgamma(1,n/2+1,2/(S(i,i) + lambda))[0];
			gamm2 = gamm + beta.t() * Omega11inv * beta;
			Omega(i,i) = gamm2(0,0);
			OmegaInvTemp = Omega11inv * beta;
			Sigma.submat(tauIPerms2, tauIPerms2) = Omega11inv + (OmegaInvTemp * OmegaInvTemp.t())/gamm;
			Sigma.submat(tauIPerms2, tauII) = -OmegaInvTemp/gamm;
			Sigma.submat(tauII,tauIPerms2) = -OmegaInvTemp.t()/gamm;
			Sigma(i,i) = 1/gamm;
		}
		OmegaList(iter) = Omega;
		Rcpp::Rcout << "iter = " << iter + 1 << std::endl;
		
	}
	// ProfilerStop();
	
	return List::create(Named("lambdas") = lambdaList,
	Named("Omegas") = OmegaList);
}
\end{lstlisting}

\section{Adaptive Bayesian Graphical Lasso C++ code}
\begin{lstlisting}{}
// [[Rcpp::export]]
List bAdgl(int n, int iters, double gammaPriors, double gammaPriort,
double lambdaii, arma::mat S, arma::mat Sigma,
arma::mat Omega, bool priorLogical, arma::mat priorHyper,
int keepLambdas){
	// ProfilerStart("myprof.log");
	
	int p = Omega.n_cols;
	IntegerMatrix idk = permFun(p);
	
	List OmegaList(iters);
	
	List LambdaList;
	if(keepLambdas==1){
		LambdaList = List(iters);
	}
	else{
		LambdaList = List(1);
	}
	
	for(int iter=0; iter < iters; ++iter){
		// Sample lambdas:
		arma::vec Omega2 = vectorise(Omega);
		arma::vec OmegaTemp = abs(upperTri(Omega));
		// arma::vec OmegaTemp = abs(Omega.elem(find(trimatu(Omega,1))));
		
		// Gamma distirbution conditional parameters:
		arma::vec tt = OmegaTemp + gammaPriort;
		
		// If informative prior:
		if(priorLogical){
			arma::vec priorHyperVec = vectorise(priorHyper);
			priorHyperVec = upperTri(priorHyper);
			// priorHyperVec = priorHyperVec.elem(find(trimatu(priorHyper,1)));
			tt = tt + priorHyperVec;
		}
		
		arma::vec s(tt.n_elem);
		s.fill(gammaPriors + 1);
		
		// Sample lambda:
		arma::vec lambda(tt.n_elem);
		for(unsigned int ii = 0; ii<tt.n_elem; ++ii){
			lambda(ii) = R::rgamma(s(ii), 1.0/tt(ii)); // Change to Rcpp sugar version
		}
		
		// Mu prime:
		arma::vec mup = lambda/OmegaTemp;
		double mupThresh = pow(10.0,12.0);
		arma::uvec mupReplace = find(mup > mupThresh);
		mup.elem(mupReplace).fill(mupThresh);
		
		// Sample tau:
		arma::vec rIG(p*(p-1)/2);
		for(unsigned int i = 0; i < mup.n_elem; i++){
			rIG(i) = rInvG(mup(i),pow(lambda(i),2));
		}
		arma::vec tauVec = 1.0 / rIG;
		arma::mat tau(Omega.n_rows,Omega.n_cols,arma::fill::zeros);
		int k = 0;
		for(unsigned int j = 0; j < tau.n_cols; j++){
			for(unsigned int i = 0; i < j; i++){
				tau(i,j) = tauVec(k);
				tau(j,i) = tauVec(k);
				k++;
			}
		}
		
		IntegerVector tauIPerms(p);
		arma::uvec tauIPerms2(p), tauII(1);
		arma::vec Sigma12(p-1), rnorm1(p-1), beta(p-1);
		arma::mat tauI(p-1,1), Sigma11(p-1,p-1), Omega11inv(p-1,p-1), Ci(p-1,p-1), CiChol(p-1,p-1), mui(p-1,1);
		arma::mat OmegaInvTemp(p-1,1);
		double gamm;
		arma::mat gamm2;
		for(int i = 0; i < p; i++){
			tauIPerms = idk(_,i);
			tauIPerms2 = as<arma::uvec>(tauIPerms);
			tauII(0) = i;
			tauI = tau.submat(tauIPerms2, tauII);
			Sigma11 = Sigma.submat(tauIPerms2, tauIPerms2);
			Sigma12 = Sigma.submat(tauIPerms2, tauII);
			Omega11inv = Sigma11 - ((Sigma12 * Sigma12.t()) / Sigma(i,i));
			Ci = (S(i,i) + lambdaii) * Omega11inv + diagmat(1 / tauI);
			
			CiChol = arma::chol(Ci);
			mui = arma::solve(-Ci, S.submat(tauIPerms2, tauII));
			rnorm1 = rnorm(p-1);
			beta = mui + arma::solve(CiChol, rnorm1);
			Omega.submat(tauIPerms2, tauII) = beta;
			Omega.submat(tauII, tauIPerms2) = beta.t();
			gamm = rgamma(1,n/2+1,2/(S(i,i) + lambdaii))[0];
			gamm2 = gamm + beta.t() * Omega11inv * beta;
			Omega(i,i) = gamm2(0,0);
			OmegaInvTemp = Omega11inv * beta;
			Sigma.submat(tauIPerms2, tauIPerms2) = Omega11inv + (OmegaInvTemp * OmegaInvTemp.t())/gamm;
			Sigma.submat(tauIPerms2, tauII) = -OmegaInvTemp/gamm;
			Sigma.submat(tauII,tauIPerms2) = -OmegaInvTemp.t()/gamm;
			Sigma(i,i) = 1/gamm;
		}
		OmegaList(iter) = Omega;
		
		arma::mat lambdaMat(Omega.n_rows,Omega.n_cols,arma::fill::zeros);
		int kk = 0;
		for(unsigned int j = 0; j < lambdaMat.n_cols; j++){
			for(unsigned int i = 0; i < j; i++){
				lambdaMat(i,j) = lambda(kk);
				kk++;
			}
		}
		if(keepLambdas==1){
			LambdaList(iter) = lambdaMat;
		}
		
		double detOmega = arma::det(Omega);
		Rcpp::Rcout << "iter = " << iter + 1 << " det(Omega) = " << detOmega << std::endl;
	}
	
	return List::create(Named("lambdas") = LambdaList,
	Named("Omegas") = OmegaList);
}
\end{lstlisting}