\begin{DoubleSpace*}
\section{Challenges in metabolomic data analyses}
Metabolomics, or the study of small molecules in biological systems, has enjoyed great success in enabling researchers to make inferences regarding disease associated metabolic dysregulation as well as in furnishing biomarkers of disease and phenotypic states \cite{dunn2013,carlisle2016,johnson2016,newgard2017}. In spite of recent technological advances in the analytical platforms utilized in metabolomics \cite{gowda2014,gowda2016}, advances in strategies for studying disease associated changes in metabolism \cite{bruntz2017,krycer2017}, and advances in the democratization (such as by web-based tools) of metabolomics data analyses \cite{warth2017,guijas2018,chong2018}, significant challenges in metabolomics data analyses remain. The first challenge that we confront in the current work is making ``higher order inference'' in metabolomics (Chapters~\ref{higher},\ref{structureADBGL}, and \ref{hdInteractome}). Prior to arriving at our discussion of higher order inference in metabolomics, we introduce metabolism and metabolomics more generally in Chapter~\ref{prelimMetab}. We define higher order inference as the testing of hypotheses that are more complex than single metabolite hypothesis tests. In studying metabolism, research questions are often over systems of metabolites or metabolic processes. For example, a biomedical scientist might wish to know how a statin impacts cholesterol metabolism. Likewise a cancer researcher may wish to know how the knockdown or knockout of a specific gene alters cellular metabolism. Formulating such research questions as hypothesis tests is not straightforward. The first step in developing such statistical tests is developing a reference model for the system of interest. In the example of a biomedical scientist studying a statin, a reference model of how the metabolites of cholesterol metabolism interact (probabilistically) is required in order to determine how the system has changed following the treatment of a human (or model organism) with a statin. As we argue in Chapter~\ref{higher}, the current paradigm of pathway-based or \emph{a priori} knowledge-based enrichment analyses may suffer from extreme bias, especially in the case of analyzing biofluids such as blood plasma, serum, or urine. Unbiased, or data-dependent approaches such as the construction of correlation networks also suffer from significant methodological issues, as we discuss (Chapter~\ref{higher}). Consequently, we propose a new approach that allows for the construction of metabolite ``interactomes'' or probabilistic models which are organism and sample media specific that can be used as reference models for studying perturbations in metabolic systems. Given the high-dimensionality of metabolite abundance data from untargeted metabolomics experiments, we propose utilizing informative priors that are generated via the analysis of molecular structure similarity in the estimation of such models. 

In the second part of this work, we confront the challenge of developing diagnostic classification models utilizing metabolite abundances that do not ``overfit'' relatively small sample sizes. Often poor generalization error results from unrestricted optimization of likelihoods in models with a high ratio of model parameters to sample size of a training dataset \cite{hastie2009}. This is often referred to as the $p>>n$ problem (where, in this context, $p$ is the number of metabolites and $n$ is the number of samples). Many metabolomics datasets are characterized by a large $p$ (e.g. $>30,000$ mass features can be observed in high resolution mass spectrometry experiments), while the substantial costs associated with the acquisition of metabolomics data implies that small datasets are a persistent feature of the field. As more resources are devoted to personalized medicine \cite{hamburg2010,wishart2016} and deep phenotyping \cite{delude2015}, diagnostic classifiers that are robust given $p>>n$ and minimize the likelihood of overfitting are essential. 

In Chapter~\ref{diagnostic} we discuss such a classification methodology, applied to a critical clinical problem: the detection and discrimination of the subtype of acute myocardial infarction (colloquially: heart attack) utilizing metabolite abundance data quantified from blood plasma. Heart disease is the leading cause of global mortality \cite{benjamin2017}. Myocardial Infarction (MI), an acute manifestation of heart disease, is characterized by heterogeneous etiology \cite{thygesen2012}. Consequently, a blood-based and non-invasive diagnostic test that could be administered upon presentation to an emergency department and could differentiate between thrombotic MI and non-thrombotic MI would be of great utility. Our laboratory (the Atherosclerosis / Atherothrombosis Research Laboratory [AARL] at the University of Louisville) has recruited two clinical cohorts that are designed for the development and validation of a diagnostic test capable of differentiating thrombotic MI, non-thrombotic MI, and stable coronary artery disease; the first cohort has been described previously \cite{defilippis2015,defilippis2017,trainor2017}. As a critical step towards the development of such a diagnostic, in Chapter~\ref{diagnostic} we present multinomial logistic regression models utilizing the metabolite abundances for the metabolites that were selected as part of a separate feature selection work \cite{trainor2018}. We compare the performance of the Bayesian approach we have employed for estimating model coefficients with maximum likelihood estimation and discuss the merits of the Bayesian approach. 

In the final part of this dissertation we discuss one of the most challenging aspects of mass spectrometry-based metabolomics \cite{dunn2012}, that is, compound identification. In this part, we review Bayesian approaches for compound identification in metabolomics experiments that utilize liquid chromatography-mass spectrometry (LC-MS).

\section{Bayesian methodology for metabolomics}
While this dissertation presents three different analytical problems that practitioners in the field of metabolomics currently face, the common aspect of the solutions that we describe is Bayesian reasoning. Bayesian methodologies for the three types of data analysis problems discussed in the current work are seldom utilized and enjoy scant popularity in the field of metabolomics. As we hope to demonstrate in the current work, Bayesian approaches allow practitioners to incorporate prior scientific knowledge into the data analysis process. In the first part, (the generation of metabolite interactomes) the prior belief that is postulated is that the enzyme catalyzed biochemical reactions that convert one intermediate into another generate both dependence in molecular structure similarity and statistical dependence in the distribution of metabolite abundances. In the second part (generation of a statistical classifier for discriminating MI type), the prior belief that is postulated is that the magnitude of regression coefficients should be small given the relatively small sample size (and high ratio of $p$ to $n$). In the final part (Bayesian methods for LC-MS compound identification), various prior beliefs are incorporated such as that the presence of one compound in a dataset implies that the probability closely related (by chemical modifications) compounds is more likely. 

\section{Layout of the dissertation}
The layout of the dissertation is as follows: Chapter~\ref{prelimMetab} provides preliminary theory regarding metabolism and metabolomics, Chapter~\ref{prelimBayesian} provides a cursory introduction to Bayesian statistics, Chapter~\ref{prelimGGM} provides an introduction to the class of model we will utilize for generating metabolite interactomes, Chapter~\ref{higher} presents the current paradigms for making higher order inferences in metabolomics and sets the stage for Chapter~\ref{structureADBGL} in which we present our novel methodology for using molecular structure similarity to generate informative priors for the estimation of metabolite interactomes. Chapter~\ref{hdInteractome} presents such an interactome for stable heart disease. We then transition to the development of a Bayesian diagnostic model for the detection and differentiation of MI type in Chapter~\ref{diagnostic}. In Chapter~\ref{bayesianID} we transition again, and provide a review of Bayesian approaches utilized to identify compounds in metabolomics experiments using LC-MS. Finally, we conclude this dissertation in Chapter~\ref{conclusions} by presenting conclusions we have drawn from the complete work and a discussion of the methodologies herein, as well as future directions that we are pursuing as a result of this work.

\end{DoubleSpace*}