\begin{DoubleSpace*}
\section{Challenges in metabolomic data analyses}
Metabolomics, or the study of small molecules in biological systems, has enjoyed great success in enabling researchers to make inferences regarding disease associated metabolic dysregulation as well as in furnishing biomarkers of disease and phenotypic states \cite{dunn2013,carlisle2016,johnson2016,newgard2017}. In spite of recent technological advances in the analytical platforms utilized in metabolomics \cite{gowda2014,gowda2016}, advances in strategies for studying disease associated changes in metabolism \cite{bruntz2017,krycer2017}, and advances in the democratization (such as by web-based tools) of metabolomics data analyses \cite{warth2017,guijas2018,chong2018}, significant challenges in metabolomics data analyses remain. The first challenge that we confront in the current work is making ``higher order inference'' in metabolomics (Chapters~\ref{higher},\ref{structureADBGL}, and \ref{hdInteractome}). We define higher order inference as the testing of hypotheses that are more complex than single metabolite hypothesis tests. In studying metabolism, research questions are often over systems of metabolites or metabolic processes. For example, a biomedical scientist might wish to know how a statin impacts cholesterol metabolism. Likewise a cancer researcher may wish to know how the knockdown or knockout of a specific gene alters cellular metabolism. Formulating such research questions as hypothesis tests is not straightforward. The first step in developing such statistical tests is developing a reference model for the system of interest. In the example of a biomedical scientist studying a statin, a reference model of how the metabolites of cholesterol metabolism interact (probabilistically) is required in order to determine how the system has changed following the treatment of a human (or model organism) with a statin. As we argue in the current work, the current paradigm of pathway-based or \emph{a priori} knowledge-based enrichment analyses may suffer from extreme bias, especially in the case of analyzing biofluids such as blood plasma, serum, or urine. Unbiased, or data-dependent approaches such as the construction of correlation networks also suffer from significant methodological issues, as we discuss (Chapter~\ref{higher}). Consequently, we propose a new approach that allows for the construction of metabolite ``interactomes'' or probabilistic models which are organism and sample media specific that can be used as reference models for studying perturbations in metabolic systems. Given the high-dimensionality of metabolite abundance data from untargeted metabolomics experiments, we propose utilizing informative priors that are generated via the analysis of molecular structure similarity in the estimation of such models. 

In the second part of this work, we confront the challenge of developing diagnostic classification models utilizing metabolite abundances that do not ``overfit'' relatively small sample sizes. Often poor generalization error results from unrestricted optimization of likelihoods in models with a high ratio of model parameters to sample size of a training dataset \cite{hastie2009}. This is often referred to as the $p>>n$ problem (where, in this context, $p$ is the number of metabolites and $n$ is the number of samples). Many metabolomics datasets are characterized by a large $p$ (e.g. $>30,000$ mass features can be observed in high resolution mass spectrometry experiments), while the substantial costs associated with the acquisition of metabolomics data implies that small datasets are a persistent feature of the field. As more resources are devoted to personalized medicine \cite{hamburg2010,wishart2016} and deep phenotyping \cite{delude2015}, diagnostic classifiers that are robust given $p>>n$ and minimize the likelihood of overfitting are essential. In Chapter~\ref{diagnostic} we discuss such a classification methodology, applied to a critical clinical problem: the detection and discrimination of the subtype of acute myocardial infarction (colloquially: heart attack) utilizing metabolite abundance data quantified from blood plasma. 

\section{Bayesian statistics}
Blah

\section{Why a bayesian approach to metabolomics}

\section{Aim 1 Introduction}

\section{Aim 2 Introduction}
\end{DoubleSpace*}