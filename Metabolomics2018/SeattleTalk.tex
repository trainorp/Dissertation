\documentclass[xcolor=dvipsnames]{beamer} 
\usetheme{AnnArbor}
\usecolortheme{beaver}

\usepackage{amsmath,graphicx,booktabs,tikz,subfig,color}
\definecolor{mycol}{rgb}{.4,.85,1}
\setbeamercolor{title}{bg=mycol,fg=black} 
\setbeamercolor{palette primary}{use=structure,fg=white,bg=red}
\setbeamercolor{block title}{fg=white,bg=red!50!black}
%\setbeamercolor{block title}{fg=white,bg=blue!75!black}

\DeclareMathOperator{\PP}{P}
\DeclareMathOperator{\EE}{E}
\DeclareMathOperator{\argmin}{argmin}
\DeclareMathOperator{\argmax}{argmax}
\DeclareMathOperator{\ent}{H}
\DeclareMathOperator{\tr}{tr}
\DeclareMathOperator{\sign}{sign}
\DeclareMathOperator{\DE}{DE}
\DeclareMathOperator{\EXP}{EXP}
\DeclareMathOperator{\GA}{GA}
\DeclareMathAlphabet\mathbfcal{OMS}{cmsy}{b}{n}

\begin{document}
	
\title[Bayesian Interactome Modeling]{Inferring metabolite interactomes via Bayesian graphical model selection utilizing molecular structure informative priors}
\author[P.J. Trainor]{Patrick J. Trainor}
\institute[U of L]
{
	
	\includegraphics[scale=.4]{/home/patrick/gdrive//MetabClass/Old/RL2015/IMClogo}\\
	Division of Cardiovascular Medicine \& \\
	Interdisciplinary Ph.D. Program in Bioinformatics \\
	University of Louisville %\\[2ex]
}
\date[June 2018]{June 26, 2018}

\begin{frame}
	\titlepage
\end{frame}

\begin{frame}{Motivation}
	\vspace{-15.5pt}
	{\LARGE In someone else's words...}
	\begin{center}
		\includegraphics[scale=.5]{JohnsonNatureReview}
	\end{center}
	(2016). \emph{Nature Rev Mol Cell Bio, 17}. doi: 10.1038/nrm.2016.25
\end{frame}

\begin{frame}{Motivation}
	\vspace{-10pt}
		{\LARGE A tale of two aims:}
		
		\begin{enumerate}[{1)}]
			\item ``...Aim 2 determines which TAM metabolic pathway(s) is regulated by c-Maf using systems metabolomics approach. We will first determine TAM metabolic pathways using Stable Isotope Resolved Metabolomics (SIRM) approach...'' [1R01CA213990-01]
			\item[]
			\item ``...The objective of the proposed project is to perform untargeted metabolite profiling to reveal metabolite abnormalities that are common to sALS patients and to determine metabolic pathways whose dysregulation contributes to the disease...'' [5F31NS095698-02]
		\end{enumerate}
\end{frame}

\begin{frame}{Motivation}{Biomarker hypotheses}
	\vspace{-10pt}
	You are interested in whether a therapeutic ``changes'' the abundance of a metabolite, $X$ in plasma (e.g. a specific lipid)
	\begin{itemize}
		\item $H_0$: The abundance of $X$ does not differ between two (or more) phenotypes 
		\item $H_A$: The abundance of $X$ does differ between two (or more) phenotypes 
		\item[]
	\end{itemize} 
	
	Or formulated as an equivalence test:
	\begin{itemize}
		\item $H_0$: The abundance of $X$ is not equivalent between two (or more) phenotypes  
		\item $H_A$: The abundance of $X$ is equivalent between two (or more) phenotypes  
	\end{itemize}
\end{frame}

\begin{frame}
	\begin{center}
			\includegraphics[scale=.4]{../Plots/cholesterol.png}
			
			Lieberman, M., Ahles, P., et al. \emph{Wikipathways}: WP197
	\end{center}
\end{frame}

\begin{frame}{Motivation}{Hypotheses}
	\vspace{-10pt}
	You are interested in whether a therapeutic ``changes'' cholesterol biosynthesis
	\begin{itemize}
		\item $H_0$: cholesterol biosynthesis does not differ between two (or more) phenotypes 
		\item $H_A$: cholesterol biosynthesis does differ between two (or more) phenotypes
		\item[]
	\end{itemize} 
	
	Or formulated as an equivalence test:
	\begin{itemize}
		\item $H_0$: cholesterol biosynthesis is the same between two (or more) phenotypes 
		\item $H_A$: cholesterol biosynthesis is not the same between two (or more) phenotypes
	\end{itemize}
\end{frame}

\begin{frame}
	\begin{center}
		\includegraphics[scale=.4]{../Plots/cholesterol.png}
		
		Lieberman, M., Ahles, P., et al. \emph{Wikipathways}: WP197
	\end{center}
\end{frame}

\begin{frame}{Motivation}{Higher order inference}
	\vspace{-10pt}
	\begin{itemize}
		\item Higher-order inference in metabolomics corresponds to the testing of hypotheses that are more complex than single metabolite hypothesis tests
	\end{itemize}
\end{frame}

\begin{frame}{Bias in pathway enrichment analysis}{The reference set}
	\vspace{-15.5pt}
	\begin{figure}
		\includegraphics[scale=.32]{/home/patrick/gdrive/Dissertation/Proposal/vennHell2_1}
	\end{figure}
\end{frame}

\begin{frame}{Bias in pathway enrichment analysis}{The reference set}
	\vspace{-15.5pt}
	\begin{figure}
		\includegraphics[scale=.32]{/home/patrick/gdrive/Dissertation/Proposal/vennHell2_2}
	\end{figure}
\end{frame}

\begin{frame}
	\begin{align}
	L(\boldsymbol{\Omega}|\textbf{X})&=(2 \pi)^{-np/2}|\boldsymbol{\Omega}|^{n/2} \exp \left(-\frac{1}{2}\sum_{i=1}^{n} (\textbf{x}_i-\boldsymbol{\mu})^T \boldsymbol{\Omega} (\textbf{x}_i-\boldsymbol{\mu})\right) \\
	&=(2 \pi)^{-np/2}|\boldsymbol{\Omega}|^{n/2} \exp \left(-\frac{1}{2} \langle \textbf{K}, \boldsymbol{\Omega}\rangle\right)
	\end{align}
\end{frame}

\end{document}