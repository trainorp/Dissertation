\documentclass[xcolor=dvipsnames]{beamer} 
\usetheme{AnnArbor}
\usecolortheme{beaver}

\usepackage{amsmath,graphicx,booktabs,tikz,subfig,color}
\definecolor{mycol}{rgb}{.4,.85,1}
\setbeamercolor{title}{bg=mycol,fg=black} 
\setbeamercolor{palette primary}{use=structure,fg=white,bg=red}
\setbeamercolor{block title}{fg=white,bg=red!50!black}
%\setbeamercolor{block title}{fg=white,bg=blue!75!black}

\DeclareMathOperator{\PP}{P}
\DeclareMathOperator{\EE}{E}
\DeclareMathOperator{\argmin}{argmin}
\DeclareMathOperator{\argmax}{argmax}
\DeclareMathOperator{\ent}{H}
\DeclareMathOperator{\tr}{tr}
\DeclareMathOperator{\sign}{sign}
\DeclareMathOperator{\DE}{DE}
\DeclareMathOperator{\EXP}{EXP}
\DeclareMathOperator{\GA}{GA}
\DeclareMathAlphabet\mathbfcal{OMS}{cmsy}{b}{n}

\begin{document}
	
\title[Bayesian Interactome Modeling]{Inferring metabolite interactomes via Bayesian graphical model selection utilizing molecular structure informative priors}
\author[P.J. Trainor]{Patrick J. Trainor}
\institute[U of L]
{
	
	\includegraphics[scale=.4]{/home/patrick/gdrive//MetabClass/Old/RL2015/IMClogo}\\
	Division of Cardiovascular Medicine \& \\
	Interdisciplinary Ph.D. Program in Bioinformatics \\
	University of Louisville %\\[2ex]
}
\date[June 2018]{June 26, 2018}

\begin{frame}
	\titlepage
\end{frame}

\begin{frame}{Motivation}
	\vspace{-15.5pt}
	{\LARGE In someone else's words...}
	\begin{center}
		\includegraphics[scale=.5]{JohnsonNatureReview}
	\end{center}
	(2016). \emph{Nature Rev Mol Cell Bio, 17}. doi: 10.1038/nrm.2016.25
\end{frame}

\begin{frame}{Motivation}
	\vspace{-10pt}
		{\LARGE A tale of two aims:}
		
		\begin{enumerate}[{1)}]
			\item ``...Aim 2 determines which TAM metabolic pathway(s) is regulated by c-Maf using systems metabolomics approach. We will first determine TAM metabolic pathways using Stable Isotope Resolved Metabolomics (SIRM) approach...'' [1R01CA213990-01]
			\item[]
			\item ``...The objective of the proposed project is to perform untargeted metabolite profiling to reveal metabolite abnormalities that are common to sALS patients and to determine metabolic pathways whose dysregulation contributes to the disease...'' [5F31NS095698-02]
		\end{enumerate}
\end{frame}

\begin{frame}{Motivation}{Biomarker hypotheses}
	\vspace{-10pt}
	You are interested in whether a therapeutic ``changes'' the abundance of a metabolite, $X$ in plasma (e.g. a specific lipid)
	\begin{itemize}
		\item $H_0$: The abundance of $X$ does not differ between two (or more) phenotypes 
		\item $H_A$: The abundance of $X$ does differ between two (or more) phenotypes 
		\item[]
	\end{itemize} 
	
	Or formulated as an equivalence test:
	\begin{itemize}
		\item $H_0$: The abundance of $X$ is not equivalent between two (or more) phenotypes  
		\item $H_A$: The abundance of $X$ is equivalent between two (or more) phenotypes  
	\end{itemize}
\end{frame}

\begin{frame}
	\begin{center}
			\includegraphics[scale=.4]{../Plots/cholesterol.png}
			
			Lieberman, M., Ahles, P., et al. \emph{Wikipathways}: WP197
	\end{center}
\end{frame}

\begin{frame}{Motivation}{Hypotheses}
	\vspace{-10pt}
	You are interested in whether a therapeutic ``changes'' cholesterol biosynthesis
	\begin{itemize}
		\item $H_0$: cholesterol biosynthesis does not differ between two (or more) phenotypes 
		\item $H_A$: cholesterol biosynthesis does differ between two (or more) phenotypes
		\item[]
	\end{itemize} 
	
	Or formulated as an equivalence test:
	\begin{itemize}
		\item $H_0$: cholesterol biosynthesis is the same between two (or more) phenotypes 
		\item $H_A$: cholesterol biosynthesis is not the same between two (or more) phenotypes
	\end{itemize}
\end{frame}

\begin{frame}
	\begin{center}
		\includegraphics[scale=.4]{../Plots/cholesterol.png}
		
		Lieberman, M., Ahles, P., et al. \emph{Wikipathways}: WP197
	\end{center}
\end{frame}

\begin{frame}{Motivation}{Higher order inference}
	\vspace{-10pt}
	{\Large Higher-order inference: the testing of hypotheses that are more complex than single metabolite hypothesis tests} \vspace{7pt}
	
	\begin{itemize}
		\item \textbf{``Data dependent'' / ``Unbiased'' Approaches:} 
		\begin{itemize}
		\item Analyses of correlation networks
		\item Module analyses (WGCNA)
		\item Multivariate statistical approaches (Clustering; Latent component modeling such as PCA, PLS-R, PLS-DA)
		\end{itemize}
		\item[]
		\item \textbf{\emph{A priori approaches}--Pathway (or other ontology) enrichment analyses:} 
		\begin{itemize}
			\item Discrete statistical tests (e.g. hypergeometric)
			\item KS statistic-based set enrichments (e.g. GSEA/MSEA)
		\end{itemize}
	\end{itemize}
	
\end{frame}

\begin{frame}{Pathway enrichment analysis}
	\vspace{-10pt}
{\Large Test for pathway enrichment: Does a pathway have more ``significant'' differences in metabolite abundances than expected?} \vspace{10pt}

	\begin{itemize}
		\item pmf for the hypergeometric distribution: 
		\begin{itemize}
			\item $N=$ total metabolites,
			\item $K=$ number of differentially abundant metabolites, 
			\item $n=$ metabolites in the pathway of interest, 
			\item $k=$ number of differentially abundant metabolites in the pathway of interest 
		\end{itemize}
		\begin{align*}
		\PP(X=k) = \frac{  {{K}\choose{k}} {{N-K}\choose{n-k}}  }{ {{N}\choose{n}} } 
		\end{align*}
	\end{itemize}
\end{frame}

\begin{frame}{Bias in pathway enrichment analysis}{The reference set}
	\vspace{-15.5pt}
	\begin{figure}
		\includegraphics[scale=.32]{/home/patrick/gdrive/Dissertation/Proposal/vennHell2_1}
	\end{figure}
\end{frame}

\begin{frame}{Bias in pathway enrichment analysis}{The reference set}
	\vspace{-15.5pt}
	\begin{figure}
		\includegraphics[scale=.32]{/home/patrick/gdrive/Dissertation/Proposal/vennHell2_2}
	\end{figure}
\end{frame}

\begin{frame}{Correlation networks}{What do they tell you?}
	{I simulated this...}
	\begin{center}
		\includegraphics[scale=.5]{../../Aim2/Plots/AR1_Om}
	\end{center}
\end{frame}

\begin{frame}{Correlation networks}{What do they tell you?}
	{And fit a correlation network...}
	\begin{center}
		\includegraphics[scale=.5]{../../Aim2/Plots/AR1_Cor}
	\end{center}
\end{frame}

\begin{frame}{Partial correlation}{A better approach}
	\vspace{-15.5pt}
	{\Large \textbf{Gaussian Graphical Models (GGM)}: from correlation to partial correlation...}
	
	\begin{center}
		\includegraphics[scale=.75]{ggmPrev} \\
		Krumsiek et al. (2011). \emph{BMC Syst Bio, 5}(21) doi: 10.1186/1752-0509-5-21
	\end{center}
\end{frame}

\begin{frame}{Partial correlation}{A better approach}
	\vspace{-15.5pt}
	\begin{center}
		\includegraphics[scale=.75]{ggmPrev} \\
		Krumsiek et al. (2011). \emph{BMC Syst Bio, 5}(21) doi: 10.1186/1752-0509-5-21
	\end{center}
	\begin{itemize}
		\item Problem: requires estimating $\frac{p*(p-1)}{2}$ coefficients
		\item[]
		\item Not feasible for most metabolomics studies 
	\end{itemize}
\end{frame}

\begin{frame}{Motivation}
	\begin{itemize}
		\item Succinct problem statement: 
		\begin{itemize}
			\item To make higher-order inference a model of how metabolites are related is required
			\item[]
			\item Unbiased approach (using a GGM to estimate partial correlation coefficients) is often unfeasible 
			\item[]
			\item Pathway-based approach may be inappropriate (given the experimental system, sample media, etc.)
		\end{itemize}
		\item[]
		\item A Bayesian call to action!
	\end{itemize}

\end{frame}

\begin{frame}
	\begin{align}
	L(\boldsymbol{\Omega}|\textbf{X})&=(2 \pi)^{-np/2}|\boldsymbol{\Omega}|^{n/2} \exp \left(-\frac{1}{2}\sum_{i=1}^{n} (\textbf{x}_i-\boldsymbol{\mu})^T \boldsymbol{\Omega} (\textbf{x}_i-\boldsymbol{\mu})\right) \\
	&=(2 \pi)^{-np/2}|\boldsymbol{\Omega}|^{n/2} \exp \left(-\frac{1}{2} \langle \textbf{K}, \boldsymbol{\Omega}\rangle\right)
	\end{align}
\end{frame}

\end{document}