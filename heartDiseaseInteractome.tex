\section{Cohort}

In order to determine changes in the plasma metabolome associated with myocardial infarction (MI) characterized by thrombotic etiology versus non-thrombotic etiology, DeFilippis and colleagues assembled a human cohort as previously described (DeFilippis et al., 2016; DeFilippis et al., 2017; Trainor et al., 2017). Briefly, 80 human subjects presenting with suspected acute MI or stable coronary artery disease (CAD) were enrolled. Utilizing a stringent criteria based on clinical presentation, angiographic evidence, and histological evidence, MI subjects were adjudicated as thrombotic MI or non-thrombotic MI. Blood samples were collected at the time of acute presentation (presentation to the coronary artery catheterization lab prior to procedures) and at a follow-up evaluation approximately three months later. To estimate the structure of a stable heart disease plasma interactome, we used the follow-up evaluations from all available MI subjects as well as the evaluations from stable CAD subjects. The analytical sample thus consisted of 47 whole blood samples from human subjects with definitive heart disease who were not experiencing an acute event at the time of sampling. 

\section{Plasma metabolomics}
Details of the metabolite quantification have been described previously (Trainor et al., 2017), but a brief overview is provided as follows. Plasma samples were prepared from whole blood and a recovery standard was added. Vigorous shaking was applied utilizing a GenoGrinder 2000 (Glen Mills, Metuchen, NJ) and methanol was added and to precipitate proteins. The extract containing small molecules was divided into five aliquots, four of which were analyzed using different platforms while the remaining aliquot was reserved. Two aliquots were analyzed by ultra-performance liquid chromatography-tandem mass spectrometry (UPLC-MS/MS) with negative and positive ion mode electrospray ionization (ESI). A third aliquot was also analyzed by UPLC-MS/MS with negative ion mode ESI and a method optimized for polar metabolite detection. The fourth aliquot was analyzed by gas chromatography-mass spectrometry (GC-MS). 1,032 chemical features  were detected utilizing the multiple platforms in the analysis of the plasma samples. Of these, 590 compounds were identified by matching to authentic standards based on retention index, mass to charge ratio, and MS2 data; 73 were identified based on experimental data matched to curated databases; and 369 could not be confidently identified. As the original data dependent acquisition was conducted utilizing both acute event samples and stable heart disease samples, metabolites not detected in the stable heart disease samples were removed. Metabolites missing from greater than 70\% of the samples or without compound identification were also removed, resulting in a final dataset with 522 metabolites across 47 samples. Minimum values were then imputed for the remaining metabolite relative abundances with missing data. As many of the metabolites exhibited approximately log-normal relative abundance distributions, metabolite abundances were log-transformed. Finally, the data was mean centered so that each metabolite’s relative abundance distribution was centered about zero. 

To approximate the posterior distribution of $p(\boldsymbol{\Omega}|\textbf{X})$, a Markov Chain was generated of length 1,000 with a 250 iteration burn-in period. From each sample from the posterior distribution $p(\boldsymbol{\Omega}|\textbf{X})$, a partial correlation coefficient matrix was computed, yielding a simulated posterior distribution for the matrix of partial correlation coefficients.