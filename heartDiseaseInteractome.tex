\section{Cohort}

In order to determine changes in the plasma metabolome associated with myocardial infarction (MI) characterized by thrombotic etiology versus non-thrombotic etiology, DeFilippis and colleagues assembled a human cohort as previously described (DeFilippis et al., 2016; DeFilippis et al., 2017; Trainor et al., 2017). Briefly, 80 human subjects presenting with suspected acute MI or stable coronary artery disease (CAD) were enrolled. Utilizing a stringent criteria based on clinical presentation, angiographic evidence, and histological evidence, MI subjects were adjudicated as thrombotic MI or non-thrombotic MI. Blood samples were collected at the time of acute presentation (presentation to the coronary artery catheterization lab prior to procedures) and at a follow-up evaluation approximately three months later. To estimate the structure of a stable heart disease plasma interactome, we used the follow-up evaluations from all available MI subjects as well as the evaluations from stable CAD subjects. The analytical sample thus consisted of 47 whole blood samples from human subjects with definitive heart disease who were not experiencing an acute event at the time of sampling. 

\section{Plasma metabolomics}
Details of the metabolite quantification have been described previously (Trainor et al., 2017), but a brief overview is provided as follows. Plasma samples were prepared from whole blood and a recovery standard was added. Vigorous shaking was applied utilizing a GenoGrinder 2000 (Glen Mills, Metuchen, NJ) and methanol was added and to precipitate proteins. The extract containing small molecules was divided into five aliquots, four of which were analyzed using different platforms while the remaining aliquot was reserved. Two aliquots were analyzed by ultra-performance liquid chromatography-tandem mass spectrometry (UPLC-MS/MS) with negative and positive ion mode electrospray ionization (ESI). A third aliquot was also analyzed by UPLC-MS/MS with negative ion mode ESI and a method optimized for polar metabolite detection. The fourth aliquot was analyzed by gas chromatography-mass spectrometry (GC-MS). 1,032 chemical features  were detected utilizing the multiple platforms in the analysis of the plasma samples. Of these, 590 compounds were identified by matching to authentic standards based on retention index, mass to charge ratio, and MS2 data; 73 were identified based on experimental data matched to curated databases; and 369 could not be confidently identified. As the original data dependent acquisition was conducted utilizing both acute event samples and stable heart disease samples, metabolites not detected in the stable heart disease samples were removed. Metabolites missing from greater than 70\% of the samples or without compound identification were also removed, resulting in a final dataset with 522 metabolites across 47 samples. Minimum values were then imputed for the remaining metabolite relative abundances with missing data. As many of the metabolites exhibited approximately log-normal relative abundance distributions, metabolite abundances were log-transformed. Finally, the data was mean centered so that each metabolite’s relative abundance distribution was centered about zero. 

\section{Generation of chemical structure informed priors}
A heatmap representation of the structural similarity between metabolites is shown in Figure~\ref{fig:heatmap}. This heatmap was constructed using agglomerative hierarchical clustering using Ward’s method and squared distances (with distance computed as $d_{ij}=1-s_{ij}$, where $s_{ij}$ is the structural similarity between compounds). For illustrative purposes, the cluster containing cholate was retrieved from the root dendrogram by extracting the branch with height X, as the structural-adaptive BGL subnetwork generated by cholate is considered later. Considering clusters generated by branches with low merge heights (high structural similarity), cholate was a member of a cluster with other closely related compounds such as deoxycholate, 3b-hydroxy-5-cholenoic acid, and glycocholate. Considering the more inclusive cluster generated by the branch at join height x, other members included many intermediates in progestagen, androgen, glucocorticoid, and mineralocorticoid steroid metabolic pathways. These steroid hormone metabolites were all members of a cluster with similar within-cluster distances. Finally, at the same branch height that joined steroid hormone and cholate metabolites, a branch consisting of tocopherols cluster and squalene cluster was also joined.

\newpage
\KOMAoptions{paper=landscape}
\recalctypearea
\begin{figure}[ht]
	\includegraphics[scale=.85]{../Aim2/Plots/StructHeatmaps}
	%\resizebox{\textwidth}{!}{\includegraphics*{../Aim2/Plots/StructHeatmaps}}
	\caption[Add caption]{Add caption \label{fig:heatmap} }
\end{figure}
\newpage
\KOMAoptions{paper=portrait,pagesize}
\recalctypearea

\section{Gibbs sampling}
To approximate the posterior distribution of $p(\boldsymbol{\Omega}|\textbf{X})$, a Markov Chain was generated of length 1,000 with a 250 iteration burn-in period. From each sample from the posterior distribution $p(\boldsymbol{\Omega}|\textbf{X})$, a partial correlation coefficient matrix was computed, yielding a simulated posterior distribution for the matrix of partial correlation coefficients.

\begin{figure}[h!]
	\resizebox{1.18\textwidth}{!}{\includegraphics*{../Aim2/Plots/MCMC}}
	\caption[Add caption]{Time series plots for the MCMC sampler for the shrinkage parameter $\lambda_{ij}$ and for the concentration matrix entry $\omega_{ij}$ for the following metabolite pairs: (cholate, tyrosine), (cholate, cortisone), (cholate, glycochenodeoxycholate).  \label{fig:mcmc} }
\end{figure}

Elements of the MCMC sampling iterations are presented in Figure~\ref{fig:mcmc}. Continuing with the working example of the metabolite cholate, Markov chains are presented for the estimation of the concentration parameters for the pairs (cholate, tyrosine), (cholate, cortisone), and (cholate, glychochenodeoxycholate). These metabolites are highlighted as exemplars of metabolites with relatively low, medium, and relatively high chemical structure similarity with cholate. The time series of the MCMC sampling for the shrinkage parameter, $\lambda_{ij}$, demonstrated differences between the three pairs. Averaged across iterations, more shrinkage was applied with decreasing chemical similarity between the metabolite pairs. While the shrinkage parameter sample values for (cholate, cortisone) tended to be significantly smaller than the sample values for (cholate, cortisone), substantial overlap was observed in the posterior distribution of the concentration parameters for the same pairs.

From the simulated posterior distribution of $\boldsymbol{\Omega}$, the posterior mean $\EE(\boldsymbol{\Omega}|\textbf{X})$ was estimated after discarding burn in iterations. The posterior mean of the distribution of partial correlation coefficients was also computed. The resulting plasma metabolite interactome inferred by the structure-adaptive Bayesian Graphical Lasso is presented in Figure 6. This figure presents both the entire graph representing the posterior mean partial correlations as well as the subgraph generated by considering the neighbors of cholate. For ease of viewing, only edges for which $|\rho_{ij}|>0.05$ are plotted in the presentation of the full graph. 

Positive partial correlation coefficients were observed between cholic acid the following other primary bile acids: glycocholic acid, chenodeoxycholic acid, and glycochenodeoxycholic acid. Negative partial correlation coefficients were observed between cholate and the following: taurocholic acid, taurodeoxycholic acid, and taurochenodeoxycholic acid. In addition, the bile acid 7-Hoca and the bile acid conjugate taurolithocholate 3-sulfate were first neighbors of cholic acid. Multiple conjugated androsterones were observed to be first neighbors of cholic acid as was the glucocorticoid cortisone and the steroidal alkaloid solanidine. Other metabolites that were first neighbors of cholic acid included: 3-Carboxy-4-methyl-5-propyl-2-furanpropionic acid (CMPF), eugenol sulfate, erythritol, 2,3-dihydroxyisolvalerate, threonate, quinate, pimelate, and azelate.
