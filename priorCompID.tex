\begin{DoubleSpace*}
\section{The challenge of identifying compounds from mass spectrometry data}
The identification of compounds from chromatography-coupled mass spectrometry experiments remains a great challenge of untargeted metabolomics \cite{domingo2017,uppal2017,uppal2016,dunn2012}. Many pieces of information may be available to assist in determining the chemical identity of features detected in such experiments including \cite{dunn2012}: the mass-to-charge ratio (m/z) of the ions that have been observed, the fragmentation pattern of either parent or fragment ions, isotopic distribution (e.g. the relative abundance of isotopes can be evaluated by comparing the ratio of 13C to 12C), the observed elution times from separation such as liquid or gas chromatography, and peak shape. Which of these classes of information should be used to identify compounds is largely analytical platform dependent. A further complication of assigning compound identity to the features observed in an experiment is that each compound that elutes from a column and is ionized (such as by electrospray ionization) generates multiple types of ions and adduct ions. In GC-MS analyses the derivatization process is not a uniform process \cite{halket2003}. When electrospray ionization (ESI) is utilized to ionize injected molecules, multiple types of adducted ions may be formed (e.g. sodium adducts) in pronated, non-pronated, or de-pronated form dependent on ESI mode \cite{dunn2012}. Additionally other isotopologue ions may be present \cite{kuhl2011}. 

A system for representing the level of confidence in a compound annotation has been proposed previously as part of the Metabolomics Standards Initiative (MSI) \cite{sumner2007}. The first level (Level 1) represents the most confident type of identification and requires matching two ``orthogonal'' properties of an observed feature with the properties observed from the analysis of an authentic standard under identical analytical conditions. An example of two orthogonal properties for an LC-MS analysis would be observed m/z and retention time or retention index. In an LC-MS/MS experiment observed m/z and MS/MS fragmentation pattern compared to an authentic standard analyzed under identical analytical condition would constitute a Level 1 identification.  A Level 2 identification also requires matching on ``orthogonal'' properties, but in comparison to previously archived data (e.g. databases such as the NIST libraries) as opposed to authentic standards analyzed under identical conditions). A Level 3 identification is more general and requires association of the properties of an observed feature to those of classes of biochemicals as opposed to specific compounds. This framework asserts in a general way that metabolite annotation is a probabilistic process, in which the likelihood of a match from mass feature to compound label depends on both \emph{a priori} knowledge and empirical data. Consequently, a Bayesian approach to compound identification is sensible. 

\section{Prior approaches for ID}


\end{DoubleSpace*}