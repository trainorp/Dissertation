\section{Section 1}
 Blah
 
\section{Section 2}
Blah

\begin{enumerate}
\item Quick enumeration
\item Fake item
\end{enumerate}

\section{Overview on Figures}
The nebula is shown in Fig.~\ref{fig:optical}.

\begin{figure}
\resizebox{\textwidth}{!}{\includegraphics*{ros_rgb.jpg}}
\caption[Rosette Tricolor]{A false color view of the Rosette nebula composited from imagery 
collected during this project. The standard Hubble color scheme is used, with red [SII] emission, green H$\alpha$, 
and blue [OIII] emission. All filters are square root scaled, and relative flux from [OIII] and [SII] has been 
enhanced for clarity. \cite{huber2017}\label{fig:optical} }
\end{figure}


\section{Equations}
Some fake Equations
\begin{align}
n(H^0) \int_{\nu_0}^\infty \frac{4\pi J_{\nu}}{h\nu}a_{\nu} (H^0)d\nu &= n(H^0) \int_{\nu_0}^\infty \phi_{\nu} a_{\nu} (H^0) d\nu  \\
=n(H^0)\Gamma(H^0) &= n_e n_p \alpha(H^0, T).  \label{eq:manylines}
\end{align}

\begin{equation}
\int_{\nu_0}^\infty \frac{L_{\nu}}{h\nu} d\nu =  N_{UV} = (\frac{4\pi}{3})R_S^3 n_e n_p \alpha_B. \label{eq:oneline}
\end{equation}

As Eqs.~\ref{eq:manylines} and \ref{eq:oneline} illustrate
