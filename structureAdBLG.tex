\section{Desired inference}

\section{Sketch of the current methodology and validation}
We evaluated the methodology using simulation studies that follow two different schema. Under the first schema, autoregressive processes were simulated for representing linear biological processes in which the correlation between simulated metabolites decreased in tandem with decreasing structural similarity. Under the second schema, random covariance matrices corresponding to structural similarity were simulated using the C-vines method \cite{lewandowski2009}. In this case, a hierarchical model was used, in which structural similarity was simulated first, followed by abundance distributions in which the correlation between abundances increased with structural similarity. Given both schema, we evaluated the ability of the proposed method to recover the true pairwise conditional correlations structures that were specified in advance. We then evaluate our methodology for generating graphical models for representing the relationships between metabolites detected and quantified from human plasma, and specifically for the development of a reference model for stable coronary artery disease.  

\section{Molecular structure priors and the BGL}
We propose that to incorporate prior knowledge regarding the relatedness of compounds, the scale hyperparameter can be linked to structural similarity, that is by specifying the prior distribution $\lambda_{ij}\sim Gamma(r,s_{ij})$ where $s_{ij}$ is a measure of structural similarity between compound $i$ and compound $j$. The conditional expected value of each $\lambda_{ij}$ is then: $\EE(\lambda_{ij}|\Omega)=(1+r)/(|\omega_{ij} |+s_{ij})$.

To generate informative shrinkage priors for the adaptive Bayesian graphical Lasso, we utilized a local structure similarity metric. This metric was adapted from the previously described Chemically Aware Substructure Search (CASS) algorithm \cite{mitchell2014}. In this adaptation, the structural similarity between any two chemical structures (A and B)  was estimated using strings representing local chemical structure (referred to as the atom’s color) centered at every atom in the two structures. The color of every atom was constructed as follows. First, for every bonded atom, its element type and the order of the bond connecting it to the center atom are joined to form a component string that is added to a list of components. For example, if the center atom has a double bonded oxygen, this would contribute a ``O2'' component to the component list. Every component represents a portion of the local bonded structure at the center atom. Second, the components strings are then sorted alphanumerically and concatenated to produce a description of the bonded structure one bond away from the center atom. Finally, to the front of this string, the element type of the center atom is then added to yield the atom’s color. Each color uniquely maps to a single locally bonded structure (e.g. the ``CC1O1O2'' coloring represents a carbon of a carboxylate). Since the component list was first sorted alphanumerically, this color is consistent for all identical local structures regardless of how they are ordered in their representation. Each chemical structure can be represented as the list of its constituent atom’s colors and these lists of colors can be compared to determine structural similarity. To determine structural similarity between compounds using the color string representations, the Tanimoto coefficient between pairs of compounds was computed, which is defined as \cite{chen2002}:
\begin{align}
	s(A,B)=\frac{\sum_{i=1}^m \min \left(n_i(A),n_i(B) \right)}{\sum_{i=1}^m n_i(A)+\sum_{i=1}^m n_i(B)-\sum_{i=1}^m \min \left(n_i(A),n_i(B) \right)}
\end{align}
where $n_i(A)$ represents the count of unique atom pairs indexed by $i=1,2,\hdots,m$ for molecule $A$. The Tanimoto dissimilarity is then $d(A,B)=1-s(A,B)$. 

\begin{figure}[ht]
	\resizebox{1.1\textwidth}{!}{\includegraphics*{../Aim2/Plots/lambdasVsSim}}
	\caption[Structural similarity versus shrinkage]{Theoretical relationship between the structural similarity of two metabolites and the expected value of the shrinkage parameter $\lambda_{ij}$. On the vertical axis... \label{fig:simShrink} }
\end{figure}

After determining the Tanimoto dissimilarity between each pair of metabolites, the gamma hyperprior distribution for the shrinkage parameter $\lambda$ can be determined by linking the gamma distribution shape to the dissimilarity, that is by setting $s_{ij}=f(1-d(i,j))$ where $i,j$ index metabolites and $f(x)$ is a monotonic function. The conditional distribution of the shrinkage parameter is then $\lambda_{ij} |\Omega \sim Gamma(1+r,|\omega_{ij} |+s_{ij} )$. A plot of the relationship between the structural similarity of two hypothetical metabolites and the expected value of the shrinkage parameter is shown in  Fig.~\ref{fig:simShrink}. 