\begin{DoubleSpace*}
\section{Summary}
Collectively the current work covers three subtopics in the field of Metabolomics: (1) making higher order inferences in metabolomics using untargeted mass spectrometry data, (2) developing diagnostic models for discriminating disease phenotypes, and (3) determining the identity of compounds in LC-MS data. For the first two subtopics we have introduced novel methodology, while we have only provided a review of a class of methods in the third. What unites these three disparate aims into a cohesive work is the use of Bayesian statistical methodology for integrating extra-experimental scientific knowledge for solving high-dimensional challenges in metabolomics.

To introduce our novel methodology for making higher order inferences in metabolomics (Chapter~\ref{structureADBGL}) and to apply this methodology to characterizing the probabilistic interaction of metabolites given stable heart disease (Chapter~\ref{hdInteractome}) we provided some preliminary background. In Chapter~\ref{prelimMetab} we provided a cursory overview of metabolism, the link between metabolism and multiple diseases, and the use of the techniques of metabolomics for studying metabolism and human diseases. In Chapter~\ref{prelimBayesian} we provided an introduction to Bayesian reasoning and in Chapter~\ref{prelimGGM} we introduced Gaussian Graphical Models (GGMs) and both frequentist and Bayesian approaches for the estimation of such models. These preliminaries were necessary foundations for developing our novel methodology for using molecular structure to generate informative priors for estimating GGMs for making higher order inferences in metabolomics. This section culminated in a plasma metabolite interactome for stable heart disease (Chapter~\ref{hdInteractome}). In this section we highlighted how higher order inferences can be made by focusing on the primary bile acid cholate. We focused on cholate as dysregulated bile acid metabolism is a component of atherosclerosis, and thus heart disease. Interestingly, we observed that our methodology balanced both the molecular structure prior information as well as the empirical data. Relatively low penalization of edges between cholate and steroids hormones, and cholate and secondary bile acids was observed, however the posterior distribution showed higher posterior probability of strong edges between cholate and secondary bile acids.

In the second part of the current work, we focused on developing a statistical classifier capable of discriminating between three phenotypes: stable coronary artery disease (CAD), non-thrombotic myocardial infarction (MI), and thrombotic MI (Chapter~\ref{diagnostic}). We constructed two models, the first based on metabolite abundances only, and the second with the inclusion of clinical troponin values as these values would be available at presentation as well. We observed that a Bayesian multinomial logistic regression model demonstrated positive performance characteristics in discriminating between the phenotypes. Both Bayesian models performed better than the frequentist model with the same variables fit by maximum likelihood estimation. We hypothesize that the superior performance of the Bayesian models is due to a reduced likelihood of overfitting. 

In the final section (Chapter~\ref{bayesianID}) we discuss Bayesian approaches for compound identification given LC-MS data. The first two methods that we discussed utilized a Dirichlet-categorical prior over possible compound label to mass feature assignments. The third method relied instead on Gaussian mixture modeling over mass features. While only three Bayesian approaches for compound identification have been developed to date (to our knowledge), these methods have increased in sophistication over time. The first method described primarily included a likelihood term based in the measurement error of mass-to-charge ratio (although retention time error and isotope profile error were discussed). The prior probability distribution utilized possible transformations that could account for the dependence between compounds observed. The next method was largely similar, with a more complex measurement error model and the ability to use metabolite databases (e.g. KEGG) for generating priors. Finally, the third method recast the identification problem as a clustering followed by identification problem, which is sensible given that the mass features occur in clusters comprised of one compound in multiple adduct / ionic forms. 

\section{Future directions: Bayesian interactome models}
Through the development and utilization of our novel methodology for estimating metabolite interactomes future directions have been suggested which will be pursued. First, the prior distributions of the concentration matrix parameters are continuous with all real numbers as the support. As a result, the posterior probability density for each concentration matrix parameter has the same support and the probability of a concentration matrix parameter being equal to zero is zero. Consequently, edge selection, or the process of setting some concentration matrix parameters equal to zero must happen after model selection. This suggests that a hierarchical model or a model that allows probability mass at zero would be beneficial. In the first case, a hierarchical model could include a Bernoulli component that models whether an edge between metabolites is present or absent, with a continuous prior distribution given that an edge is present. In the second case, a ``slab and spike'' prior could be utilized, which has a point mass at zero and more conventional continuous tails.

\section{Future directions: Diagnostic modeling}

\end{DoubleSpace*}
