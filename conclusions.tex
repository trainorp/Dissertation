\begin{DoubleSpace*}
\section{Summary}
Collectively the current work covers three subtopics in the field of Metabolomics: (1) making higher order inferences in metabolomics using untargeted mass spectrometry data, (2) developing diagnostic models for discriminating disease phenotypes, and (3) determining the identity of compounds in LC-MS data. For the first two subtopics we have introduced novel methodology, while we have only provided a review of a class of methods in the third. What unites these three disparate aims into a cohesive work is the use of Bayesian statistical methodology for integrating extra-experimental scientific knowledge for solving high-dimensional challenges in metabolomics.

To introduce our novel methodology for making higher order inferences in metabolomics (Chapter~\ref{structureADBGL}) and to apply this methodology to characterizing the probabilistic interaction of metabolites given stable heart disease (Chapter~\ref{hdInteractome}) we provided some preliminary background. In Chapter~\ref{prelimMetab} we provided a cursory overview of metabolism, the link between metabolism and multiple diseases, and the use of the techniques of metabolomics for studying metabolism and human diseases. In Chapter~\ref{prelimBayesian} we provided an introduction to Bayesian reasoning and in Chapter~\ref{prelimGGM} we introduced Gaussian Graphical Models (GGMs) and both frequentist and Bayesian approaches for the estimation of such models. These preliminaries were necessary foundations for developing our novel methodology for using molecular structure to generate informative priors for estimating GGMs for making higher order inferences in metabolomics. This section culminated in a plasma metabolite interactome for stable heart disease (Chapter~\ref{hdInteractome}). In this section we highlighted how higher order inferences can be made by focusing on the primary bile acid cholate. We focused on cholate as dysregulated bile acid metabolism is a component of atherosclerosis, and thus heart disease. Interestingly, we observed that our methodology balanced both the molecular structure prior information as well as the empirical data. Relatively low penalization of edges between cholate and steroids hormones, and cholate and secondary bile acids was observed, however the posterior distribution showed higher posterior probability of strong edges between cholate and secondary bile acids.

In the second part of the current work, we focused on developing a statistical classifier capable of discriminating between three phenotypes: stable coronary artery disease (CAD), non-thrombotic myocardial infarction (MI), and thrombotic MI (Chapter~\ref{diagnostic}). 

\section{Future directions: Bayesian interactome models}

\section{Future directions: Diagnostic modeling} 

\end{DoubleSpace*}
